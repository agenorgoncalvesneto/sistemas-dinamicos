\documentclass[a4paper, 11pt]{article}

\usepackage[utf8]{inputenc}
\usepackage[brazil]{babel}

\usepackage[left=3cm, top=3cm, right=2cm, bottom=2cm]{geometry}
\usepackage[onehalfspacing]{setspace}
\usepackage{indentfirst}

\usepackage{subfiles}
\usepackage{enumitem}
\usepackage{graphicx}

\usepackage{amsfonts}
\usepackage{amsmath}
\usepackage{amssymb}
\usepackage{amsthm}

\theoremstyle{definition}
\newtheorem{definition}{Definição}[section]
\newtheorem{example}[definition]{Exemplo}

\theoremstyle{plain}
\newtheorem{affirmation}[definition]{Afirmação}
\newtheorem{corollary}[definition]{Corolário}
\newtheorem{lemma}[definition]{Lema}
\newtheorem{proposition}[definition]{Proposição}
\newtheorem{theorem}[definition]{Teorema}

\newcommand{\RR}{\mathbb{R}}
\newcommand{\basin}{\mathcal{B}}
\newcommand{\class}{\mathcal{C}}
\newcommand{\orbit}{\mathcal{O}}
\newcommand{\schwarz}{\mathcal{S}}

\DeclareMathOperator{\per}{Per}
\DeclareMathOperator{\sen}{sen}

\title{Uma Introdução aos Sistemas Dinâmicos Caóticos}
\author{Agenor Gonçalves Neto}
\date{São Paulo, 2020}

\begin{document}

\maketitle
\tableofcontents

\subfile{sections/conceitos-elementares}
\subfile{sections/familia-quadratica}
\subfile{sections/estudo-inicial}
\subfile{sections/conjuntos-de-cantor}
\subfile{sections/caos}
\subfile{sections/conjugacao-topologica}
\subfile{sections/dinamica-simbolica}
\subfile{sections/matriz-de-transicao}
\subfile{sections/bifurcacao}
\subfile{sections/teorema-de-sharkovsky}
\subfile{sections/teorema-de-singer}

\end{document}
