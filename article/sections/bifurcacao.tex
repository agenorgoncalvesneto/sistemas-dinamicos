\section{Bifurcação}

Ao longo de seção, $f_\lambda$ representará uma família parametrizada de funções no parâmetro $\lambda$ de modo que a função
$$G(x, \lambda) = f_\lambda(x),$$
definida num aberto de $\RR^2$, seja de classe $\CC^\infty$ nas variáveis $x$ e $\lambda$.

\begin{theorem}
\label{theorem1}
Seja $f_\lambda$ uma família parametrizada de funções.
Suponha que
\begin{enumerate}
\item $f_{\lambda_0}(x_0) = x_0$,
\item $f'_{\lambda_0}(x_0) \neq 1$. 
\end{enumerate}
Então existem vizinhanças $I$ e $J$ de $\lambda_0$ e $x_0$, respectivamente, e uma função $p: I \to J$ de classe $\CC^\infty$ tais que
\begin{enumerate}
\item $p(\lambda_0) = x_0$, 
\item $f_\lambda(p(\lambda)) = p(\lambda)$ para todo $\lambda \in I$.
\end{enumerate}
Além disso, $f_\lambda$ não possui outros pontos fixos em $J$.
\end{theorem}

\begin{proof}
Basta aplicar o Teorema da Função Implícita para a função $G(x, \lambda) = f_\lambda(x) - x$ no ponto $(x_0, \lambda_0)$.
\end{proof}

De acordo com o Teorema anterior, se $x_0$ é um ponto fixo hiperbólico de $f_{\lambda_0}$, então $f_\lambda$ possui um único ponto fixo numa vizinhança de $x_0$ para cada $\lambda$ numa vizinhança de $\lambda_0$.
 
Utilizando a notação do Teorema anterior, considere a função $g_\lambda(x) = f_\lambda(x + p(\lambda)) - p(\lambda)$. Observe que $g_\lambda(0) = f(p(\lambda)) - p(\lambda) = 0$ para todo $\lambda \in I$, ou seja, $0$ é ponto fixo de $g_\lambda$ para todo $\lambda \in I$. Além disso, $f_\lambda$ e $g_\lambda$ são topologicamente conjugadas por $h_\lambda(x) = x - p(\lambda)$.

\begin{theorem}[Bifurcação Tangente]
Suponha que
\begin{enumerate}
\item $f_{\lambda_0}(0) = 0$,
\item $f'_{\lambda_0}(0) = 1$,
\item $f''_{\lambda_0}(0) \neq 0$,
\item $\frac{\partial f_\lambda}{\partial \lambda} |_{\lambda = \lambda_0}(0) \neq 0$.
\end{enumerate}
Então existem uma vizinhança $I$ de $0$ e uma função $p: I \to \RR$ de classe $\CC^\infty$ tais que
\begin{enumerate}
\item $p(0) = \lambda_0$,
\item $f_{p(x)}(x) = x$.
\end{enumerate}
Além disso, $p'(0) = 0$ e $p''(0) \neq 0$.
\end{theorem}

\begin{proof}
Aplicando o TFI para a função $G(x, \lambda) = f_\lambda(x) - x$ no ponto $(0, \lambda_0)$, existe uma função $p: I \to \RR$ de classe $\class^\infty$, onde $I$ é uma vizinhança de $0$, tal que $p(0) = \lambda_0$ e $G(x, p(x)) = 0$ para todo $x \in I$.

Além disso, pela Regra da Cadeia, temos que
$$p'(0) = - \frac{\frac{\partial G}{\partial x}(0, \lambda_0)}
{\frac{\partial G}{\partial \lambda}(0, \lambda_0)} = - \frac{f'_{\lambda_0}(0) - 1}{\frac{\partial f_\lambda}{\partial \lambda}|_{\lambda = \lambda_0}(0)} = 0$$
e
$$ p''(0) = - \frac{\frac{\partial^2 G}{\partial x^2}(0, \lambda_0)\frac{\partial G}{\partial \lambda}(0, \lambda_0) - \frac{\partial G}{\partial x}(0, \lambda_0) \frac{\partial^2 G}{ \partial x \partial \lambda}(0, \lambda_0)}
{\left( \frac{\partial G}{\partial \lambda}(0, \lambda_0) \right)^2}  = - \frac{\frac{\partial^2 G}{\partial x^2}(x, \lambda_0)}{ \frac{\partial f_\lambda}{\partial \lambda}|_{\lambda = \lambda_0}(0)} \neq 0.$$
\end{proof}

No Teorema anterior, se $p''(0) > 0$, então a concavidade de $p$ é para cima. Esboçando o gráfico de $p$, podemos observar que $f$ não possui pontos fixos para $\lambda < \lambda_0$, possui um único ponto fixo para $\lambda = \lambda_0$ e possui dois pontos fixos para $\lambda > \lambda_0$. Se $p''(0) < 0$, a concavidade de $p$ é para baixo e a conclusão é análoga, invertendo os sentidos.

\begin{theorem}[Bifurcação com Duplicação de Período]
Suponha que
\begin{enumerate}
\item $f_{\lambda}(0) = 0$ para todo $\lambda$ numa vizinhança de $\lambda_0$,
\item $f'_{\lambda_0}(0) = -1$,
\item $\left. \frac{\partial (f^2_\lambda)'}{\partial \lambda} \right|_{\lambda = \lambda_0}(0) \neq 0$,
\item $(\schwarz f_{\lambda_0})(0) \neq 0$.
\end{enumerate}
Então existem uma vizinhança $I$ de $0$ e uma função $p: I \to \RR$ de classe $\CC^\infty$ tais que 
\begin{enumerate}
\item $p(0) = \lambda_0$,
\item $f_{p(x)}(x) \neq x$ para todo $x \in I - \lbrace 0 \rbrace$,
\item $f^2_{p(x)}(x) = x$ para todo $x \in I$.
\end{enumerate}
Além disso, $p'(0) = 0$ e $p''(0) \neq 0$.
\end{theorem}

\begin{proof}
Seja $G(x, \lambda) = f^2_\lambda (x) - x$. Desse modo, a função
\[ H(x, \lambda) =
    \begin{cases} 
      \dfrac{G(x, \lambda)}{x} & \textrm{ se } x \neq 0 \\
      \\
      \dfrac{\partial G}{\partial x}(0, \lambda) & \textrm{ se } x = 0
   \end{cases}
\]
é de classe $\CC^\infty$ e são válidas as igualdades
$$\frac{\partial H}{\partial x}(0, \lambda_0) = \frac{1}{2}\frac{\partial^2 G}{\partial x^2}(0, \lambda_0) \quad \text{ e } \quad \frac{\partial^2 H}{\partial x^2}(0, \lambda_0) = \frac{1}{3} \frac{\partial^3 
G}{\partial x^3}(0, \lambda_0).$$
De fato, fixado $\lambda$ numa vizinhança de $\lambda_0$, basta dividir por $x \neq 0$ a série de Taylor de $G(x, \lambda)$ em torno de $x_0 = 0$ para obter a série de Taylor de $H(x, \lambda)$ em torno de $x_0 = 0$.

Além disso, é imediato que  $H(0, \lambda_0) = 0$ e $\frac{\partial H}{\partial \lambda}(0, \lambda_0) \neq 0$.
Desse modo, pelo TFI, existe uma função $p: I \to \RR$ de classe $\CC^\infty$, onde $I$ é uma vizinhança de $0$, tal que $p(0) = \lambda_0$ e $H(x, p(x)) = 0$ para todo $x \in I$. Em particular, se $x \neq 0$, então
$$0 = \frac{G(x, p(x))}{x} = \frac{f^2_{p(x)}(x) - x}{x}$$
e, portanto, $f^2_{p(x)}(x) = x$ para todo $x \in I$. Além disso, pelo Teorema \ref{theorem1}, $f_\lambda$ possui um único ponto fixo numa vizinhança de $0$ e, portanto, podemos considerar que $f_{p(x)}(x) \neq x$ para todo $x \in I$, $x \neq 0$.

Por fim, utilizando a Regra da Cadeia, podemos mostrar que
$$\frac{\partial^2 G}{\partial x^2}(0, \lambda_0) = 0 \quad \text{ e } \quad \frac{\partial^3 G}{\partial x^3}(0, \lambda_0) = 2 (\schwarz f_{\lambda_0})(0).$$ 
e, portanto, concluímos que
$$p'(0) = -\frac{\frac{\partial H}{\partial x}(0, \lambda_0) }{\frac{\partial H }{\partial \lambda}(0, \lambda_0)} = -\frac{1}{2} \frac{\frac{\partial^2 G}{\partial x^2}(0, \lambda_0) }{\frac{\partial H }{\partial \lambda}(0, \lambda_0)} = 0$$
e
$$ p''(0) = - \frac{\frac{\partial^2 H}{\partial x^2}(0, \lambda_0)\frac{\partial H}{\partial \lambda}(0, \lambda_0)}{ \left( \frac{ \partial H}{\partial \lambda}(0, \lambda_0) \right)^2} \\
= -\frac{1}{3} \frac{\frac{\partial^3 G}{\partial x^3}(0, \lambda_0)}{\frac{\partial H}{\partial \lambda}(0, \lambda_0)} = -\frac{2}{3} \frac{S_{f_{\lambda_0}}(0)}{\frac{\partial H}{\partial \lambda}(0, \lambda_0)} \neq 0.$$
\end{proof}