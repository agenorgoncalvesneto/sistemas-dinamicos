\subsection{Bifurcação}

A família quadrática exibe outro fenômeno que ocorre em sistemas dinâmicos: a bifurcação. Esse fenômeno ajuda a explicar como a dinâmica de $h$, que é simples para $\mu$ pequeno, se torna caótica para $\mu$ suficientemente grande.

Seja $f_\lambda$ uma família parametrizada de funções no parâmetro $\lambda$ de modo que a função
$$G(x, \lambda) = f_\lambda(x),$$
definida num aberto de $\RR^2$, seja de classe $\class^\infty$ nas variáveis $x$ e $\lambda$.
Dizemos que $f_\lambda$ sofre uma bifurcação em $\lambda_0$ se existe $\varepsilon > 0$ com a seguinte propriedade: se $\lambda_1 \in (\lambda_0 - \varepsilon, \lambda_0)$ e $\lambda_2 \in (\lambda_0, \lambda_0 + \varepsilon)$, então a dinâmica de $f_{\lambda_1}$ e $f_{\lambda_2}$ são diferentes, isto é, $f_{\lambda_1}$ e $f_{\lambda_2}$ não são topologicamente conjugadas.
Por exemplo, ocorre uma bifurcação quando altera a estrutura dos pontos periódicos conforme o parâmetro varia.

\begin{example}
\begin{enumerate}[label=\alph*)]\item[]
\item 
Seja $E_\lambda$ a família de funções dadas por $E_\lambda(x) = \lambda e^x$, onde $\lambda > 0$. Graficamente, podemos observar que $E_\lambda$ sofre uma bifurcação em $\lambda_0 = \frac{1}{e}$.

Se $\lambda > \frac{1}{e}$, então $E_\lambda$ não possui pontos fixos; se $\lambda = \frac{1}{e}$, então $E_\lambda$ possui um único ponto fixo em $x = 1$; e se $\lambda < \frac{1}{e}$, então $E_\lambda$ possui dois pontos fixos.

\item Seja $h_\mu$ a família quadrática. Graficamente, podemos observar que $h_\mu$ sofre uma bifurcação em $\mu_0 = 3$.

Se $\mu < 3$, então $p_\mu$ é um ponto fixo atrator e se $\mu >3$, então $p_\mu$ é um ponto fixo repulsor. Observando o gráfico de $h^2$ é possível ver que, conforme $\mu$ se torna maior que $3$, nasce uma órbita de $h$ de período $2$. Essa é a chamada bifurcação com duplicação de período.

\end{enumerate}
\end{example}

Observe nos exemplos que $E_{\lambda_0}'(1) = 1$ e $h_{\mu_0}(p_\mu) = -1$, ou seja, as bifurcações ocorreram quando a derivada em módulo no ponto fixo se tornou igual à $1$. O teorema a seguir mostra que isso não é coincidência.

\begin{theorem}
\label{theorem1}
Seja $f_\lambda$ uma família parametrizada de funções.
Suponha que
\begin{enumerate}
\item $f_{\lambda_0}(x_0) = x_0$,
\item $f'_{\lambda_0}(x_0) \neq 1$. 
\end{enumerate}
Então existem vizinhanças $I$ e $J$ de $\lambda_0$ e $x_0$, respectivamente, e uma função $p: I \to J$ de classe $\class^\infty$ tais que
\begin{enumerate}
\item $p(\lambda_0) = x_0$, 
\item $f_\lambda(p(\lambda)) = p(\lambda)$ para todo $\lambda \in I$.
\end{enumerate}
Além disso, $f_\lambda$ não possui outros pontos fixos em $J$.
\end{theorem}

\begin{proof}
Basta aplicar o Teorema da Função Implícita para a função $G(x, \lambda) = f_\lambda(x) - x$ no ponto $(x_0, \lambda_0)$.
\end{proof}

De acordo com o Teorema anterior, se $x_0$ é um ponto fixo hiperbólico de $f_{\lambda_0}$, então $f_\lambda$ possui um único ponto fixo numa vizinhança de $x_0$ para cada $\lambda$ numa vizinhança de $\lambda_0$.

Vamos olhar com mais detalhes a bifurcação com duplicação de período que ocorre na família quadrática.
Inicialmente, quando $\mu$ se torna maior que $2$, existe $p_\mu'$ menor que $p_\mu$ tal que $h(p_\mu') = p_\mu$. Vamos desenhar o gráfico de $h^2$ juntamente com um quadrado de lados $(p_\mu', p_\mu)$, $(p_\mu, p_\mu)$, $(p_\mu, p_\mu')$ e $(p_\mu', p_\mu')$.

Observe que o gráfico de $h^2$ restrito aos quadrados, após uma rotação de $2 \pi$, se assemelha ao gráfico da própria $h$ no intervalo $[0, 1]$. Vamos deixar essa ideia mais precisa através do operador de renormalização.

Seja $L_\mu: [p_\mu', p_\mu] \to [0, 1]$ a função dada por
$$L_\mu(x) = \frac{1}{p_\mu' - p_\mu} (x - p_\mu).$$
Observe que $L_\mu(p_\mu) = 0$ e $L_\mu(p_\mu') = 1$. A inversa de $L_\mu$ é a função $L_\mu^{-1}: [0, 1] \to [p_\mu', p_\mu]$ dada por $L_\mu^{-1}(x) = (p_\mu' - p_\mu)x + p_\mu$. Desse modo, definimos a renormalização de $h$ como a função $Rh: [0, 1] \to [0, 1]$ dada por $(Rh)(x) = L_\mu \circ h^2 \circ L_\mu^{-1}(x)$. A renormalização de $h$ possui algumas semelhanças com $h$.

\begin{proposition}
\begin{enumerate}\item[]
\item $(Rh)(0) = (Rh)(1) = 0$.
\item $\frac{1}{2}$ é o único ponto crítico de $Rh$.
\end{enumerate}
\end{proposition}

Observe que um ponto fixo de $Rh$ está unicamente relacionado com um ponto periódico de $h$ de período $2$. Além disso, o gráfico de $Rh$ não está contido em $[0, 1]$ para algum $\mu < 4$. Desse modo, podemos fazer uma análise análoga para concluir que $Rh$ passa por uma bifurcação com duplicação de período. 
