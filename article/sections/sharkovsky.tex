\section{Teorema de Sharkovsky}

Ao longo dessa seção, consideraremos $f: \RR \to \RR$ uma função contínua.
Além disso, escreveremos $J_0 \longrightarrow J_1 \longrightarrow \cdots \longrightarrow J_n$ quando $J_0, \, J_1, \, \dots, \, J_n$ são intervalos fechados e $f(J_k) \supset J_{k+1}$ para todo $0 \leq k < n$.

\begin{lemma}
Se $J_0 \longrightarrow J_1$, então existe um intervalo fechado $J_0' \subset J_0$ tal que $f(J_0') = J_1$.
\end{lemma}

\begin{proof}
Sejam $p, \, q \in [a, b]$ tais que $f(p) = c$ e $f(q) = d$, onde $J_0 = [a, b]$ e $J_1 = [c, d]$.
Se $p \leq q$, definimos $b' = \inf \{ x \in [p, q] : f(x) = d \}$ e $a' = \sup \{ x \in [p, b'] : f(x) = c \}$ e, pela continuidade de $f$, podemos concluir que $f(J_0') = J_1$, onde $J_0' = [a', b']$.
Se $q \leq p$, a demonstração é análoga.
\end{proof}

\begin{lemma}
Se $J_0 \longrightarrow J_1 \longrightarrow \cdots \longrightarrow J_{n-1} \longrightarrow J_0$, então existe $p \in J_0$ tal que as seguintes condições são válidas:
\begin{enumerate}
\item $f^k(p) \in J_k$ para todo $1 \leq k < n$.
\item $f^n(p) = p$.
\end{enumerate}
\end{lemma}

\begin{proof}
Pelo Lema anterior, podemos construir uma sequência de intervalos fechados $J_0', \, J_1', \, \dots, \, J_{n-1}'$ com as seguintes propriedades:
\begin{enumerate}[label=\alph*.]
\item $J_0 \supset J_0' \supset J_1' \supset \cdots \supset J_{n-1}'$.
\item $f^k(J_{k-1}') = J_k$ para todo $1 \leq k < n$.
\item $f^n(J_{n-1}') = J_0$.
\end{enumerate}

Desse modo, existe $p \in J_{n-1}'$ tal que $f^n(p) = p$.
Em particular, $p \in J_0$ e $f^k(p) \in J_k$ para todo $1 \leq k < n$.
\end{proof}

\begin{theorem}
Se $\per_3(f) \neq \emptyset$, então $\per_k(f) \neq \emptyset$ para todo $k \geq 1$.
\end{theorem}

\begin{proof}
Sejam $p_1 < p_2 < p_3$ os pontos da órbita de um elemento de $\per_3(f)$ e suponha que $f(p_1) = p_2$ e $f(p_2) = p_3$.
Se $f(p_1) = p_3$ e $f(p_3) = p_2$, a demonstração é análoga.
Definindo $J_0 = [p_1, p_2]$ e $J_1 = [p_2, p_3]$, temos que $J_0 \longrightarrow J_1$, $J_1 \longrightarrow J_0$ e $J_1 \longrightarrow J_1$. Com isso, podemos demonstrar as seguintes afirmações:

\begin{enumerate}[label=\alph*.]
\item $\per_1(f) \neq \emptyset$.

De fato, $J_1 \longrightarrow J_1$ implica que existe $p \in J_1$ tal que $f(p) = p$.

\item $\per_2(f) \neq \emptyset$.

De fato, $J_0 \longrightarrow J_1 \longrightarrow J_0$ implica que existe $p \in J_0$ tal que $f(p) \in J_1$ e $f^2(p) = p$. Se $f(p) = p$, então $p \in J_0 \cap J_1$, o que é um absurdo, pois $J_0 \cap J_1 = \{ p_2 \}$ e $p_2 \in \per_3(f)$.

\item $\per_n(f) \neq \emptyset$, $n \geq 4$.

Se $J_1 \longrightarrow \cdots \longrightarrow J_1 \longrightarrow J_0 \longrightarrow J_1$ é um ciclo de tamanho $n$, existe $p \in J_1$ tal que $f^k(p) \in J_1$, para todo $k = 1, \dots, n-2$, $f^{n-1}(p) \in J_0$ e $f^n(p) = p$. Se $f^{n-1}(p) = p$, então $p \in J_0 \cap J_1 = \{p_2\}$, o que é um absurdo pois implica que $f(p) = p_3 \in J_0$. Se $f^k(p) = p$ para algum $k = 1, \dots, n-2$ implica que $f^k(p) \in J_1$, para todo $k \geq 1$. Em particular, $f^{n-1}(p) \in J_0 \cap J_1 = \{p_2\}$ e, portanto, $p = f^n(p) = p_3$, o que é um absurdo pois implica que $f(p) = p_1 \in J_1$. 
\end{enumerate}






\begin{enumerate}[label=\alph*.]
\item $n = 1$: Como $J_1 \longrightarrow J_1$, existe $p \in J_1$ tal que $f(p) = p$.
\item $n = 2$: Como $J_0 \longrightarrow J_1 \longrightarrow J_0$, existe $p \in J_0$ tal que $f(p) \in J_1$ e $f^2(p) = p$. Se $f(p) = p$, então $p \in J_0 \cap J_1 = \{p_2\}$, o que é um absurdo pois $p_2$ possui período principal 3. Desse modo, o período principal de $p$ é $2$.
\item $n > 3$: Se $J_1 \longrightarrow \cdots \longrightarrow J_1 \longrightarrow J_0 \longrightarrow J_1$ é um ciclo de tamanho $n$, existe $p \in J_1$ tal que $f^k(p) \in J_1$, para todo $k = 1, \dots, n-2$, $f^{n-1}(p) \in J_0$ e $f^n(p) = p$. Se $f^{n-1}(p) = p$, então $p \in J_0 \cap J_1 = \{p_2\}$, o que é um absurdo pois implica que $f(p) = p_3 \in J_0$. Se $f^k(p) = p$ para algum $k = 1, \dots, n-2$ implica que $f^k(p) \in J_1$, para todo $k \geq 1$. Em particular, $f^{n-1}(p) \in J_0 \cap J_1 = \{p_2\}$ e, portanto, $p = f^n(p) = p_3$, o que é um absurdo pois implica que $f(p) = p_1 \in J_1$. 
\end{enumerate}
\end{proof}

\begin{definition}[Ordenação de Sharkovsky]
$$3 \,\triangleright\, 5 
\,\triangleright\, \cdots \,\triangleright\,
2 \cdot 3 \,\triangleright\, 2 \cdot 5 
\,\triangleright\, \cdots \,\triangleright\,
2^2 \cdot 3 \,\triangleright\, 2^2 \cdot 5
\,\triangleright\, \cdots \,\triangleright\,
2^k \cdot 3 \,\triangleright\, 2^k \cdot 5
\,\triangleright\, \cdots \,\triangleright\,
2^2 \,\triangleright\, 2 \,\triangleright\, 1.$$
\end{definition}

\begin{theorem}[Sharkovsky]
Se $f$ admite ponto de período principal $n$, então $f$ admite ponto de período principal $m$, para todo $m \triangleleft n$.
\end{theorem}

\begin{theorem}
Para todo $n \geq 1$ existe uma função $f$ que admite ponto periódico de período principal $n$ e que não admite ponto de período principal $m$ se $m \,\triangleright\, n$.
\end{theorem}

\begin{proof}
Seja $T: [0,1] \to [0,1]$ a função dada por $T(x) = 1 - |2x - 1|$ e considere a família de funções $T_h(x) = \textrm{min}\{h, T(x)\}$ definidas em $[0,1]$, com o parâmetro $h$ variando em $[0,1]$. Observe que $T_1 = T$, pois $T(x) \leq 1$ para todo $x \in [0,1]$. Além disso, observando o gráfico de $T_1$ concluímos que a função possui $2^k$ pontos periódicos de período $k$ e assim podemos definir, para cada $k \geq 1$, $$h(k) = \textrm{min} \{ \textrm{max} \{ \mathcal{O} : \mathcal{O} \textrm{ é uma órbita de tamanho } n \textrm{ de } T_1\} \}$$

A ideia principal da prova consiste no fato de que $h(k)$ desempenha os papéis de parâmetro, máximo e ponto de uma órbita de $T_{h(k)}$. As seguintes afirmações tornarão preciso esse fato.

\begin{enumerate}[label = (\alph*)]
\item Se $\mathcal{O} \subset [0, h)$ é uma órbita de $T_h$, então $\mathcal{O}$ é uma órbita de $T_1$.

Se $p \in \mathcal{O}$ então $T_h(p) \in [0, h)$. Desse modo, $T_h(p) = \textrm{min}\{h, T(p)\} = T(p) = T_1(p)$, ou seja, $T_h$ e $T_1$ coincidem em $\mathcal{O}$ e, portanto, $\mathcal{O}$ é uma órbita de $T_1$.

\item Se $\mathcal{O} \subset [0, h]$ é uma órbita de $T_1$, então $\mathcal{O}$ é uma órbita de $T_h$.

Se $p \in \mathcal{O}$ então $T_1(p) \in [0, h]$. Desse modo, $T_h(p) = \textrm{min}\{h, T(p)\} = \textrm{min}\{h, T_1(p)\} = T_1(p)$, ou seja, $T_h$ e $T_1$ coincidem em $\mathcal{O}$ e, portanto, $\mathcal{O}$ é uma órbita de $T_h$.

\item $T_{h(k)}$ possui uma órbita $\mathcal{O} \in [0, h(k))$ de tamanho $l$ se e somente se $h(k) > h(l)$.

Se $T_{h(k)}$ possui uma órbita $\mathcal{O} \in [0, h(k))$ de tamanho $l$, então $\mathcal{O}$ é uma órbita de $T_1$ por (a) e, pela definição de $h(l)$, concluímos que $h(l) < h(k)$.

Por outro lado, se $h(l) < h(k)$, então $T_1$ possui uma órbita $\mathcal{O} \subset [0, h(l)] \subset [0, h(k)]$ de tamanho $l$ e, desse modo, $\mathcal{O}$ é uma órbita de $T_{h(k)}$ por (b).

\item A órbita de $T_1$ que contém $h(k)$ é uma órbita de tamanho $k$ de $T_{h(k)}$. Além disso, todas as outras órbitas de $T_{h(k)}$ estão em $[0, h(k))$. 

Pela definição de $h(k)$, $T_1$ possui uma órbita $\mathcal{O} \subset [0, h(k)]$ de tamanho $k$ e, portanto, $\mathcal{O}$ é uma órbita de $T_{h(k)}$ por (b).

Para demonstrar a segunda parte, basta observar que $h(k)$ é o valor máximo de $T_{h(k)}$ e, desse modo, toda órbita de $T_{h(k)}$ está contida em $[0, h(k)]$. Em particular, se a órbita não contém $h(k)$, então ela está contida em $[0, h(k))$.

\item $k \,\triangleright\, l$ se o somente se $h(k) > h(l)$.

Suponha que $k \,\triangleright\, l$. Por (d), $T_{h(k)}$ possui uma órbita de tamanho $k$. De acordo com o Teorema de Sharkovsky e com (d), $T_{h(k)}$ admite uma órbita de tamanho $l$ contida em $[0, h(k))$. Desse modo,  $h(k) > h(l)$ por (c).

Por outro lado, suponha que $h(k) > h(l)$. Caso $l \,\triangleright\, k$, a demonstração no parágrafo anterior implicaria que $h(k) < h(l)$, contrariando a hipótese. Desse modo, $k \,\triangleright\, l$.
\end{enumerate}

Assim, para cada $n \geq 1$, $T_{h(n)}$ possui órbita de tamanho $n$. Além disso, se $m \,\triangleright\, n$ então $h(m) > h(n)$ por (e) e, portanto, $T_{h(n)}$ não possui órbita de tamanho $m$ por (c).
\end{proof}
