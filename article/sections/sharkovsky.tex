\section{Teorema de Sharkovsky}

\begin{definition}
Se $I_0, I_1, \dots, I_{n-1}$ são intervalos fechados, $n > 1$,
\begin{enumerate}
\item dizemos que \textit{$I_0$ cobre $I_1$}, e denotamos por $I_0 \longrightarrow I_1$, se $f(I_0) \supset I_1$.

\item dizemos  que \textit{$I_0, I_1, \dots, I_{n-1}$ é um caminho entre $I_0$ e $I_{n-1}$}, e denotamos por $I_0 \longrightarrow I_1 \longrightarrow \cdots \longrightarrow I_{n-1}$, se $f(I_i) \supset I_{i+1}$, $i = 0, \dots, n-2$.

\item dizemos  que \textit{$I_0, I_1, \dots, I_{n-1}$ é um ciclo entre $I_0$ e $I_{n-1}$}, e denotamos por $I_0 \longrightarrow I_1 \longrightarrow \cdots \longrightarrow I_{n-1} \longrightarrow I_0$, se $f(I_i) \supset I_{i+1}$, $i = 0, \dots, n-2$, e $f(I_{n-1}) \supset I_0$.

\end{enumerate}
\end{definition}

\begin{lemma}
Se $I_0 \longrightarrow I_1$, então $f(I'_0) = I_1$ para algum intervalo fechado $I'_0 \subset I_1$.
\end{lemma}
\begin{proof}
Se $I_0 = [a, b]$ e $I_1 = [c, d]$, pelo Teorema do Valor Intermediário, existem $p, q \in I_0$ tais que $f(p) = c$ e $f(q) = d$. Suponha que $p \leq q$ e defina $I_0' = [a', b']$, onde
$$b' = \textrm{inf}\{x \in [p, q] : f(x) = d\} \textrm{ e } a' = \textrm{sup}\{x \in [p, b'] : f(x) = c\}$$

Sendo $f$ contínua temos que $f(a') = c$ e $f(b') = d$ e, desse modo, $f(I_0') \supset I_1$. Se $f(x) < c$ para algum $x \in I_0'$, existe $y \in [x, b']$ tal que $f(y) = c$, o que é um absurdo pois nesse caso $y > a'$. Absurdo análogo ocorre se $f(x) > d$ para algum $x \in I_0'$. Portanto, $f(I_0') = I_1$.
\end{proof}

\begin{lemma}
Se $J_0 \longrightarrow J_1 \longrightarrow \cdots \longrightarrow J_{n-1} \longrightarrow J_0$, então existe $p \in J_0$ tal que $f^k(p) \in J_k$, para todo $k = 1, \dots, n-1$, e $f^n(p) = p$.
\end{lemma}

\begin{proof}
De acordo com as hipóteses e com a Proposição anterior, temos as seguintes implicações:
\begin{align*}
J_0 \longrightarrow J_1 & \Rightarrow \textrm{ existe } J_0' \subset J_0 \textrm{ tal que } f(J_0') = J_1 \\
J_1 \longrightarrow J_2 & \Rightarrow \textrm{ existe } J_1' \subset J_0' \textrm{ tal que } f^2(J_1') = J_2 \\
\vdots \\
J_{n-2} \longrightarrow J_{n-1} & \Rightarrow \textrm{ existe } J_{n-2}' \subset J_{n-3}' \textrm{ tal que } f^{n-1}(J_{n-2}') = J_{n-1} \\
J_{n-1} \longrightarrow J_0 & \Rightarrow \textrm{ existe } J_{n-1}' \subset J_{n-2}' \textrm{ tal que } f^n(J_{n-1}') = J_0 
\end{align*}

Construímos então uma sequência de $n$ intervalos fechados $J_0 \supset J_0' \supset J_1' \supset \cdots \supset J_{n-1}'$ tal que $f^k(J_{k-1}') = J_k$, para todo $k = 1, \dots, n-1$, e $f^n(J_{n-1}') = J_0$. Como $J_0 \supset J_{n-1}'$, existe $p \in J_{n-1}'$ tal que $f^n(p) = p$. Em particular, $p \in J_0$ e $f^k(p) \in J_k$, para todo $k = 1, \dots, n-1$. 
\end{proof}

\begin{theorem}
Se $f$ admite ponto periódico de período principal $3$, então $f$ admite ponto periódico de período principal $n$, para todo $n \geq 1$.
\end{theorem}

\begin{proof}
Sejam $p$ um ponto periódico de período principal $3$ e $p_1 < p_2 < p_3$ os pontos da órbita de $p$ e suponha que $f(p_1) = p_2$ e $f(p_2) = p _3$. O outro caso possível, em que $f(p_1) = p_3$ e $f(p_3) = p_2$, é demonstrado de maneira análoga. Definindo $I_1 = [p_1, p_2]$ e $I_2 = [p_2, p_3]$, temos que $I_1 \longrightarrow I_2$, $I_2 \longrightarrow I_1$ e $I_2 \longrightarrow I_2$.
\begin{enumerate}[label=(\alph*)]
\item $n = 1$: Como $I_2 \longrightarrow I_2$, existe $p \in I_2$ tal que $f(p) = p$.
\item $n = 2$: Como $I_1 \longrightarrow I_2 \longrightarrow I_1$, existe $p \in I_1$ tal que $f(p) \in I_2$ e $f^2(p) = p$. Se $f(p) = p$, então $p \in I_1 \cap I_2 = \{p_2\}$, o que é um absurdo pois $p_2$ possui período principal 3. Desse modo, o período principal de $p$ é $2$.
\item $n > 3$: Se $I_2 \longrightarrow \cdots \longrightarrow I_2 \longrightarrow I_1 \longrightarrow I_2$ é um ciclo de tamanho $n$, existe $p \in I_2$ tal que $f^k(p) \in I_2$, para todo $k = 1, \dots, n-2$, $f^{n-1}(p) \in I_1$ e $f^n(p) = p$. Se $f^{n-1}(p) = p$, então $p \in I_1 \cap I_2 = \{p_2\}$, o que é um absurdo pois implica que $f(p) = p_3 \in I_1$. Se $f^k(p) = p$ para algum $k = 1, \dots, n-2$ implica que $f^k(p) \in I_2$, para todo $k \geq 1$. Em particular, $f^{n-1}(p) \in I_1 \cap I_2 = \{p_2\}$ e, portanto, $p = f^n(p) = p_3$, o que é um absurdo pois implica que $f(p) = p_1 \in I_2$. 
\end{enumerate}
Desse modo, o resultado está provado.
\end{proof}

\begin{definition}
O Ordenação de Sharkovsky é definida por
$$3 \triangleright 5 \triangleright 7 \triangleright \cdots
\triangleright 2 \cdot 3 \triangleright 2 \cdot 5 \triangleright 2 \cdot 7 \triangleright \cdots
\triangleright 2^2 \cdot 3 \triangleright 2^2 \cdot 5 \triangleright 2^2 \cdot 7 \triangleright \cdots
\triangleright 2^n \cdot 3 \triangleright 2^n \cdot 5 \triangleright 2^n \cdot 7 \triangleright \cdots
\triangleright 2^2 \triangleright 2 \triangleright 1$$
ou seja, é formada inicialmente pelos ímpares maiores que 1 em ordem crescente; depois pelos ímpares maiores que 1, multiplicados por 2, em ordem crescente; depois pelos ímpares maiores que 1, multiplicados por $2^2$, em ordem crescente; e assim sucessivamente. Por fim, a ordem é formada por todas as potências de 2 em ordem decrescente.
\end{definition}

\begin{theorem}[Sharkovsky]
Se $f$ admite ponto de período principal $n$, então $f$ admite ponto de período principal $m$, para todo $m \triangleleft n$.
\end{theorem}

\begin{theorem}
Para todo $n \geq 1$ existe uma função $f$ que admite ponto periódico de período principal $n$ e que não admite ponto de período principal $m$ se $m \triangleright n$.
\end{theorem}

\begin{proof}
Seja $T: [0,1] \to [0,1]$ a função dada por $T(x) = 1 - |2x - 1|$ e considere a família de funções $T_h(x) = \textrm{min}\{h, T(x)\}$ definidas em $[0,1]$, com o parâmetro $h$ variando em $[0,1]$. Observe que $T_1 = T$, pois $T(x) \leq 1$ para todo $x \in [0,1]$. Além disso, observando o gráfico de $T_1$ concluímos que a função possui $2^k$ pontos periódicos de período $k$ e assim podemos definir, para cada $k \geq 1$, $$h(k) = \textrm{min} \{ \textrm{max} \{ \mathcal{O} : \mathcal{O} \textrm{ é uma órbita de tamanho } n \textrm{ de } T_1\} \}$$

A ideia principal da prova consiste no fato de que $h(k)$ desempenha os papéis de parâmetro, máximo e ponto de uma órbita de $T_{h(k)}$. As seguintes afirmações tornarão preciso esse fato.

\begin{enumerate}[label = (\alph*)]
\item Se $\mathcal{O} \subset [0, h)$ é uma órbita de $T_h$, então $\mathcal{O}$ é uma órbita de $T_1$.

Se $p \in \mathcal{O}$ então $T_h(p) \in [0, h)$. Desse modo, $T_h(p) = \textrm{min}\{h, T(p)\} = T(p) = T_1(p)$, ou seja, $T_h$ e $T_1$ coincidem em $\mathcal{O}$ e, portanto, $\mathcal{O}$ é uma órbita de $T_1$.

\item Se $\mathcal{O} \subset [0, h]$ é uma órbita de $T_1$, então $\mathcal{O}$ é uma órbita de $T_h$.

Se $p \in \mathcal{O}$ então $T_1(p) \in [0, h]$. Desse modo, $T_h(p) = \textrm{min}\{h, T(p)\} = \textrm{min}\{h, T_1(p)\} = T_1(p)$, ou seja, $T_h$ e $T_1$ coincidem em $\mathcal{O}$ e, portanto, $\mathcal{O}$ é uma órbita de $T_h$.

\item $T_{h(k)}$ possui uma órbita $\mathcal{O} \in [0, h(k))$ de tamanho $l$ se e somente se $h(k) > h(l)$.

Se $T_{h(k)}$ possui uma órbita $\mathcal{O} \in [0, h(k))$ de tamanho $l$, então $\mathcal{O}$ é uma órbita de $T_1$ por (a) e, pela definição de $h(l)$, concluímos que $h(l) < h(k)$.

Por outro lado, se $h(l) < h(k)$, então $T_1$ possui uma órbita $\mathcal{O} \subset [0, h(l)] \subset [0, h(k)]$ de tamanho $l$ e, desse modo, $\mathcal{O}$ é uma órbita de $T_{h(k)}$ por (b).

\item A órbita de $T_1$ que contém $h(k)$ é uma órbita de tamanho $k$ de $T_{h(k)}$. Além disso, todas as outras órbitas de $T_{h(k)}$ estão em $[0, h(k))$. 

Pela definição de $h(k)$, $T_1$ possui uma órbita $\mathcal{O} \subset [0, h(k)]$ de tamanho $k$ e, portanto, $\mathcal{O}$ é uma órbita de $T_{h(k)}$ por (b).

Para demonstrar a segunda parte, basta observar que $h(k)$ é o valor máximo de $T_{h(k)}$ e, desse modo, toda órbita de $T_{h(k)}$ está contida em $[0, h(k)]$. Em particular, se a órbita não contém $h(k)$, então ela está contida em $[0, h(k))$.

\item $k \triangleright l$ se o somente se $h(k) > h(l)$.

Suponha que $k \triangleright l$. Por (d), $T_{h(k)}$ possui uma órbita de tamanho $k$. De acordo com o Teorema de Sharkovsky e com (d), $T_{h(k)}$ admite uma órbita de tamanho $l$ contida em $[0, h(k))$. Desse modo,  $h(k) > h(l)$ por (c).

Por outro lado, suponha que $h(k) > h(l)$. Caso $l \triangleright k$, a demonstração no parágrafo anterior implicaria que $h(k) < h(l)$, contrariando a hipótese. Desse modo, $k \triangleright l$.
\end{enumerate}

Assim, para cada $n \geq 1$, $T_{h(n)}$ possui órbita de tamanho $n$. Além disso, se $m \triangleright n$ então $h(m) > h(n)$ por (e) e, portanto, $T_{h(n)}$ não possui órbita de tamanho $m$ por (c).
\end{proof}