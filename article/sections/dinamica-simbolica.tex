\subsection{Dinâmica Simbólica}

Dado $N \geq 2$, seja $\Sigma_N$ o conjunto das sequências de números naturais limitados entre $1$ e $N$, isto é, $\Sigma_N = \lbrace (x_0 \, x_1 \, x_2 \, \dots) : 1 \leq x_n \leq N  \text{ para todo } n \geq 0 \rbrace$.
Seja também $d_N : \Sigma_N \times \Sigma_N \to \RR$ a função dada por
$$d_N(x, y) = \sum_{k=0}^\infty \frac{|x_k - y_k|}{N^k},$$
onde $x = (x_0 \, x_1 \, x_2 \, \dots)$ e $y = (y_0 \, y_1 \, y_2 \, \dots)$.
Observe que $(\Sigma_N, d_N)$ é um espaço métrico.
Por fim, seja $\sigma: \Sigma_N \to \Sigma_N$ a função é dada por $\sigma(x_0 \, x_1 \, x_2 \, \dots) = (x_1 \, x_2 \, x_3 \, \dots)$.

\begin{proposition}
Sejam $x = (x_0 \, x_1 \, x_2 \, \dots)$ e $y = (y_0 \, y_1 \, y_2 \, \dots)$ elementos de $\Sigma_N$.
\begin{enumerate}
\item Se $x_k = y_k$ para todo $0 \leq k \leq n$, então $d_N(x, y) \leq \frac{1}{N^n}$.
\item Se $d_N(x, y) < \frac{1}{N^n}$, então $x_k = y_k$ para todo $0 \leq k \leq n$.
\end{enumerate}
\end{proposition}

\begin{proposition}
$\sigma$ é contínua.
\end{proposition}

\begin{proof}
Sejam $x = (x_0 \, x_1 \, x_2 \, \dots) \in \Sigma_N$, $\varepsilon > 0$ e $n \geq 1$ tal que $\frac{1}{2^n} < \varepsilon$.
Se $d(x, y) < \frac{1}{2^{n+1}}$, onde $y = (y_0 \, y_1 \, y_2 \, \dots) \in \Sigma_N$, então $x_k = y_k$ para todo $0 \leq k \leq n+1$.
Como $\sigma(x) = (x_1 \, x_2 \, x_3  \, \dots)$ e $\sigma(y) = (y_1 \, y_2 \, y_3 \, \dots)$, temos que as primeiras $n+1$ entradas de $\sigma(s)$ e $\sigma(t)$ são iguais. Desse modo,  $d(\sigma(s), \sigma(t)) \leq \frac{1}{2^n} < \varepsilon$ e, portanto, $\sigma$ é contínua.
\end{proof}

Para a demonstração do próximo resultado, vamos considerar $N = 2$.
Se $\Lambda_1 = [0, x_1] \cup [x_2, 1]$, sejam $J_1 = [0, x_1]$ e $J_2 = [x_2, 1]$.
Como $\Lambda \subset J_1 \cup J_2$, podemos definir a função $S: \Lambda \to \Sigma_2$ dada por $S(x) = (x_0 \, x_1 \, x_2 \, \dots)$, onde $x_k = 1$ se $h^k(x) \in J_1$ e $x_k = 2$ se $h^k(x) \in J_2$ para todo $k \geq 0$.

\begin{proposition}
Se $\mu > 2 + \sqrt{5}$, então $h|_\Lambda$ e $\sigma$ são topologicamente conjugadas por $S$.
\end{proposition}

\begin{proof}
\begin{enumerate}[label=\alph*)]\item[]
\item $S$ é injetora.

Sejam $x, y \in \Lambda$, $x < y$. Se $S(x) = S(y)$, então $h^k(x)$ e $h^k(y)$ está no mesmo lado em relação ao ponto crítico para todo $k \geq 0$ e, portanto, $h$ é monótona em cada intervalo $I_k$ cujos extremos são $h^k(x)$ e $h^k(y)$. Desse modo, se $z \in [x, y]$, então $h^k(z) \in I_k \subset J_1 \cup J_2$ para todo $k \geq 0$ e, portanto, $z \in \Lambda$, o que é um absurdo pois $\Lambda$ não contém intervalos.

\item $S$ é sobrejetora.

Seja $(x_0 \, x_1 \, x_2 \, \dots) \in \Sigma_2$. Inicialmente, para cada $n \geq 0$, considere
$$J_{x_0 \, \dots \, x_n} = \lbrace x \in [0,1] : x \in J_{x_0}, \dots, h^n(x) \in J_{x_n} \rbrace.$$

Observe $J_{x_0 \, \dots \, x_n} = J_{x_0} \cap h^{-1}(J_{x_1 \, \dots \, x_n})$ e, portanto, é possível concluir por indução que $J_{x_0 \, \dots \, x_n}$ é um intervalo fechado não vazio. Além disso, $J_{x_0 \, \dots \, x_n} = J_{x_0 \, \dots \, x_{n-1}} \cap h^{-n}(J_{x_n}) \subset J_{x_0 \, \dots \, x_{n-1}}$.

Desse modo, $(J_{x_0 \, \dots \, x_n})_{n=0}^\infty$ é uma sequência de intervalos encaixantes fechados e não vazios e, portanto, existe $x \in \cap_{n=0}^\infty J_{x_0 \, \dots \, x_n}$. Como $h^k(x) \in J_{x_k}$ para todo $k \geq 0$, concluímos que $S(x) = (x_0 \, x_1 \, x_2 \, \dots)$. Observe que $x$ é único, pois $S$ é injetora.

\item $S$ é contínua.

Sejam $x \in \Lambda$, $\varepsilon > 0$ e $k \geq 0$ tal que $\frac{1}{N^k} < \varepsilon$. Se $S(x) = (x_0 \, x_1 \, x_2 \, \dots)$, então $x \in J_{x_0 \, \dots \, x_k}$. Sendo $J_{x_0 \, \dots \, x_k}$ um intervalo fechado, existe $\delta > 0$ tal que se $y \in \Lambda$ e $|x-y| < \delta$, então $y \in J_{x_0 \, \dots \, x_k}$. Desse modo, $S(x)$ e $S(y)$ são iguais nas primeiras $k+1$ entradas e, portanto, $d_N(S(x), S(y)) \leq \frac{1}{N^k} < \varepsilon$.

\item $S^{-1}$ é contínua.

\item $S \circ h|_\Lambda = \sigma \circ S$.

Se $x \in \Lambda$ e $S(x) = (x_0 \, x_1 \, x_2 \, \dots)$, então $\cap_{n=0}^\infty J_{x_0 \, \dots \, x_n} = \lbrace x \rbrace$. Desse modo, é imediato que
$$ S \circ h|_{\Lambda}(x) = S(h(\cap_{n=0}^\infty J_{x_0 \, \dots \, x_n})) = S(\cap_{n=1}^\infty J_{x_1 \, \dots \, x_n}) = (x_1 \, x_2 \, x_3 \, \dots)  = \sigma \circ S(x).$$
\end{enumerate}
\end{proof}
