\section{Teorema de Singer}

Ao longo dessa seção, consideraremos $f: \RR \to \RR$ uma função de classe $\class^3$. Iniciamos com a seguinte definição: 

\begin{definition}[Derivada de Schwarz]
A derivada de Schwarz de $f$ é a função $\schwarz f: \RR \backslash C_f \to \RR$ dada por
$$\schwarz f(x) = \frac{D^3 f(x)}{D f(x)} - \frac{3}{2} \left( \frac{D^2 f(x)}{D f(x)} \right)^2,$$
onde $C_f = \lbrace x \in \RR: Df(x) = 0 \rbrace$.
\end{definition}

Para os nossos objetivos, estudaremos funções que possuem a derivada de Schwarz negativa. Por exemplo, $Sh(x) = \frac{-6}{(1 - 2x)^2} < 0$ para todo $x \neq \frac{1}{2}$. A característica fundamental dessa propriedade é ser preservada em composição de funções.

\begin{proposition}
Se $\schwarz f < 0$ e $\schwarz g < 0$, então $\schwarz (f \circ g) < 0$.
\end{proposition}

\begin{proof}
Basta observar que $\schwarz (f \circ g)(x) = \schwarz f(g(x))(D g(x))^2 + \schwarz g(x)$.
\end{proof}

\begin{corollary}
Se $\schwarz f < 0$, então $\schwarz f^n < 0$ para todo $n \geq 1$.
\end{corollary}

Vamos mostra que se uma função possui derivada de Schwarz negativa e um número finito de pontos críticos, então existe um limite para a quantidade de órbitas periódicas atratoras. Para isso, provaremos uma série de lemas.

\begin{lemma}\label{lem 10-1}
Se $\schwarz f < 0$ e $x_0$ é um ponto de mínimo local de $D f$, então $D f(x_0) \leq 0$.
\end{lemma}

\begin{proof}
Se $D f(x_0) \neq 0$, então
$$\schwarz f(x_0) = \frac{D^3 f(x_0)}{D f(x_0)} - \frac{3}{2} \left( \frac{D^2 f(x_0)}{D f(x_0)} \right)^2 < 0.$$
Sendo $x_0$ ponto de mínimo local de $D f$, temos que $D^2 f(x_0) = 0$ e $D^3 f(x_0) \geq 0$ e, portanto, $D f(x_0) < 0$. 
\end{proof}

\begin{lemma}\label{lem 10-2}
Se $\schwarz f < 0$ e $a<b<c$ são pontos fixos de $f$ com $D f(b) \leq 1$, então $f$ possui ponto crítico em $(a, c)$.
\end{lemma}

\begin{proof}
Pelo TVM, existem $r \in (a,b)$ e $s \in (b,c)$  tais que $D f(r) = D f(s) = 1$.
Sendo $D f$ contínua, $D f$ restrita ao intervalo $[r,s]$ possui mínimo global.
Como $b \in (r,s)$ e $D f(b) \leq 1$, temos que $D f$ possui mínimo local em $(r,s)$.
Utilizando Lema anterior e o TVI, a demonstração está concluída.
\end{proof}

\begin{lemma}\label{lem 10-3}
Se $\schwarz f < 0$ e $a<b<c<d$ são pontos fixos de $f$, então $f$ possui ponto crítico em $(a,d)$.
\end{lemma}

\begin{proof}
Se $D f(b) \leq 1$ ou $D f(c) \leq 1$, o resultado é verdadeiro pelo Lema anterior.
Se $D f(b) > 1$ e $D f(c) > 1$, existem $r, t \in (b,c)$ tais que $r<t$, $f(r) > r$ e $f(t) < t$.
Pelo TVM, existe $s \in (r,t)$ tal que $D f(s) < 1$.
Portanto, $D f$ possui mínimo local em $(b,c)$.
Utilizando Lema \ref{lem 10-1} e o TVI, a demonstração está concluída.
\end{proof}

\begin{lemma}\label{lem 10-4}
Se $f$ possui finitos pontos críticos, então $f^n$ possui finitos pontos críticos para todo $n \geq 1$.
\end{lemma}
\begin{proof}
Pelo TVM, se $c \in \RR$, então $f$ possui ponto crítico entre dois elementos de $f^{-1}(c)$ e, portanto, $f^{-1}(c)$ é finito.
De modo mais geral, é possível provar por indução que $f^{-n}(c)$ é finito para todo $n \geq 1$.

Se $n \geq 1$, então $D f^n(x) = \prod_{k=0}^{n-1} D f(f^k(x)) = 0$ se, e somente se, $f^k(x)$ é ponto crítico de $f$ para algum $1 \leq k < n$.
Portanto, o conjunto de pontos críticos de $f^n$ é finito.
\end{proof}

\begin{lemma}
Se $\schwarz f < 0$ e $f$ possui finitos pontos críticos, então $f^n$ possui finitos pontos fixos para todo $n \geq 1$.
\end{lemma}

\begin{proof}
Pelo Lema \ref{lem 10-3}, se $f^n$ possui infinitos pontos fixos para algum $n \geq 1$, então $f^n$ possui infinitos pontos críticos, o que é um absurdo pelo Lema \ref{lem 10-4}.
\end{proof}

Com isso, temos as ferramentas necessárias para demonstrar o Teorema de Singer.

\begin{theorem}[Singer]
Se $\schwarz f < 0$ e $f$ possui $n$ pontos críticos, então $f$ possui no máximo $n+2$ órbitas periódicas não repulsoras.
\end{theorem}

\begin{proof}
Seja $p$ um ponto periódico não repulsor de $f$ de período $m$.
Se $g = f^m$, então $g(p) = p$ e $|D g(p)| \leq 1$.
Seja $K$ a componente conexa de $\basin(p) = \lbrace x : \lim_{k \to \infty} g^k(x) = p \rbrace$ que contém $p$.
Inicialmente, suponha que $K$ é limitado.

Se $|D g(p)| < 1$, então é possível mostrar que $K$ é aberto, $g(K) \subset K$ e $g$ preserva os pontos extremos de $K$.

Escrevendo $K = (a, b)$, se $g(a) = a$ e $g(b) = b$, então $g$ possui ponto crítico em $K$ pelo Lema \ref{lem 10-2}; se $g(a) = b$ e $g(b) = a$, então $g^2$ possui ponto crítico em $K$ pelo Lema \ref{lem 10-2}; se $g(a) = g(b)$, então $g$ possui ponto crítico em $K$ pelo TVM.

Se $|D g(p)| = 1$, então os pontos fixos de $g$ são isolados pelo Lema anterior e, portanto, existe uma vizinhança de $p$ que não contém outros pontos fixos de $g$.

Suponha que $D g(p) = 1$.
Se $D g(p) = -1$, a demonstração é análoga considerando $g^2$.
Se $p$ possui o comportamento de um ponto repulsor, então, para $x$ numa vizinhança de $p$, $g(x) > x$ quando $x > p$ e $g(x) < x$ quando $x < p$.
Desse modo, $1$ é um mínimo local de $D g$, o que é um absurdo pelo Lema \ref{lem 10-1} e, portanto, $p$ é atrator em pelo menos um dos lados.
Desse modo, $K$ é um intervalo, $g(K) \subset K$ e $g$ preserva os pontos extremos de $K$.
Assim, é possível concluir de maneira análoga que $g$ possui ponto crítico em $K$.

Assim, cada intervalo $K$ limitado está associado à algum ponto crítico de $f$ e, portanto, existem no máximo $n$ desses intervalos.
Não é possível obter a mesma conclusão se $K$ não é limitado, mas observando que existem no máximo dois intervalos desse tipo, a demonstração está concluída.
\end{proof}

\begin{corollary}
Se $\mu > 1$, então $h$ possui no máximo $1$ órbita periódica não repulsora.
\end{corollary}
