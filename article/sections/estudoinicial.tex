\section{Função Logística}

\begin{proposition}
Se $\mu > 1$, então
\begin{enumerate}
\item $F(1) = F(0) = 0$.
\item $F \left( \frac{1}{\mu} \right) = F(p_\mu) = p_\mu$, onde $p_\mu = \frac{\mu-1}{\mu}$.
\item $0 < p_\mu < 1$.
\end{enumerate}
\end{proposition}

\begin{proof}
Aplicação direta das definições.
\end{proof}

\begin{proposition} \label{prop 2-1}
Se $\mu > 1$, então $(-\infty, 0) \cup (1, \infty) \subset \bb(\infty)$.
\end{proposition}

\begin{proof}
Se $x < 0$, a sequência  $(x, F(x), F^2(x), \dots)$ é estritamente decrescente pois  $F(x) < x$. Se $(F^n(x))_n \to x_0$ quando $n \to \infty$, a continuidade de $F$ implica que $(F^{n+1}(x))_n \to F (x_0) < x_0$. Absurdo. Portanto, $(F^n(x))_n \to -\infty$ quando $n \to \infty$. Como $F(x) < 0$ para todo $x > 1$, concluímos que $(-\infty, 0) \cup (1, \infty) \subset W^s(\infty)$.
\end{proof}

\begin{proposition}
Se $1 < \mu < 3$, então
\begin{enumerate}
\item $0$ é um ponto repulsor e $p_\mu$ é um ponto atrator.
\item $(0, 1) \subset \bb(p_\mu)$.
\end{enumerate}
\end{proposition}

\begin{proof}
A primeira parte é verdadeira pois $|F'(0)| = \mu > 1$ e $|F'(p_\mu)| = |2 - \mu| < 1$, quando $1 < \mu < 3$.
\end{proof}

Desse modo, conhecemos completamente a dinâmica de $F$ quando $1 < \mu < 3$:
$$\bb(0) = \{0, 1\}\textrm{, } \bb(p_\mu) = (0, 1)\textrm{ e }\bb(\infty) = (-\infty, 0) \cup (1, \infty).$$