\subsection{Caos}

A família quadrática exibe um fenômeno quando $\mu$ é suficientemente grande: o comportamento caótico de suas órbitas.
Para esse estudo, iniciamos com a seguinte proposição:

Um conceito importante que vamos definir é o de transitividade topológica. Intuitivamente, em um sistema topologicamente transitivo, é possível ir uma vizinhança para qualquer outra através de iterações. A definição a seguir torna isso preciso e em seguida mostramos que $h$ possui essa propriedade para $\mu$ suficientemente grande.

\begin{definition}
Seja $f: X \to X$ uma função. Dizemos que $f$ é topologicamente transitiva se dados $x, y \in X$ e $\varepsilon > 0$,  existem $z \in X$ e $k \geq 1$ tais que $|x - z| < \varepsilon$ e $|y - f^k(z)| < \varepsilon$.
\end{definition} 

\begin{proposition}
Se $\mu > 2 + \sqrt{5}$, então $h|_\Lambda$ é topologicamente transitiva.
\end{proposition}

\begin{proof}
Sejam $x, y \in \Lambda$, $\varepsilon > 0$ e $k \geq 1$ tal que $\frac{1}{\lambda^k} < \varepsilon$.
Temos que $h^k: [a, b] \to [0, 1]$ é bijetora, onde $x \in [a, b]$ e $[a, b]$ é um dos intervalos que formam $\Lambda_k$. Pelo TVI, existe $z \in [a, b]$ tal que $h^k(z) = y$. Observando que $z \in \Lambda$, concluímos que $h|_\Lambda$ é topologicamente transitiva.
\end{proof}

Outro conceito importante que vamos definir é o de dependência sensível das condições iniciais. Intuitivamente, em um sistema que possui dependência sensível das condições iniciais, existem pontos arbitrariamente próximos à $x$ que eventualmente se distanciam de $x$ em pelo menos $\delta$ através de iterações de $f$. Novamente, a definição a seguir torna isso preciso e em seguida mostramos que $h$ possui essa propriedade para $\mu$ suficientemente grande.

\begin{definition}
Seja $f: X \to X$ uma função. Dizemos que $f$ depende sensivelmente das condições iniciais se existe $\delta > 0$ com a seguinte propriedade: dados $x \in X$ e $\varepsilon > 0$, existem $y \in X$ e $k \geq 1$ tais que $|x - y| < \varepsilon$ e $|f^k(x) - f^k(y)| > \delta$.
\end{definition}

\begin{proposition}
Se $\mu > 2 + \sqrt{5}$, então $h|_\Lambda$ depende sensivelmente das condições iniciais.
\end{proposition}

\begin{proof}
Sejam $x \in \Lambda$, $\varepsilon > 0$ e $k \geq 1$ tal que $\frac{1}{\lambda^k} < \varepsilon$.
Temos que $h^k: [a, b] \to [0, 1]$ é bijetora, onde $x \in [a, b]$ e $[a, b]$ é um dos intervalos que formam $\Lambda_k$. Suponha que $h^k(a) = 0$ e $h^k(b) = 1$; se $h^k(a) = 1$ e $h^k(b) = 0$, a demonstração é análoga.

Como $h(\frac{1}{2}) > 1$ e $x \in \Lambda$, temos que $h^{k}(x) \in [0, \frac{1}{2}) \cup (\frac{1}{2}, 1]$. Se $h^{k}(x) \in [0, \frac{1}{2})$, então $|h^k(x) - h^k(b)| = |h^k(x) - 1| > \frac{1}{2}$ e se $h^{k}(x) \in (\frac{1}{2}, 1]$, então $|h^k(x) - h^k(a)| = |h^k(x)| > \frac{1}{2}$. Observando que $|x - a| < \varepsilon$ e  $|x - b| < \varepsilon$, concluímos que $h|_\Lambda$ depende sensivelmente das condições iniciais.
\end{proof}

Segundo a definição de Devaney, um sistema dinâmico é caótico se, além de possuir as duas propriedades definidas anteriormente, possui o conjunto de pontos periódicos denso. 

\begin{definition}
Seja $f: X \to X$ uma função. Dizemos que $f$ é caótica se as seguintes condições são válidas:
\begin{enumerate}[label=\roman*.]
\item $\per(f)$ é denso em $X$.
\item $f$ é topologicamente transitiva.
\item $f$ depende sensivelmente das condições iniciais.
\end{enumerate}
\end{definition}

Desse modo, basta mostrar que $\per(h|_\Lambda)$ é denso em $\Lambda$ e concluímos imediatamente que $h|_\Lambda$ é caótica quando $\mu > 2 + \sqrt{5}$.

\begin{proposition}
Se $\mu > 2 + \sqrt{5}$, então $\per(h|_\Lambda)$ é denso em $\Lambda$.
\end{proposition}

\begin{proof}
Sejam $x \in \Lambda$, $\varepsilon > 0$ e $k \geq 1$ tal que $\frac{1}{\lambda^k} < \varepsilon$.
Temos que $h^k: [a, b] \to [0, 1]$ é bijetora, onde $x \in [a, b]$ e $[a, b]$ é um dos intervalos que formam $\Lambda_k$.
Como $h^k([a, b]) \supset [a, b]$, existe $y \in [a, b]$ tal que $h^k(y) = y$. Observando que $y \in \Lambda$ e $|x - y| < \varepsilon$, concluímos que $\per(h|_\Lambda)$ é denso em $\Lambda$.
\end{proof}

\begin{theorem}
Se $\mu > 2 + \sqrt{5}$, então $h|_\Lambda$ é caótica.
\end{theorem}

Por fim, apresentamos um teorema que nos permite, sob algumas hipóteses, concluir um sistema dinâmico é caótico se ele é topologicamente transitivo e seu conjunto de pontos periódicos é denso. A demonstração desse teorema pode ser encontrada em [2].

\begin{theorem}
Seja $f: X \to X$ é uma função contínua, onde $X$ é um conjunto infinito. Se $\per(f)$ é denso em $X$ e $f$ é topologicamente transitiva, então $f$ é caótica.
\end{theorem}
