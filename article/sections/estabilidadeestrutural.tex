\section{Estabilidade Estrutural}

\begin{definition}
Sejam $f, g: D \to \RR$ funções de classe $\CC^k$. A $\CC^k$-distância entre $f$ e $g$ é definida por
$$d_k(f, g) = \sup_{x \in D} \left \{ |f(x) - g(x)|, |f'(x) - g'(x)|, \dots, |f^{(k)}(x) - g^{(k)}(x)| \right \}$$
\end{definition}

\begin{definition}
Seja $f: D \to \RR$ uma função de classe $\CC^k$. Dizemos que $f$ é $\CC^k$-estável se existe $\varepsilon > 0$ tal que se $g: D \to \RR$ é de classe $\CC^k$ e $d_k(f, g) < \varepsilon$, então $f$ e $g$ são topologicamente conjugadas.
\end{definition}

\begin{example}
Seja $L: \RR \to \RR$ a função definida por $L(x) = \frac{x}{2}$. Se $g: \RR \to \RR$ é uma função de classe $\CC^1$ com $d_1(L, g) < \frac{1}{2}$, vamos mostrar que $L$ e $g$ são topologicamente conjugadas.

Inicialmente, $g$ possui pelo menos 1 ponto fixo. Como $\left| \frac{x}{2} - g(x) \right| < \frac{1}{2}$ para todo $x \in \RR$, temos que $-\frac{1}{2} < \frac{x}{2} - g(x) < \frac{1}{2}$ e, portanto, $-\frac{1}{2} - \frac{x}{2} < g(x) - x < \frac{1}{2} - \frac{x}{2}$. Definindo $h(x) = g(x) - x$, temos que $0 < h(-1) < 1$ e $-1 < h(1) < 0$. Pelo Teorema do Valor Intermediário, existe $x_0 \in (-1, 1)$ tal que $h(x_0) = 0$ e, portanto, $g(x_0) = x_0$.

Além disso, $g$ possui no máximo 1 ponto fixo. Como $\left| \frac{1}{2} - g'(x) \right| < \frac{1}{2}$ para todo $x \in \RR$, temos que $0 < g'(x) < 1$. De acordo com o Teorema do Valor Médio, se $g$ possui $2$ pontos fixos, então existe $x_0$ tal que $g'(x_0) = 1$, o que é um absurdo.

Seja $J = [-10, -5) \cup (5, 10]$. Observe que se $x \in \RR - \{0\}$, então existe um único $n_x \in \ZZ$ tal que $L^{n_x}(x) \in J$. Analogamente, se $x \in \RR$ e $x$ não é ponto fixo de $g$, então existe um único $n_x$ tal que $g^{n_x}(x) \in [-10, g(-10)) \cup (g(10), 10]$.

Seja $h$ uma função tal que $h|_{[-10, -5]}$ é um homeomorfismo crescente entre $[-10, -5]$ e $[-10, g(-10)]$ e $h|_{[5, 10]}$ é um homeomorfismo crescente entre $[5, 10]$ e $[g(10), 10]$.

Seja $x \in \RR - \{0\}$. Como $L^{n_x}(x) \in J$, temos que $h \circ L^{n_x}(x)$ está bem definido. Sendo $g$ um homeomorfismo, $g^{-n_x} \circ h \circ L^{n_x}(x)$ também está bem definido. Defina $h(x) = g^{-n_x} \circ h \circ L^{n_x}(x)$ para todo $x \in \RR - \{0\}$. Observe que se $x \in J$, então $n_x = 0$ e, portanto, está bem definida em $J$. Por fim, defina $h(0)$ como sendo o ponto fixo de $g$. Resta mostrar que $h \circ L(x) = g \circ h(x)$ para todo $x \in \RR$.

Se $x \neq 0$, então $h(x) = g^{-n_x} \circ h \circ L^{n_x}(x)$. Se $y = L(x)$, então $y \neq 0$ e $L^{n_x - 1}(y) = L^{n_x - 1}(L(x)) = L^{n_x}(x) \in J$, ou seja, $n_y = n_x - 1$. Desse modo,
$$h \circ L(x) = h(y) = g^{-n_y} \circ h \circ L^{n_y}(y) = g \circ g^{-n_x} \circ h \circ L^{n_x}(x) = g \circ h(x)$$
e $g(h(0)) = h(0) = h(L(0))$.

Assim, $h \circ L = g \circ h$. Além disso, $h$ é um homeomorfismo pois é composição de homeomorfismos. Desse modo, $L$ e $g$ são topologicamente conjugadas e, portanto, $L$ é $\CC^1$-estável.
\end{example}

Finalmente, vamos estudar a estabilidade estrutural da função quadrática $F_\mu(x) = \mu x(1-x)$ para $\mu > 2 + \sqrt{5}$.

Relembrando, $0$ e $p_\mu = \frac{\mu - 1}{\mu}$ são os únicos pontos fixos de $F_\mu$. Além disso, $F_\mu$ possui um único ponto crítico em $\frac{1}{2}$, é estritamente crescente em $\left(-\infty, \frac{1}{2}\right)$ e é estritamente decrescente em $\left(\frac{1}{2}, \infty \right)$. Sendo $F_\mu \left(\frac{1}{2} \right) > 1$, temos que $F_\mu^{-1}(1)$ possui dois elementos. Denotando tais elementos por $y_0$ e $y_1$, com $y_0 < y_1$, temos que $|F_\mu'(x)| > 1$ para todo $x \in [0, y_0] \cup [y_1, 1]$.

Além disso, $\lim_{n \to \infty} F_\mu^n(x) = - \infty$ para todo $x \notin [0, y_0] \cup [y_1, 1]$ e, desse modo, estudamos a dinâmica de $F_\mu$ restrita ao conjunto $\Lambda = \{ x \in [0,1] : F_\mu^n(x) \in  [0, 1] \textrm{ para todo } n \geq 1\}$. Por fim, mostramos que $F_\mu|_\Lambda$ é topologicamente conjugada com a função $\sigma$ em $\Sigma_2$.

\begin{theorem}
Se $\mu > 2 + \sqrt{5}$, então $F_\mu$ é $\CC^2$-estável.
\end{theorem}

\begin{proof}
Vamos mostrar que existe $\varepsilon > 0$ tal que se $g$ é de classe $\CC^2$ e $d_2(F_\mu, g) < \varepsilon$, então $F_\mu$ e $g$ são topologicamente conjugadas.

Seja $\varepsilon_1 > 0$ tal que $d_2(F_\mu, g) < \varepsilon_1$ implica que $g'' < 0$ e, portanto, que a concavidade de $g$ é para baixo. Existe $\varepsilon_1$ com essa propriedade pois $F''_\mu = -2\mu$. 

Seja $0 < \varepsilon_2 < \varepsilon_1$ tal que $d_2(F_\mu, g) < \varepsilon_2$ implica que $g$ possui dois pontos fixos $\alpha < \beta$ com $g'(\alpha) > 1$ e $g'(\beta) < -1$. Existe $\varepsilon_2$ com essa propriedade pois $F_\mu$ possui os pontos fixos $0$ e $p_\mu$ com $F'_\mu(0) > 1$ e $F'_\mu(p_\mu) < -1$.

Pelo Teorema do Valor Médio, temos que $g$ possui um ponto crítico $c \in (\alpha, \beta)$. Sendo $g'' < 0$, o ponto crítico de $g$ é único. Além disso, $g$ é estritamente crescente em $(-\infty, c)$ e estritamente decrescente em $(c, \infty)$. Desse modo, existe $\alpha' \in (c, \infty)$ tal que $g(\alpha') = \alpha$.

Por fim, seja $0 < \varepsilon < \varepsilon_2$ tal que $d_2(F_\mu, g) < \varepsilon$ implica que $g^{-1}(\alpha')$ possui os elementos $x_0$ e $x_1$, com $x_0 < x_1$, e que $|g'(x)| > 1$ para todo $x \in [\alpha, x_0] \cup [x_1, \alpha']$.

Desse modo, se $d_2(F_\mu, g) < \varepsilon$, então os gráficos de $g$ e $F_\mu$ possuem as mesmas propriedades. Em particular, $\lim_{n \to \infty} g(x) = - \infty$ para todo $x \notin [\alpha, x_0] \cup [x_1, \alpha']$. De modo análogo ao feito para $F_\mu$ restrita ao conjunto $\Lambda$, é possível mostrar que $g$ restrita ao conjunto $\Lambda_g = \{ x \in [\alpha, \alpha'] : g^n(x) \in [\alpha, \alpha'] \textrm{ para todo } n \geq 1\}$ é topologicamente conjugada com a função $\sigma$ de $\Sigma_2$. Portanto, por transitividade, $F_\mu$ e $g$ são topologicamente conjugadas.
\end{proof}

\begin{theorem}[Hartman]
Seja $p$ um ponto fixo hiperbólico de $f$ e suponha que $f'(p) = \lambda \neq 0$. Então existem vizinhanças $U$ de $p$ e $V$ de $0$ e um homeomorfismo $h:U \to V$ que conjuga as funções $f|_U$ e $L(x) = \lambda x$, $x \in V$. 
\end{theorem}













