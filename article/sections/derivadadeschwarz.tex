\section{Derivada de Schwarz}

\begin{definition}
Seja $f: \RR \to \RR$ uma função de classe $\CC^3$.
A derivada de Schwarz de $f$ é função $\schwarz f$ dada por
$$(\schwarz f)(x) = \frac{\partial^3 f(x)}{\partial f(x)} - \frac{3}{2} \left( \frac{\partial^2 f(x)}{\partial f(x)} \right)^2$$
para todo $x$ tal que $\partial f(x) \neq 0$.
\end{definition}

\begin{proposition}
Se $\schwarz f < 0$, então $\schwarz f^n < 0$ para todo $n \geq 1$.
\end{proposition}

\begin{proof}
Pela Regra da Cadeia, podemos concluir que
$$(\schwarz f^2)(x) = (\schwarz f)(f(x)) [\partial f(x)]^2 + (\schwarz f)(x) < 0$$
para todo $x$ tal que $\partial f^2(x) \neq 0$.
Desse modo, $\schwarz f^n < 0$ para todo $n \geq 1$ por indução.
\end{proof}

\begin{lemma}
\label{lemma1}
Se $\schwarz f < 0$ e $x_0$ é um ponto de mínimo local de $\partial f$, então $\partial f(x_0) \leq 0$.
\end{lemma}

\begin{proof}
Se $\partial f(x_0) \neq 0$, então
$$(\schwarz f)(x_0) = \frac{\partial^3 f(x_0)}{\partial f(x_0)} - \frac{3}{2} \left( \frac{\partial^2 f(x_0)}{\partial f(x_0)} \right)^2 < 0.$$
Sendo $x_0$ ponto de mínimo local de $\partial f$, temos que $\partial^2 f(x_0) = 0$ e $\partial^3 f(x_0) \geq 0$ e, portanto, $\partial f(x_0) < 0$. 
\end{proof}

\begin{lemma}
\label{lemma2}
Se $\schwarz f < 0$ e $a<b<c$ são pontos fixos de $f$ com $\partial f(b) \leq 1$, então $f$ possui ponto crítico em $(a, c)$.
\end{lemma}

\begin{proof}
Pelo TVM, existem $r \in (a,b)$ e $s \in (b,c)$  tais que $\partial f(r) = \partial f(s) = 1$.
Sendo $\partial f$ contínua, $\partial f$ restrita ao intervalo $[r,s]$ possui mínimo global.
Como $b \in (r,s)$ e $\partial f(b) \leq 1$, temos que $\partial f$ possui mínimo local em $(r,s)$.
Utilizando Lema anterior e o TVI, a demonstração está concluída.
\end{proof}

\begin{lemma}
\label{lemma3}
Se $\schwarz f < 0$ e $a<b<c<d$ são pontos fixos de $f$, então $f$ possui ponto crítico em $(a,d)$.
\end{lemma}

\begin{proof}
Se $\partial f(b) \leq 1$ ou $\partial f(c) \leq 1$, o resultado é verdadeiro pelo Lema anterior.
Se $\partial f(b) > 1$ e $\partial f(c) > 1$, existem $r, t \in (b,c)$ tais que $r<t$, $f(r) > r$ e $f(t) < t$.
Pelo TVM, existe $s \in (r,t)$ tal que $\partial f(s) < 1$.
Portanto, $\partial f$ possui mínimo local em $(b,c)$.
Utilizando Lema \ref{lemma1} e o TVI, a demonstração está concluída.
\end{proof}

\begin{lemma}
Se $f$ possui finitos pontos críticos, então $f^n$ possui finitos pontos críticos para todo $n \geq 1$.
\end{lemma}
\begin{proof}
Pelo TVM, se $c \in \RR$, então $f$ possui ponto crítico entre dois elementos de $f^{-1}(c)$ e, portanto, $f^{-1}(c)$ é finito.
De modo mais geral, é possível provar por indução que $f^{-n}(c)$ é finito para todo $n \geq 1$.

Se $n \geq 1$, então $\partial f^n(x) = \prod_{k=0}^{n-1} \partial f(f^k(x)) = 0$ se, e somente se, $f^k(x)$ é ponto crítico de $f$ para algum $1 \leq k < n$.
Portanto, o conjunto de pontos críticos de $f^n$ é finito.
\end{proof}

\begin{lemma}
Se $\schwarz f < 0$ e $f$ possui finitos pontos críticos, então $f^n$ possui finitos pontos fixos para todo $n \geq 1$.
\end{lemma}

\begin{proof}
Pelo Lema \ref{lemma3}, se $f^n$ possui infinitos pontos fixos para algum $n \geq 1$, então $f^n$ possui infinitos pontos críticos, o que é um absurdo pelo Lema anterior.
\end{proof}

\begin{theorem}[Singer]
Se $\schwarz f < 0$ e $f$ possui $n$ pontos críticos, então $f$ possui no máximo $n+2$ órbitas periódicas não repulsoras.
\end{theorem}

\begin{proof}
Seja $p$ um ponto periódico não repulsor de $f$ de período $m$.
Se $g = f^m$, então $g(p) = p$ e $|\partial g(p)| \leq 1$.
Seja $K$ a componente conexa de $\basin(p) = \lbrace x : \lim_{k \to \infty} g^k(x) = p \rbrace$ que contém $p$.
Inicialmente, suponha que $K$ é limitado.

Se $|\partial g(p)| < 1$, então é possível mostrar que $K$ é aberto, $g(K) \subset K$ e $g$ preserva os pontos extremos de $K$.

Escrevendo $K = (a, b)$, se $g(a) = a$ e $g(b) = b$, então $g$ possui ponto crítico em $K$ pelo Lema \ref{lemma2}; se $g(a) = b$ e $g(b) = a$, então $g^2$ possui ponto crítico em $K$ pelo Lema \ref{lemma2}; se $g(a) = g(b)$, então $g$ possui ponto crítico em $K$ pelo TVM.

Se $|\partial g(p)| = 1$, então os pontos fixos de $g$ são isolados pelo Lema anterior e, portanto, existe uma vizinhança de $p$ que não contém outros pontos fixos de $g$.

Suponha que $\partial g(p) = 1$.
Se $\partial g(p) = -1$, a demonstração é análoga considerando $g^2$.
Se $p$ possui o comportamento de um ponto repulsor, então, para $x$ numa vizinhança de $p$, $g(x) > x$ quando $x > p$ e $g(x) < x$ quando $x < p$.
Desse modo, $1$ é um mínimo local de $\partial g$, o que é um absurdo pelo Lema \ref{lemma1} e, portanto, $p$ é atrator em pelo menos um dos lados.
Desse modo, $K$ é um intervalo não trivial, $g(K) \subset K$ e $g$ preserva os pontos extremos de $K$.
Assim, é possível concluir de maneira análoga que $g$ possui ponto crítico em $K$.

Assim, cada intervalo $K$ limitado está associado à algum ponto crítico de $f$ e, portanto, existem no máximo $n$ desses intervalos.
Não é possível obter a mesma conclusão se $K$ não é limitado, mas observando que existem no máximo dois intervalos desse tipo, a demonstração está concluída.
\end{proof}

\begin{corollary}
Se $\mu > 0$, então $F_\mu(x) = \mu x(1-x)$ possui no máximo $1$ órbita periódica não repulsora.
\end{corollary}
