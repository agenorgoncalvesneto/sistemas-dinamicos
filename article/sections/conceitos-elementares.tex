\section{Conceitos Elementares}

De maneira suficiente para os nossos objetivos, definimos um sistema dinâmico como uma função $f: X \to X$, onde $X$ é um espaço métrico.
Usualmente, $X$ será um subconjunto de $\RR$ e será considerada a distância usual.
Dado $x \in X$ e denotando por $f^n$ a composição de $f$ com ela mesma $n-1$ vezes, queremos estudar as propriedades da sequência $x, \, f(x), \, f^2(x), \, \dots$.
Para esse estudo, iniciamos definindo alguns conceitos importantes que estarão presentes durante todo o texto.

Se $p \in X$ e $f(p) = p$, então $p$ é um ponto fixo de $f$.
Se $f^n(p) = p$ para algum $n \geq 1$, então $p$ é um ponto periódico de $f$ de período $n$.
Se $f^n(p) = p$ para algum $n \geq 1$ e $f^k(p) \neq p$ para todo $1 \leq k < n$, então $p$ é um ponto periódico $f$ de período principal $n$.
O conjunto dos pontos periódicos de $f$ será denotado por $\per(f)$ e o conjunto dos pontos periódicos de $f$ de período principal $n$ será denotado por $\per_n(f)$.

Se $x \in X$, então o conjunto $\lbrace x, \, f(x), \, f^2(x), \, \dots \rbrace$, que será denotado por $\orbit(x)$, é a órbita de $x$.
Se $p$ um ponto periódico de período $n$, então o conjunto $\lbrace x \in  X: \lim_{k \to \infty} f^{kn}(x) = p \rbrace$, que será denotado por $\basin(p)$, é a bacia de atração de $p$. Por fim, o conjunto $\lbrace x \in  X: \lim_{k \to \infty} |f^k(x)| = \infty \rbrace$, que será denotado por $\basin(\infty)$, é a bacia de atração do infinito.

A Proposição \ref{prop 1-1} nos fornece uma maneira simples para verificar se uma função contínua definida num intervalo de $\RR$ possui ponto fixo.

\begin{proposition}\label{prop 1-1}
Seja $f: [a, b] \to \RR$ uma função contínua. Se $f([a, b]) \subset [a, b]$ ou $f([a, b]) \supset [a, b]$, então $f$ possui ponto fixo.
\end{proposition}

\begin{proof}
Considere a função contínua $g: [a, b] \to \RR$ dada por $g(x) = f(x) - x$. Em ambos os casos é possível verificar, pelo Teorema do Valor Intermediário (TVI), que existe $p \in [a, b]$ tal que $g(p) = 0$.
\end{proof}

Se $f$ for de classe $\class^1$, então podemos conhecer o comportamento de pontos numa vizinhança de um ponto periódico $p$ cuja derivada $Df^n(p)$ em módulo é diferente de $1$.

\begin{theorem}\label{teo 1-2}
Sejam $f: \RR \to \RR$ uma função de classe $\class^1$ e $p \in \per_n(f)$.
\begin{enumerate}
\item Se $|D f^n(p)| < 1$, então existe uma vizinhança de $p$ contida em $\basin(p)$.
\item Se $|D f^n(p)| > 1$, então existe uma vizinhança $V$ de $p$ com a seguinte propriedade: se $x \in V$ e $x \neq p$, então  $f^{kn}(x) \notin V$ para algum $k \geq 1$. 
\end{enumerate}
\end{theorem}

\begin{proof}
\begin{enumerate}\item[]
\item Sendo $D f^n$ contínua, existe $\varepsilon > 0$ tal que $|D f^n(x)| \leq \lambda < 1$ para todo $x \in (p - \varepsilon,\, p + \varepsilon)$. Pelo TVM, se $x \in (p - \varepsilon,\, p + \varepsilon)$, então $|f^n(x) - p| \leq \lambda|x - p|$. Por indução, $|f^{kn}(x) - p| \leq \lambda^k|x - p|$ para todo $k \geq 1$. Desse modo, $\lim_{k \to \infty} f^{kn}(x) = p$.
\item Exercício.
\end{enumerate}
\end{proof}

\begin{definition}\label{def 1-3}
Sejam $f: \RR \to \RR$ uma função de classe $\class^1$ e $p \in \per_n(f)$.
\begin{enumerate}[label=\roman*.]
\item Se $|D f^n(p)| < 1$, então $p$ é um ponto atrator.
\item Se $|D f^n(p)| > 1$, então $p$ é um ponto repulsor.
\end{enumerate}
\end{definition}

A Definição \ref{def 1-3}, que segue naturalmente do Teorema \ref{teo 1-2}, pode ser estendida para órbitas de pontos periódicos. De fato, é imediato verificar pela Regra da Cadeia que são iguais as derivadas de todos os pontos de uma órbita de um ponto periódico. Desse modo, se a órbita de um ponto periódico possui um ponto atrator, então dizemos que ela é uma órbita atratora. Uma definição análoga segue se a órbita possui um ponto repulsor.