\section{Conceitos Elementares}

De maneira suficiente para os nossos objetivos, definimos um sistema dinâmico como uma função $f: X \to X$, onde $X$ é um espaço métrico.
Usualmente, $X$ será um subconjunto de $\RR$ com a distância usual.
Dado $x \in X$, nosso objetivo é estudar as propriedades da sequência definida recursivamente por
$$f^0(x) = x \quad \text{ e } \quad f^k(x) = f(f^{k-1}(x))$$
para todo $k \geq 1$. Para isso, vamos iniciar com algumas definições que estarão presentes durante todo o texto.

Se $p \in X$ e $f(p) = p$, então $p$ é um ponto fixo de $f$.
Se $f^n(p) = p$ para algum $n \geq 1$, então $p$ é um ponto periódico de $f$ de período $n$.
Se $f^n(p) = p$ para algum $n \geq 1$ e $f^k(p) \neq p$ para todo $1 \leq k < n$, então $p$ é um ponto periódico $f$ de período principal $n$.
O conjunto dos pontos periódicos de $f$ será denotado por $\per(f)$ e o conjunto dos pontos periódicos de $f$ de período principal $n$ será denotado por $\per_n(f)$.

Se $x \in X$, então $\orbit(x) = \lbrace f^k(x) : k \geq 0 \rbrace$ é a órbita de $x$.
Se $p$ um ponto periódico de período $n$, então $\basin(p) = \lbrace x \in  X: \lim_{k \to \infty} f^{kn}(x) = p \rbrace$ é o conjunto estável de $p$.
Por fim, o conjunto $\basin(\infty) = \lbrace x \in  X: \lim_{k \to \infty} |f^k(x)| = \infty \rbrace$ é o conjunto estável do infinito.

A Proposição \ref{prop 1-1} nos fornece uma maneira útil de verificar se uma função contínua definida num intervalo compacto de $\RR$ possui ponto fixo.

\begin{proposition}\label{prop 1-1}
Seja $f: [a, b] \to \RR$ uma função contínua. Se $f([a, b]) \subset [a, b]$ ou $f([a, b]) \supset [a, b]$, então $f$ possui ponto fixo.
\end{proposition}

\begin{proof}
Considere a função contínua $g: [a, b] \to \RR$ dada por $g(x) = f(x) - x$. Em ambos os casos é possível verificar, pelo Teorema do Valor Intermediário (TVI), que existe $p \in [a, b]$ tal que $g(p) = 0$.
\end{proof}

Se a função for de classe $\class^1$, podemos conhecer o comportamento dos pontos numa vizinhança de um ponto fixo cuja derivada em módulo é diferente de $1$.

\begin{theorem}\label{teo 1-2}
Sejam $f: \RR \to \RR$ uma função de classe $\class^1$ e $p \in \per_n(f)$.
\begin{enumerate}
\item Se $|D f^n(p)| < 1$, então existe uma vizinhança de $p$ contida em $\basin(p)$.
\item Se $|D f^n(p)| > 1$, então existe uma vizinhança $V$ de $p$ com a seguinte propriedade:\\se $x \in V \backslash \lbrace p \rbrace$, então  $f^{kn}(x) \notin V$ para algum $k \geq 1$. 
\end{enumerate}
\end{theorem}

\begin{proof}
\begin{enumerate}\item[]
\item Sendo $D f^n$ contínua, existe uma vizinhança $V$ de $p$ tal que $|D f^n(V)| \leq \lambda < 1$.
Pelo Teorema do Valor Médio (TVM), se $x \in V$, então $|f^n(x) - p| \leq \lambda|x - p|$.
Por indução, $|f^{kn}(x) - p| \leq \lambda^k|x - p|$ para todo $k \geq 1$ e, portanto, $\lim_{k \to \infty} f^{kn}(x) = p$.
\item Exercício.
\end{enumerate}
\end{proof}



\begin{definition}\label{def 1-3}
Sejam $f: \RR \to \RR$ uma função de classe $\class^1$ e $p \in \per_n(f)$.
\begin{enumerate}[label=\roman*.]
\item Se $|D f^n(p)| < 1$, então $p$ é um ponto atrator.
\item Se $|D f^n(p)| > 1$, então $p$ é um ponto repulsor.
\end{enumerate}
\end{definition}

A Definição \ref{def 1-3}, que segue naturalmente do Teorema \ref{teo 1-2}, pode ser estendida para órbitas de pontos periódicos. De fato, pela Regra da Cadeia, é possível verificar que se um ponto é atrator (repulsor), então todos os pontos de sua órbita também são atratores (repulsores) e, nesse caso, dizemos que sua órbita é atratora (repulsora).
