\section{Conceitos Elementares}

Seja $f: X \to X$ uma função, onde $X$ é um espaço métrico.
Dado $x \in X$ e denotando por $f^n$ a $n$-ésima composição de $f$ com ela mesma, queremos estudar as propriedades da sequência $(x, \, f(x), \, f^2(x), \, \dots)$.

\begin{definition}
\begin{enumerate}[label=\alph*.]\item[]
\item Se $x \in X$, então $\orbit(x) = \lbrace x, \, f(x), \, f^2(x), \, \dots \rbrace$ é a órbita de $x$.
\item Se $p \in X$ e $f^n(p) = p$ para algum $n > 0$, então $p$ é um ponto periódico de período $n$. Em particular, se $n = 1$, então $p$ é um ponto fixo. Denotaremos por $\per(f)$ o conjunto de todos os pontos periódicos.
\item Se $p$ é um ponto periódico de período $n$ e $f^k(p) \neq p$ para todo $0 < k < n$, então $n$ é o período principal de $p$. Denotaremos por $\per_n(f)$ o conjunto de todos os pontos periódicos de período principal $n$.
\item Se $p$ um ponto periódico de período $n$, então $\basin(p) = \lbrace x \in  X: \lim_{k \to \infty} f^{kn}(x) = p \rbrace$ é a bacia de atração de $p$. Além disso, $\basin(\infty) = \lbrace x \in  X: \lim_{k \to \infty} |f^k(x)| = \infty \rbrace$ é a bacia de atração do $\infty$.
\end{enumerate}
\end{definition}

\begin{proposition}
Seja $f: [a, b] \to \RR$ uma função contínua. Se $f([a, b]) \subset [a, b]$ ou $f([a, b]) \supset [a, b]$, então $f$ possui ponto fixo.
\end{proposition}

\begin{proof}
Suponha que $f([a, b]) \subset [a, b]$. Considere a função $g(x) = f(x) - x$ definida em $[a, b]$. Observando que $g(a) \geq 0$ e $g(b) \leq 0$ e utilizando o TVI, existe $p \in [a, b]$ tal que $g(p) = 0$.

Suponha que $f([a, b]) \supset [a, b]$ e sejam $c, d \in [a, b]$ tais que $f(c) = a$ e $f(d) = b$. Considere a função $g(x) = f(x) - x$ definida em $[a, b]$. Observando que $g(c) \leq 0$ e $g(d) \geq 0$ e utilizando o TVI, existe $p \in [a, b]$ tal que $g(p) = 0$.
\end{proof}

\begin{definition}
Sejam $f: [a, b] \to [a, b]$ uma função de classe $\class^1$ e $p \in \per_n(f)$.
\begin{enumerate}[label=\alph*.]
\item Se $|D f^n(p)| \neq 1$, então $p$ é um ponto hiperbólico.
\item Se $|D f^n(p)| < 1$, então $p$ é um ponto atrator.
\item Se $|D f^n(p)| > 1$, então $p$ é um ponto repulsor.
\end{enumerate}
\end{definition}

\begin{theorem}
Sejam $f: [a, b] \to [a, b]$ uma função de classe $\class^1$ e $p \in \per_n(f)$.
\begin{enumerate}
\item Se $p$ é um ponto atrator, então existe uma vizinhança de $p$ contida em $\basin(p)$.
\item Se $p$ é um ponto repulsor, então existe uma vizinhança $V$ de $p$ com a seguinte propriedade: se $x \in V$ e $x \neq p$, então  $f^{kn}(x) \notin V$ para algum $k \geq 1$. 
\end{enumerate}
\end{theorem}

\begin{proof}
\begin{enumerate}\item[]
\item Sendo $D f^n$ contínua, existe $\varepsilon > 0$ tal que $|D f^n(x)| \leq \lambda < 1$ para todo $x \in (p - \varepsilon,\, p + \varepsilon)$. Pelo TVM, se $x \in (p - \varepsilon,\, p + \varepsilon)$, então $|f^n(x) - p| \leq \lambda|x - p|$. Por indução, $|f^{kn}(x) - p| \leq \lambda^k|x - p|$ para todo $k \geq 1$. Desse modo, $\lim_{k \to \infty} f^{kn}(x) = p$.
\item Demonstração análoga.
\end{enumerate}
\end{proof}
