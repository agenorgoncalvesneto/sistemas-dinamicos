\section{Conceitos Iniciais}

\begin{definition}
Sejam $f: X \to X$ uma função e $p \in X$. Se $f^n(p) = p$ para algum $n \geq 1$, dizemos que $p$ é um ponto periódico de período $n$.
\end{definition}

\begin{definition}
Sejam $f: X \to X$ uma função e $x \in X$. Dizemos que $\oo(x) = \{ x, f(x), f^2(x), \dots \}$ é a órbita de $x$.
\end{definition}

\begin{definition}
Sejam $f: X \to X$ uma função e $p$ um ponto periódico de período $n$. Dizemos que $\bb (p) = \{ x \in  X: \lim_{j \to \infty} f^{jn}(x) = p \}$ é a bacia de atração de $p$ e que $\bb (\infty) = \{ x \in  X: \lim_{j \to \infty} |f^j(x)| = \infty \}$ é a bacia de atração de $\infty$.
\end{definition}

\begin{proposition}
Seja $f: X \to X$ uma função. Se $p$ e $q$ são pontos periódicos distintos, então $\bb(p) \cap \bb(q) = \emptyset$.
\end{proposition}

\begin{proof}
Sejam $n_1$, $n_2$ os períodos de $p_1$, $p_2$, respectivamente. Suponha que exista $x \in W^s(p_1) \cap W^s(p_2)$. Sabemos que $|f^{kn_1}(x) - p_1| \to 0$ e $|f^{kn_2}(x) - p_2| \to 0$, quando $k \to \infty$. Desse modo, dado $\varepsilon > 0$ existe $N \geq 1$ tal que $|f^{kn_1}(x) - p_1| < \frac{\varepsilon}{2}$ e $|f^{kn_2}(x) - p_2| < \frac{\epsilon}{2}$ para todo $k > N$. Portanto, $|p_1 - p_2| = |p_1 - f^{kn_1n_2}(x) + f^{kn_1n_2}(x) - p_2| \leq |f^{kn_2n_1}(x) - p_1| + |f^{kn_1n_2}(x) - p_2| < \varepsilon$. Temos então que $p = q$, pois $\varepsilon$ é arbitrário. Absurdo.
\end{proof}

\begin{proposition} \label{prop 1-1}
Seja $f: [a, b] \to \RR$ uma função contínua. Se $f([a, b]) \subset [a, b]$ ou $f([a, b]) \supset [a, b]$, então $f$ possui ponto fixo.
\end{proposition}

\begin{proof}
Seja $I = [a, b]$. Suponha que $f(I) \subset I$. Considere a função contínua $g(x) = f(x) - x$ definida em $I$. Como $f(a), f(b) \in I$, temos que $g(a) = f(a) - a \geq 0$ e $g(b) = f(b) - b \leq 0$. Pelo Teorema do Valor Intermediário, existe $p \in I$ tal que $g(p) = f(p) -p = 0$. Desse modo, $p$ é ponto fixo de $f$.

Suponha que $f(I) \supset I$. Por definição, existem $c, d \in I$ tais que $f(c) = a$ e$f(d) = b$. Considere a função contínua $g(x) = f(x) - x$ definida em $I$. Temos que $g(c) = a - c \leq 0$ e $g(d) = b - d \geq 0$. Pelo Teorema do Valor Intermediário, existe $p \in I$ tal que $g(p) = f(p) - p = 0$. Desse modo, $p$ é ponto fixo de $f$.
\end{proof}

\begin{theorem}
Seja $f:[a, b] \to [a, b]$ uma função derivável. Se $|Df(x)| < 1$ para todo $x \in [a, b]$, então $f$ admite um único ponto fixo. Além disso, $|f(x) - f(y)| < |x - y|$ para todo $x, y \in [a, b]$ distintos.
\end{theorem}

\begin{proof}
Sejam $x, y \in I$, $x < y$. Pelo Teorema do Valor Médio, existe $c \in [x, y]$ tal que $f(x) - f(y) = f'(c)(x - y)$. Portanto, $|f(x) - f(y)| = |f'(c)||x - y| < |x - y|$.

Pela Proposição \ref{prop 1-1}, $f$ admite um ponto fixo $p$. Suponha que exista um ponto fixo $q$ diferente de $p$. Então, pela primeira parte da demonstração, $|p - q| = |f(p) - f(q)| < |p - q|$. Absurdo.
\end{proof}

\begin{definition}
Sejam $f: [a, b] \to [a, b]$ uma função derivável e $p$ um ponto periódico de período principal $n$. Dizemos que $p$ é um ponto hiperbólico se $|Df^n(p)| \neq 1$. Dizemos que $p$ é um ponto atrator se $|Df^n(p)| < 1$ e que é um ponto repulsor se $|Df^n(p)| > 1$.
\end{definition}

\begin{theorem}
Sejam $f: [a, b] \to [a, b]$ uma função de classe $\cc^1$ e $p$ um ponto periódico de período principal $n$.
\begin{enumerate}
\item Se $p$ é um ponto atrator, então existe uma vizinhança de $p$ contida em $\bb(p)$.
\item Se $p$ é um ponto repulsor, então existe uma vizinhança $V$ de $p$ tal que, se $x \in V$ e $x \neq p$, então  $f^{jn}(x) \notin V$ para algum $j \geq 1$. 
\end{enumerate}
\end{theorem}

\begin{proof}
Suponha que $p$ é um ponto hiperbólico atrator. Como $f'$ é contínua, existe $\varepsilon > 0$ tal que $|(f^n)'(x)| \leq \lambda < 1$ para todo $x \in (p - \varepsilon, p + \varepsilon)$. Pelo Teorema do Valor Médio, se $x \in U$ então $|f^n(x) - p| = |f^n(x) - f^n(p)| \leq \lambda|x - p|$. Por indução, $|f^{kn}(x) - p| \leq \lambda^k|x - p|$. Desse modo, $f^{kn}(x) \to p$ quando $k \to \infty$.

Suponha que $p$ é ponto hiperbólico repulsor. De maneira análoga, existe $\varepsilon > 0$ tal que $|(f^n)'(x)| \geq \lambda > 1$ para todo $x \in (p- \varepsilon, p + \varepsilon)$. Fixado $x \in (p - \varepsilon, p + \varepsilon)$, $x \neq p$, suponha que $f^{kn}(x) \in (p - \varepsilon, p + \varepsilon)$ para todo $k \geq 1$. Pelo Teorema do Valor Médio, $|f^{kn}(x) - p| \geq \lambda^k|x - p|$ para todo $k \geq 1$. Absurdo, pois $\lambda^k|x - p| \to \infty$ quando $k \to \infty$.
\end{proof}
