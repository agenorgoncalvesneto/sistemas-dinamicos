\section{Conceitos Iniciais}

\begin{definition}
Sejam $f: X \to X$ uma função e $p \in X$. Se $f^n(p) = p$ para algum $n \geq 1$, dizemos que $p$ é um ponto periódico de período $n$.
\end{definition}

\begin{definition}
Sejam $f: X \to X$ uma função e $x \in X$. Dizemos que $\oo(x) = \{ x, f(x), f^2(x), \dots \}$ é a órbita de $x$.
\end{definition}

\begin{definition}
Sejam $f: X \to X$ uma função e $p$ um ponto periódico de período $n$. Dizemos que $\bb (p) = \{ x \in  X: \lim_{k \to \infty} f^{kn}(x) = p \}$ é a bacia de atração de $p$ e que $\bb (\infty) = \{ x \in  X: \lim_{k \to \infty} |f^k(x)| = \infty \}$ é a bacia de atração de $\infty$.
\end{definition}

\begin{proposition}
Seja $f: X \to X$ uma função. Se $p$ e $q$ são pontos periódicos distintos, então $\bb(p) \cap \bb(q) = \emptyset$.
\end{proposition}

\begin{proof}
Sejam $m$ e $n$ os períodos de $p$ e $q$, respectivamente. Se existe $x \in \bb(p) \cap \bb(q)$, então
$$p = \lim_{k \to \infty} f^{(kn)m}(x) = \lim_{k \to \infty} f^{(km)n}(x) = q.$$
\end{proof}

\begin{proposition} \label{prop 1-1}
Seja $f: [a, b] \to \RR$ uma função contínua. Se $f([a, b]) \subset [a, b]$ ou $f([a, b]) \supset [a, b]$, então $f$ possui ponto fixo.
\end{proposition}

\begin{proof}
Suponha que $f([a, b]) \subset [a, b]$. Considere a função $g(x) = f(x) - x$ definida em $[a, b]$. Observando que $g(a) \geq 0$ e $g(b) \leq 0$ e utilizando o TVI, existe $p \in [a, b]$ tal que $g(p) = 0$.

Suponha que $f([a, b]) \supset [a, b]$ e sejam $c, d \in [a, b]$ tais que $f(c) = a$ e $f(d) = b$. Considere a função $g(x) = f(x) - x$ definida em $[a, b]$. Observando que $g(c) \leq 0$ e $g(d) \geq 0$ e utilizando o TVI, existe $p \in [a, b]$ tal que $g(p) = 0$.
\end{proof}

\begin{definition}
Sejam $f: [a, b] \to [a, b]$ uma função derivável e $p$ um ponto periódico de período principal $n$. Dizemos que $p$ é um ponto hiperbólico se $|\partial f^n(p)| \neq 1$. Dizemos que $p$ é um ponto atrator se $|\partial f^n(p)| < 1$ e que $p$ é um ponto repulsor se $|\partial f^n(p)| > 1$.
\end{definition}

\begin{theorem}
Sejam $f: [a, b] \to [a, b]$ uma função de classe $\cc^1$ e $p$ um ponto periódico de período principal $n$.
\begin{enumerate}
\item Se $p$ é um ponto atrator, então existe uma vizinhança de $p$ contida em $\bb(p)$.
\item Se $p$ é um ponto repulsor, então existe uma vizinhança de $p$ com a seguinte propriedade: se $x$ pertence à essa vizinhança e $x \neq p$, então  $f^{kn}(x) \notin V$ para algum $k \geq 1$. 
\end{enumerate}
\end{theorem}

\begin{proof}
\begin{enumerate}
\item Sendo $\partial f$ é contínua, existe $\varepsilon > 0$ tal que $|\partial f^n(x)| \leq \lambda < 1$ para todo $x \in (p - \varepsilon,\, p + \varepsilon)$. Pelo TVM, se $x \in (p - \varepsilon,\, p + \varepsilon)$, então $|f^n(x) - p| \leq \lambda|x - p|$. Por indução, $|f^{kn}(x) - p| \leq \lambda^k|x - p|$ para todo $k \geq 1$. Desse modo, $\lim_{k \to \infty} f^{kn}(x) = p$.
\item Demonstração análoga.
\end{enumerate}
\end{proof}
