\subsection{Estudo Inicial}

Iniciamos o estudo da família quadrática observando que $h$ possui dois pontos fixos.

\begin{proposition}
Se $\mu > 1$, então $h(0) = 0$ e $h(p_\mu) = p_\mu$, onde $p_\mu = \frac{\mu-1}{\mu}$.
\end{proposition}

\begin{proof}
Exercício.
\end{proof}

A Proposição \ref{prop 2-1} nos permite restringir o estudo da dinâmica de $h$ ao intervalo $[0, 1]$, pois conhecemos o comportamento dos pontos que não pertencem à esse intervalo.

\begin{proposition}\label{prop 2-1}
Se $\mu > 1$, então $(-\infty, 0) \cup (1, \infty) \subset \basin(\infty)$.
\end{proposition}

\begin{proof}
Basta observar que a sequência $x, \, h(x), \, h^2(x), \, \dots$ é estritamente decrescente e ilimitada quando $x \in (-\infty, 0)$.
\end{proof}

A Proposição \ref{prop 2-2}, que pode visualizada graficamente, nos mostra que a dinâmica de $h$ é simples para valores baixo de $\mu$.

\begin{proposition}\label{prop 2-2}
Se $\mu \in (1, 3)$, então
\begin{enumerate}
\item $0$ é um ponto repulsor e $p_\mu$ é um ponto atrator.
\item $(0, 1) \subset \basin(p_\mu)$.
\end{enumerate}
\end{proposition}

\begin{proof}
Exercício.
\end{proof}

Desse modo, a dinâmica de $h$ está completamente determinada quando $\mu \in (1, 3)$. De fato,
$$\basin(0) = \lbrace 0, 1 \rbrace, \quad \basin(p_\mu) = (0, 1) \quad \text{e}  \quad \basin(\infty) = (-\infty, 0) \cup (1, \infty).$$
