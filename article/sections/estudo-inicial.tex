\subsection{Estudo Inicial}

\begin{proposition}
Se $\mu > 1$, então
\begin{enumerate}
\item $h(1) = h(0) = 0$.
\item $h(\frac{1}{\mu}) = h(p_\mu) = p_\mu$, onde $p_\mu = \frac{\mu-1}{\mu}$.
\item $0 < p_\mu < 1$.
\end{enumerate}
\end{proposition}

\begin{proposition}
Se $\mu > 1$, então $(-\infty, 0) \cup (1, \infty) \subset \basin(\infty)$.
\end{proposition}

\begin{proof}
Inicialmente, se $x \in (1, \infty)$, então $h(x) \in (-\infty, 0)$. Por fim, observamos que a sequência  $(x,\, h(x),\, h^2(x),\, \dots)$ é estritamente decrescente e ilimitada quando $x \in (-\infty, 0)$.
\end{proof}

\begin{proposition}
Se $1 < \mu < 3$, então
\begin{enumerate}
\item $0$ é um ponto repulsor e $p_\mu$ é um ponto atrator.
\item $(0, 1) \subset \basin(p_\mu)$.
\end{enumerate}
\end{proposition}
