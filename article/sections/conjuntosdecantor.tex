\subsection{Conjuntos de Cantor}

Se $\mu > 4$, então $F(\frac{1}{2}) = \frac{\mu}{4} > 1$, ou seja, existem pontos em $[0, 1]$ que não permanecem em $[0, 1]$ após uma iteração de $F$. Em vista da Proposição \ref{prop 2-1}, tais pontos pertencem ao conjunto $\bb(\infty)$. De modo mais geral, se um ponto de $[0, 1]$ não permanece em $[0, 1]$ após um número finito de iterações, então ele pertence ao conjunto $\bb(\infty)$. 

Desse modo, considere o conjunto $\Lambda_n = \{x \in [0, 1] : F^n(x) \in [0, 1]\}$ formado pelos pontos que permanecem em $[0, 1]$ após $n$ iterações de $F$ e considere o conjunto $\Lambda = \cap_{n=0}^{\infty} \Lambda_n = \{x \in [0, 1] : F^n(x) \in [0, 1] \textrm{ para todo } n \geq 1\}$, que é formado pelos pontos de $[0, 1]$ que sempre permanecem em $[0, 1]$ por iterações de $F$. Observe que $\Lambda_n \supset \Lambda_{n+1}$, para todo $n \geq 1$, pois se $F^{n+1}(x) = F(F^n(x)) \in [0,1]$, então $F^n(x) \in [0,1]$.

\begin{proposition} \label{proposicao conjuntosdecantor 1}
Se $\mu > 4$, então
\begin{enumerate}
\item $\Lambda_1 = [0, x_1] \cup [x_2, 1]$, onde $x_1 = \frac{1}{2} - \frac{\sqrt{\mu^2 - 4\mu}}{2\mu}$ e $x_2 = \frac{1}{2} + \frac{\sqrt{\mu^2 - 4\mu}}{2\mu}$.
\item $\Lambda_n$ é a união de $2^n$ intervalos fechados disjuntos e $F^n: [a, b] \to [0, 1]$ é bijetora, onde $[a, b]$ é um desses intervalos.
\end{enumerate}
\end{proposition}

\begin{proof}
\begin{enumerate}
\item[]
\item Basta resolver a equação de segundo grau $\mu x(1-x) = 1$.
\end{enumerate}
Analisando $F'$ observamos que $F$ é estritamente crescente no intervalo $\left[0, \frac{1}{2}\right]$ e estritamente decrescente no intervalo $\left[\frac{1}{2}, 1\right]$. Como $F(0) = F(1) = 0$ e $F\left(\frac{1}{2}\right) > 1$, o Teorema do Valor Intermediário garante que existem $x_1 \in \left(0, \frac{1}{2}\right)$ e $x_2 \in \left(\frac{1}{2}, 1\right)$ tais que $F(x_1) = F(x_2) = 1$. Os valores de $x_1$ e $x_2$ são encontrados resolvendo a equação de segundo grau $\mu x(1-x) = 1$. Logo, $F([0, x_1]) = F([x_2, 1]) = [0, 1]$ e $F(x) > 1$ para todo $x \in (x_1, x_2)$. Portanto, $\Lambda_1 = [0, x_1] \cup [x_2, 1]$ e o item 1 está demonstrado.

A demonstração dos itens $2$ e $3$ será feita por indução. De acordo com a primeira parte, $\Lambda_1$ é a união de $2^1 = 2$ intervalos fechados disjuntos e $F$ restrita é cada um desses intervalos é uma bijeção com o intervalo $[0,1]$.

Suponha que $\Lambda_{k-1}$ é a união de $2^{k-1}$ intervalos fechados disjuntos de modo que $F^{k-1}: [a, b] \to [0, 1]$ é bijetora para todo intervalo $[a, b]$ que forma $\Lambda_{k-1}$. Sendo $F^{k-1}$ bijetora, $(F^{k-1})'(x) > 0$ ou $(F^{k-1})'(x) < 0$ para todo $x \in [a, b]$. Como as demonstrações para os dois casos são análogas, podemos supor que $(F^{k-1})'(x) > 0$.

Como $F^{k-1}$ é estritamente crescente, o Teorema do Valor Intermediário afirma que existem únicos $\overline{x_1}, \overline{x_2} \in (a, b)$ tais que
\begin{enumerate}[label=(\alph*)]
\item$a < \overline{x_1} < \overline{x_2} < b$,
\item$F^{k-1}([a, \overline{x_1}]) = [0, x_1]$,
\item$F^{k-1}((\overline{x_1}, \overline{x_2})) = (x_1, x_2)$ e
\item$F^{k-1}([\overline{x_2}, 1]) = [x_2, 1]$.
\end{enumerate}

As condições acima garantem que os intervalos $[a, \overline{x_1}]$, $[\overline{x_2}, b]$ são disjuntos e que $F^k(x) = F(F^{k-1}(x)) > 1$ para todo $x \in (\overline{x_1}, \overline{x_2})$. Também, temos que $F^k([a, \overline{x_1}]) = F([0, x_1]) = [0, 1]$ e, analogamente, $F^k([\overline{x_2}, 1]) = [0, 1]$. Além disso, $$(F^k)'([a, \overline{x_1}]) = F'(F^{k-1}([a, \overline{x_1}]))(F^{k-1})'([a, \overline{x_1}]) = F'([0, x_1])(F^{k-1})'([a, \overline{x_1}]) > 0$$ e, analogamente, $$(F^k)'([\overline{x_2}, 1]) = F'([x_2, 1])(F^{k-1})'([\overline{x_2}, 1]) < 0.$$ Logo, $F^k$ é uma bijeção entre $[a, \overline{x_1}]$ e $[0, 1]$ e entre $[\overline{x_2}, 1]$ e $[0, 1]$.

Portanto, a partir de cada intervalo fechado de $\Lambda_{k-1}$, construímos dois novos intervalos fechados disjuntos tais que $F^k$ restrita em cada um desses intervalos é um bijeção com $[0, 1]$ e, dessa maneira, esses intervalos estão contidos em $\Lambda_k$. Desse modo, se $\Lambda_{k-1}$ é formado por $2^{k-1}$ intervalos fechados disjuntos, então $\Lambda_k$ é formado por $2 \times 2^{k-1} = 2^k$ intervalos fechados disjuntos. Assim, o resultado está provado.
\end{proof}

\begin{definition}
Seja $\Gamma \subset \RR$ um conjunto não vazio. Dizemos que $\Gamma$ é um conjunto de Cantor se as seguintes condições são válidas:
\begin{enumerate}[label=\roman*.]
\item $\Gamma$ é compacto.
\item $\Gamma$ não possui intervalos.
\item Todo ponto de $\Gamma$ é um ponto de acumulação de $\Gamma$.
\end{enumerate}
\end{definition}

\begin{lemma} \label{lema conjuntosdecantor 1}
Se $\mu > 2 + \sqrt{5}$, então
\begin{enumerate}
\item  existe $\lambda > 1$ tal que $|\partial F(x)| > \lambda$ para todo $x \in \Lambda_1$.
\item $b - a < \frac{1}{\lambda^n}$, onde $[a, b]$ é um dos intervalos que formam $\Lambda_n$.
\item dados $x \in \Lambda$ e $\varepsilon > 0$, existe um intervalo $[a, b] \subset \Lambda_n$ para algum $n \geq 1$ tal que $x \in [a, b]$, $b - a < \varepsilon$ e $F^n: [a, b] \to [0,1]$ é bijetora.
\end{enumerate}
\end{lemma}

\begin{proof}
\begin{enumerate}
\item Inicialmente, observamos que $\mu^2 - 4\mu > 1$ quando $\mu > 2 + \sqrt{5}$. Desse modo, $F'(x_1) = \sqrt{\mu^2 - 4\mu} > 1$ e $F'(x_2) = -\sqrt{\mu^2 - 4\mu} < -1$, onde $x_1$ e $x_2$ são como na Proposição \ref{proposicao conjuntosdecantor 1}. Observamos também que $F'$ é estritamente decrescente, pois $F''(x) = -2\mu < 0$.  Portanto, $F'(x) \geq F'(x_1) > 1$ para todo $x \in [0, x_1]$ e $F'(x) \leq F'(x_2) < -1$ para todo $x \in [x_2, 1]$. De acordo com a Proposição \ref{proposicao conjuntosdecantor 1}, $\Lambda_1 = [0, x_1] \cup [x_2, 1]$ e, desse modo, $|F'(x)| > 1$ para todo $x \in \Lambda_1$. Sendo $F'$ contínua e $\Lambda_1$ compacto, existe $\lambda > 1$ tal que $|F'(x)| > \lambda$ para todo $x \in \Lambda_1$.

\item De acordo com a Proposição \ref{proposicao conjuntosdecantor 1}, $\Lambda_n$ é formado pela união de $2^n$ intervalos disjuntos. Seja $[a, b]$ um desses intervalos. Se $x \in [a, b]$, em particular $F^k(x) \in \Lambda_1$ para todo $0 \leq k < n$. Desse modo, de acordo com o item anterior, temos que $(F^n)'(x) = F'(F^{n-1}(x)) \times F'(F^{n-2}(x)) \times \cdots \times F'(x) > \lambda^n$.

Pelo Teorema do Valor Médio, existe $c \in [a, b]$ tal que $$|F^n(b) - F^n(a)| = |(F^n)'(c)||b - a| > \lambda^n|b - a|$$ Como $F^n: [a, b] \to [0 ,1]$ é contínua e bijetora, temos que $|F^n(b) - F^n(a)| = 1$. Desse modo, $|b - a| < \frac{1}{\lambda^n}$ e a afirmação está provada.

\item  Sejam $x \in \Lambda$, $\varepsilon > 0$ e $n \geq 1$ tal que $\frac{1}{\lambda^n} < \varepsilon$, onde $\lambda > 1$ é como no primeiro item. Em particular, $x \in \Lambda_n$. Seja $I$ um dos intervalos que formam $\Lambda_n$ e que contém $x$. Pelo item anterior, o tamanho de $I$ é menor que $\varepsilon$. Além disso, pela Proposição \ref{proposicao conjuntosdecantor 1}, $F^n: I \to [0,1]$ é bijetora e, portanto, a afirmação está provada.
\end{enumerate}
\end{proof} 

\begin{theorem} \label{teorema conjuntosdecantor 1}
Se $\mu > 2 + \sqrt{5}$, então $\Lambda$ é um conjunto de Cantor.
\end{theorem}

\begin{proof}
$\Lambda$ é não vazio pois $0 \in \Lambda$, é limitado pois $\Lambda_1 \subset [0, 1]$ e é fechado pois é intersecção de conjuntos fechados.

Agora, suponha que $\Lambda$ contém algum intervalo. Então, existem $x, y \in I$, $x < y$, tais que $[x, y] \subset \Lambda$. Seja $k$ tal que $\frac{1}{\lambda^k} < |x - y|$. Em particular, $[x, y] \subset \Lambda_k$. Mas, de acordo com o Lema \ref{lema conjuntosdecantor 1}, os intervalos de $\Lambda_k$ possuem tamanho menor que $\frac{1}{\lambda^k}$. Absurdo e, portanto, $\Lambda$ não possui intervalos.

Por fim, observe que, se $x$ é um ponto extremo de algum intervalo de $\Lambda_n$, então $x \in \Lambda$ pois $F^{n+1}(x) = 0$. Sejam $x \in \Lambda$, $\varepsilon > 0$ e $k \geq 1$ tal que $\frac{1}{\lambda^k} < \varepsilon$. Em particular, $x \in \Lambda_k$ e, portanto, $x$ é elemento de algum intervalo cujo tamanho é menor que $\varepsilon$, de acordo com o Lema \ref{lema conjuntosdecantor 1}. Portanto, existe $y \in \Lambda$ ponto extremo do intervalo que contém $x$ tal que $|x - y| < \varepsilon$. Como $\varepsilon$ é arbitrário, concluímos que $x$ é um ponto de acumulação de $\Lambda$.
\end{proof}
