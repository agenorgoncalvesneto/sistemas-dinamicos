\subsection{Conjuntos de Cantor}

Se $\mu > 4$, então $F(\frac{1}{2}) = \frac{\mu}{4} > 1$, ou seja, existem pontos em $[0, 1]$ que não permanecem em $[0, 1]$ após uma iteração de $F$. Em vista da Proposição \ref{prop 2-1}, tais pontos pertencem ao conjunto $\bb(\infty)$. De modo mais geral, se um ponto de $[0, 1]$ não permanece em $[0, 1]$ após um número finito de iterações, então ele pertence ao conjunto $\bb(\infty)$. 

Desse modo, considere o conjunto $\Lambda_n = \{x \in [0, 1] : F^n(x) \in [0, 1]\}$ formado pelos pontos que permanecem em $[0, 1]$ após $n$ iterações de $F$ e o conjunto $\Lambda =  \{x \in [0, 1] : F^n(x) \in [0, 1] \textrm{ para todo } n \geq 1\}$ formado pelos pontos de $[0, 1]$ que sempre permanecem em $[0, 1]$ por iterações de $F$.

Podemos verificar que $\Lambda_1 = [0, x_1] \cup [x_2, 1]$, onde $x_1 = \frac{1}{2} - \frac{\sqrt{\mu^2 - 4\mu}}{2\mu}$ e $x_2 = \frac{1}{2} + \frac{\sqrt{\mu^2 - 4\mu}}{2\mu}$. De modo mais geral, temos o resultado a seguir.

\begin{proposition} \label{proposicao conjuntosdecantor 1}
Se $\mu > 4$, então $\Lambda_n$ é a união de $2^n$ intervalos fechados disjuntos e $F^n: [a, b] \to [0, 1]$ é bijetora, onde $[a, b]$ é um desses intervalos.
\end{proposition}

\begin{proof}
Suponha que $\Lambda_{k-1}$ é a união de $2^{k-1}$ intervalos fechados disjuntos e que $F^{k-1}: [a, b] \to [0, 1]$ é bijetora, onde $[a, b]$ é um desses intervalos.
Suponha que $F^{k-1}$ é estritamente crescente em $[a, b]$; se $F^{k-1}$ é estritamente decrescente em $[a, b]$, a demonstração é análoga.

Inicialmente, observamos que existem $x_1' < x_2'$ tais que $F^{k-1}([a, x_1']) = [0, x_1]$, $F^{k-1}((x_1', x_2')) = (x_1, x_2)$ e $F^{k-1}([x_2', 1]) = [x_2, 1]$.
Desse modo, os intervalos $[a, x_1']$ e $[x_2', b]$ são disjuntos.
Além disso, temos que $F^k([a, x_1']) = [0, 1]$, $F^k((x_1', x_2')) > 1$  e $F^k([x_2', 1]) = [0, 1]$.
Pela Regra da Cadeia, temos também que $(F^k)'([a, x_1']) = F'([0, x_1])(F^{k-1})'([a, x_1']) > 0$ e $(F^k)'([x_2', 1]) = F'([x_2, 1])(F^{k-1})'([x_2', 1]) < 0$.

Portanto, é imediato concluir que $\Lambda_k$ é a união de $2^k$ intervalos fechados disjuntos e que $F^k: [a, b] \to [0, 1]$ é bijetora, onde $[a, b]$ é um desses intervalos.
\end{proof}

\begin{definition}
Seja $\Gamma \subset \RR$ um conjunto não vazio. Dizemos que $\Gamma$ é um conjunto de Cantor se as seguintes condições são válidas:
\begin{enumerate}[label=\roman*.]
\item $\Gamma$ é compacto.
\item $\Gamma$ não possui intervalos.
\item Todo ponto de $\Gamma$ é um ponto de acumulação de $\Gamma$.
\end{enumerate}
\end{definition}

\begin{lemma} \label{lema conjuntosdecantor 1}
Se $\mu > 2 + \sqrt{5}$, então
\begin{enumerate}
\item  existe $\lambda > 1$ tal que $|F'(\Lambda_1)| > \lambda$.

\item $b - a < \frac{1}{\lambda^n}$, onde $[a, b]$ é um dos intervalos que formam $\Lambda_n$.
\end{enumerate}
\end{lemma}

\begin{proof}
\begin{enumerate}\item[]
\item Se $\mu > 2 + \sqrt{5}$, então $F'(x_1) = \sqrt{\mu^2 - 4\mu} > 1$ e $F'(x_2) = -\sqrt{\mu^2 - 4\mu} < -1$.
Desse modo, podemos concluir que existe $\lambda > 1$ tal que $|F'(\Lambda_1)| > \lambda$.

\item Se $x \in [a, b]$, então $F^k(x) \in \Lambda_1$ para todo $0 \leq k < n$.
Desse modo, $(F^n)'(x) = \prod_{k=0}^{n-1} F'(F^k(x)) > \lambda^n$ e, pelo TVM, existe $c \in (a, b)$ tal que
$$1 = |F^n(b) - F^n(a)| = |(F^n)'(c)||b - a| > \lambda^n|b - a|.$$
\end{enumerate}
\end{proof} 

Sejam $x \in \Lambda$ e $\varepsilon > 0$. Pelo Lema anterior,  existe um intervalo $[a, b] \subset \Lambda_n$ para algum $n \geq 1$ tal que $x \in [a, b]$, $b - a < \varepsilon$ e $F^n: [a, b] \to [0,1]$ é bijetora.

\begin{theorem} \label{teorema conjuntosdecantor 1}
Se $\mu > 2 + \sqrt{5}$, então $\Lambda$ é um conjunto de Cantor.
\end{theorem}

\begin{proof}
Obviamente, $\Lambda$ é não vazio e compacto.

Se $\Lambda$ contém algum intervalo, então existem $x < y$ tais que $[x, y] \subset \Lambda$.
Seja $k$ tal que $\frac{1}{\lambda^k} < |x - y|$.
Em particular, $[x, y] \subset \Lambda_k$, o que é um absurdo pois os intervalos de $\Lambda_k$ possuem tamanho menor que $\frac{1}{\lambda^k}$.

Sejam $x \in \Lambda$ e $\varepsilon > 0$.
Como $\frac{1}{\lambda^k} < \varepsilon$ para algum $k \geq 0$, os intervalos que formam $\Lambda_k$ possuem tamanho menor que $\varepsilon$.
Se $x \in [a, b]$, onde $[a, b]$ é um desses intervalos, então $a \in \Lambda$ e $|x - a| < \varepsilon$.
\end{proof}
