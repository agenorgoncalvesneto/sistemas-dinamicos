\documentclass[a4paper, 12pt]{article}

\usepackage[utf8]{inputenc}
\usepackage[brazil]{babel}

%abnt
\usepackage[left=3cm, top=3cm, right=2cm, bottom=2cm]{geometry}
\usepackage[onehalfspacing]{setspace}
\usepackage{indentfirst}

%math
\usepackage{amsmath}
\usepackage{amsthm}
\usepackage{amsfonts}
\usepackage{amssymb}
\usepackage{dsfont}
\usepackage{mathtools}

\newtheorem{definition}{Definição}[section]
\newtheorem{theorem}[definition]{Teorema}
\newtheorem{proposition}[definition]{Proposição}

%shortcuts
\newcommand{\RR}{\mathbb{R}}
\newcommand{\NN}{\mathbb{N}}

\begin{document}

\section{Definições elementares}

Ao longo do texto $f^n(x)$ representará a n-ésima iterada de$f$ no ponto $x$ e os conjuntos $I$ e $J$ representarão intervalos fechados de $\RR$. 
 
\begin{definition}
Sejam $f: I \rightarrow J$ uma função, $p \in I$ e $n \geq 1$.  Dizemos que $p$ é um ponto periódico de $f$ com período $n$ se $f^n(p) = x$. Se $f^k(p) \neq x$ para todo $1 \leq k < n$, então $n$ é chamado de período principal. Em particular, se $n=1$, dizemos que $p$ é um ponto fixo de $f$.
\end{definition}

\begin{definition}
Sejam $f: I \rightarrow J$ uma função, $p \in I$ e $n \geq 1$. Dizemos que $p$ é um ponto eventualmente periódico de $f$, com período $n$, se existe $m > 1$ tal que $f^k(p) = f^{k+n}(p)$ para todo $k \geq m$. Em particular, se $n = 1$, dizemos que p é um ponto eventualmente fixo de f.
\end{definition}

\begin{definition}
Sejam $f:I \rightarrow J$ uma função e $x \in I$. A órbita de $x$ é o conjunto $O(x) = \lbrace x, f(x), f^2(x), \cdots \rbrace$.
\end{definition}

\begin{definition}
Sejam $f: I \rightarrow J$ uma função, $p$ um ponto periódico de período $n$ e $x \in I$. Dizemos que $x$ tende assintoticamente para $p$ se $lim_{k \rightarrow \infty} f^{kn}(x) = p$. O conjunto $W^s(p)$ dos pontos que tendem assintoticamente para $p$ é chamado chamado de conjunto estável de $p$.
\end{definition}

\begin{proposition}
Os conjuntos estáveis de dois pontos periódicos distintos possuem intersecção vazia.
\end{proposition}

\begin{proof}
Suponha que existam pontos periódicos distintos $p$ e $q$ de uma função $f$, de períodos $m$ e $n$ respectivamente, tais que $W^s(p) \cap W^s(q) \neq \emptyset$. Seja $x \in W^s(p) \cap W^s(q)$. Temos que $|f^{km}(x) - p| \rightarrow 0$ e $|f^{kn}(x) - q| \rightarrow 0$ quando $k \rightarrow \infty$.

Desse modo, dado $\varepsilon > 0$ existe $N \in \NN$ tal que $|f^{km}(x) - p| < \frac{\varepsilon}{2}$ e $|f^{kn}(x) - q| < \frac{\epsilon}{2}$ para todo $k > N$. Portanto, $|p - q| = |p - f^{kmn}(x) + f^{kmn}(x) - q| \leq |f^{(kn)m}(x) - p| + |f^{(km)n}(x) - q| < \frac{\varepsilon}{2} + \frac{\varepsilon}{2} = \varepsilon$. Temos então que $p = q$, pois $\varepsilon$ é arbitrário. Absurdo.
\end{proof}

O objetivo do estudo de Sistemas Dinâmicos é entender a natureza das órbitas, identificando pontos periódicos, eventualmente periódicos, que tendem assintoticamente, etc.

\section{Implicações da diferenciabilidade}
  
\begin{proposition}
\label{prop ponto fixo}
Seja $f: I \rightarrow I$ uma função contínua. Então $f$ possui ponto fixo.
\end{proposition}

\begin{proof}
Seja $I = [a, b]$ e considere a função contínua $g(x) = f(x) - x$ definida em $I$. Como $f(a), f(b) \in I$, temos que $g(a) = f(a) - a \geq 0$ e $g(b) = f(b) - b \leq 0$. Pelo Teorema do Valor Intermediário, existe $p \in I$ tal que $g(p) = f(p) -p = 0$. Desse modo, $p$ é ponto fixo de $f$.
\end{proof}

\begin{theorem}
\label{teo ponto fixo unico}
Seja $f:I \rightarrow I$ uma função diferenciável. Suponha que $|f'(x)|<1$ para todo $x \in I$. Então $|f(x) - f(y)| < |x - y|$ para todo $x, y \in I$, $x \neq y$. Além disso,  $f$ admite um único ponto fixo.
\end{theorem}

\begin{proof}
Sejam $x, y \in I$, $x < y$. Pelo Teorema do Valor Médio, existe $c \in [x, y]$ tal que $f(x) - f(y) = f'(c)(x - y)$. Portanto, $|f(x) - f(y)| = |f'(c)||x - y| < |x - y|$.

Pela Proposição \ref{prop ponto fixo}, $f$ admite um ponto fixo $p$. Suponha que exista um ponto fixo $q$ diferente de $p$. Então, pela primeira parte da demonstração, $|p - q| = |f(p) - f(q)| < |p - q|$. Absurdo.
\end{proof}

\begin{definition}
Sejam $f: I \rightarrow J$ uma função diferenciável e $p$ um ponto periódico de $f$ com período principal $n$. Dizemos que $p$ é um ponto hiperbólico se $|(f^n)'(p)| \neq 1$. Se $|(f^n)'(p)| > 1$ dizemos que $p$ é um ponto hiperbólico atrator e se $|(f^n)'(p)| < 1$ dizemos que $p$ é um ponto hiperbólico repulsor. Dizemos que $p$ é um ponto não-hiperbólico se $|(f^n)'(p)| = 1$.
\end{definition}

\begin{theorem}
Sejam $f: I \rightarrow I$ uma função $C^1$ e $p$ um ponto periódico de $f$ com período principal $n$. Se $p$ é um ponto hiperbólico atrator, existe uma vizinhança $U$ de $p$ tal que $lim_{k \rightarrow \infty} f^{kn}(x) = p$ para todo $x \in U$. Se $p$ é um ponto hiperbólico repulsor, existe uma vizinhança $V$ de $p$ tal que, se $x \in V$ e $x \neq p$, $f^{kn}(x) \notin V$ para algum $k \geq 1$. 
\end{theorem}

\begin{proof}
Suponha que $p$ é um ponto hiperbólico atrator. Como $f'$ é contínua, existe uma vizinhança $U$ de $p$ tal que $|(f^n)'(x)| \leq \lambda$ para todo $x \in U$ e para algum $\lambda < 1$. Pelo Teorema do Valor Médio, se $x \in U$ então $|f^n(x) - p| = |f^n(x) - f^n(p)| < \lambda|x - p|$. Por indução, $|f^{kn}(x) - p| < \lambda^k|x - p|$. Desse modo, $f^{kn}(x) \rightarrow p$ quando $k \rightarrow \infty$.

Suponha que $p$ é ponto hiperbólico repulsor. Como $f'$ é contínua, existe uma vizinhança $V$ de $p$ tal que $|(f^n)'(x)| \geq \lambda$ para todo $x \in V$ e para algum $\lambda > 1$. Pelo Teorema do Valor Médio, se $x \in V$ e $x \neq p$ então $|f^n(x) - p| = |f^n(x) - f^n(p)| > \lambda|x - p|$. Por indução, $|f^{kn}(x) - p| > \lambda^k|x - p|$. Como $\lambda^k|x - p| \rightarrow \infty$ quando $k \rightarrow \infty$, temos que  $f^{kn}(x) \notin V$ para algum $k \geq 1$. 
\end{proof}

\end{document}
