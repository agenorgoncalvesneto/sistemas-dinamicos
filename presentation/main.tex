\documentclass[11pt, portuguese]{beamer}

\usepackage[utf8]{inputenc}
\usepackage[brazil]{babel}

\usepackage{float}
\usepackage{graphicx}
\usepackage{multicol}
\usepackage{subfiles}

\usepackage{amsfonts}
\usepackage{amsmath}
\usepackage{amssymb}
\usepackage{amsthm}

\newcommand{\RR}{\mathbb{R}}
\newcommand{\basin}{\mathcal{B}}
\newcommand{\class}{\mathcal{C}}
\newcommand{\orbit}{\mathcal{O}}
\newcommand{\schwarz}{\mathcal{S}}

\DeclareMathOperator{\per}{Per}
\DeclareMathOperator{\sen}{sen}
\DeclareMathOperator{\tr}{Tr}

\usetheme[progressbar=frametitle]{metropolis}
\useoutertheme{metropolis}
\useinnertheme{metropolis}
\usefonttheme{metropolis}
\usecolortheme{seagull}
\setbeamercolor{background canvas}{bg=white}
\setbeamercovered{transparent}
\setbeamertemplate{footline}{}
%\setbeamersize{text margin left=0cm,text margin right=0cm} 

\AtBeginSection[]
{
  \begin{frame}
  \vspace{5pt}
  \frametitle{Sumário}
  \tableofcontents[currentsection]
  \end{frame}
}

\AtBeginSubsection[]
{
  \begin{frame}
  \vspace{5pt}
  \frametitle{Sumário}
  \tableofcontents[currentsection, currentsubsection]
  \end{frame}
}


\title{Uma Introdução aos Sistemas Dinâmicos\\Discretos}\author{Agenor Gonçalves Neto \footnote{Graduando em Bacharelado em Matemática (IME-USP), orientado pelo Prof. Salvador Addas Zanata (IME-USP).\\}}
\date{São Paulo, 2020}

\uselanguage{portuguese}
\languagepath{portuguese}
\newtheorem{proposition}[theorem]{Proposição}
\deftranslation[to=portuguese]{Theorem}{Teorema}
\deftranslation[to=portuguese]{Definition}{Definição}

\renewcommand{\raggedright}{\leftskip=5pt \rightskip=5pt plus 0cm}

\begin{document}
\metroset{block=fill}

\begin{frame}
\titlepage
\end{frame}

\begin{frame}
\vspace{5pt}
\frametitle{Sumário}
\tableofcontents
\end{frame}

\subfile{sections/conceitos-elementares.tex}
\subfile{sections/familia-quadratica.tex}
\subfile{sections/estudo-inicial.tex}
\subfile{sections/conjuntos-de-cantor.tex}
\subfile{sections/caos.tex}
\subfile{sections/conjugacao-topologica.tex}
\subfile{sections/dinamica-simbolica.tex}
\subfile{sections/matriz-de-transicao.tex}
\subfile{sections/bifurcacao.tex}
\subfile{sections/teorema-de-sharkovsky}
\subfile{sections/teorema-de-singer.tex}

\begin{frame}
\vspace{5pt}
\frametitle{Referências}
\begin{columns}
\column{\dimexpr\paperwidth-15pt}

\nocite{burns, devaney, holmgren}
\bibliographystyle{apalike-pt}
\bibliography{bibliography.bib}

\end{columns}
\end{frame}
\end{document}
