\documentclass{beamer}

\usepackage[utf8]{inputenc}
\usepackage[brazil]{babel}
\usepackage{babelbib}

\usepackage{float}
\usepackage{graphicx}
\usepackage{multicol}
\usepackage{subfiles}

\usepackage{amsfonts}
\usepackage{amsmath}
\usepackage{amssymb}
\usepackage{amsthm}

\DeclareMathOperator{\per}{Per}
\DeclareMathOperator{\sen}{sen}
\DeclareMathOperator{\tr}{Tr}

\usetheme[progressbar=frametitle]{metropolis}
\useoutertheme{metropolis}
\useinnertheme{metropolis}
\usefonttheme{metropolis}
\usecolortheme{spruce}
\setbeamercolor{background canvas}{bg=white}
\setbeamercovered{transparent}
\setbeamertemplate{footline}{}
%\setbeamersize{text margin left=0cm,text margin right=0cm} 

\AtBeginSection[]
{
  \begin{frame}[t]
  \vspace{5pt}
  \frametitle{Sumário}
  \tableofcontents[currentsection]
  \end{frame}
}

\AtBeginSubsection[]
{
  \begin{frame}[t]
  \vspace{5pt}
  \frametitle{Sumário}
  \tableofcontents[currentsection, currentsubsection]
  \end{frame}
}


\title{Uma Introdução aos Sistemas Dinâmicos Discretos}\author{Agenor Gonçalves Neto \footnote{Orientado pelo Prof. Salvador Addas Zanata (IME-USP).\\}}
\date{São Paulo, 2020}

\uselanguage{brazilian}
\languagepath{brazilian}
\deftranslation[to=brazilian]{Theorem}{Teorema}
\deftranslation[to=brazilian]{Definition}{Definição}

\begin{document}
\metroset{block=fill}

\begin{frame}
\titlepage
\end{frame}

\begin{frame}[t]
\vspace{5pt}
\frametitle{Sumário}
\tableofcontents%[pausesections, pausesubsections]
\end{frame}

\section{Conceitos Elementares}
\begin{frame}[t]
\frametitle{Conceitos Elementares}
\end{frame}

\section{Família Quadrática}
\begin{frame}[t]
\frametitle{Família Quadrática}
\end{frame}

\subsection{Estudo Inicial}
\begin{frame}[t]
\frametitle{Estudo Inicial}
\end{frame}

\subsection{Conjuntos de Cantor}
\begin{frame}[t]
\frametitle{Conjuntos de Cantor}    
\end{frame}   

\subsection{Caos}
\begin{frame}[t]
\frametitle{Caos}
\end{frame}

\subsection{Conjugação Topológica}
\begin{frame}[t]
\frametitle{Conjugação Topológica}
\end{frame}

\subsection{Dinâmica Simbólica}
\begin{frame}[t]
\frametitle{Dinâmica Simbólica}
\end{frame}

\subsection{Matriz de Transição}
\begin{frame}[t]
\frametitle{Matriz de Transição}
\end{frame}

\subsection{Bifurcação}
\begin{frame}[t]
\frametitle{Bifurcação}

\begin{columns}
  \column{\dimexpr\paperwidth-5pt}
  \begin{block}{Teorema de Sharkovsky}
Se $\per_n(f) \neq \emptyset$, então $\per_m(f) \neq \emptyset$ para todo $n \, \triangleright \, m$.
  
  \end{block}
  
  \begin{enumerate}
  \item 5
  \item 6asdf
  \end{enumerate}
\end{columns}

\end{frame}

\subfile{sections/sharkovsky}

\section{Teorema de Singer}
\begin{frame}[t]
\frametitle{Teorema de Singer}
\cite{burns}
\cite{devaney}
\cite{holmgren}
\end{frame}

\section*{Referências}
\begin{frame}[t]
\frametitle{Referências}
\nocite{burns, devaney, holmgren}
\bibliographystyle{apalike-pt}
\bibliography{bibliography.bib}
\end{frame}

\end{document}
