\subsection{Estudo Inicial}

\begin{frame}
\vspace{5pt}
\frametitle{Família Quadrática: Estudo Inicial}
\begin{columns}
\column{\dimexpr\paperwidth-15pt}

\begin{proposition}
Se $\mu > 1$, então $h(0) = 0$ e $h(p_\mu) = p_\mu$, onde $p_\mu = \frac{\mu-1}{\mu}$.
\end{proposition}

\begin{proposition}
Se $\mu > 1$, então $\lim_{k \to \infty} h^k(x) = - \infty$ para todo $x \in (-\infty, 0) \cup (1, \infty)$.
\end{proposition}

\end{columns}
\end{frame}

%--------------------

\begin{frame}
\vspace{5pt}
\frametitle{Família Quadrática: Estudo Inicial}
\begin{columns}
\column{\dimexpr\paperwidth-15pt}

\begin{proposition}
Se $1 < \mu < 3$, então
\begin{enumerate}
\item $0$ é um ponto repulsor e $p_\mu$ é um ponto atrator.
\item $\lim_{k \to \infty} h^k(x) = p_\mu$ para todo $0 < x < 1$.
\end{enumerate}
\end{proposition}

Desse modo, a dinâmica de $h$ está completamente determinada quando $1 < \mu < 3$. De fato,
$$\basin(0) = \lbrace 0, 1 \rbrace, \quad \basin(p_\mu) = (0, 1) \quad \text{ e } \quad \basin(\infty) = (-\infty, 0) \cup (1, \infty).$$

\end{columns}
\end{frame}
