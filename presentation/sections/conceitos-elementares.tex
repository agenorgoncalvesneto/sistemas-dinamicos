\section{Conceitos Elementares}

%--------------------

\begin{frame}
\frametitle{Conceitos Elementares}
\begin{columns}
\column{\dimexpr\paperwidth-15pt}

\begin{definition}
Um sistema dinâmico é função $f: X \to X$, onde $X$ é um espaço métrico.
\end{definition}
\pause
Dado $x \in X$, nosso objetivo é estudar as propriedades da sequência definida recursivamente por
$$f^0(x) = x \quad \text{ e } \quad f^k(x) = f(f^{k-1}(x))$$
para todo $k \geq 1$.

\end{columns}
\end{frame}

%--------------------

\begin{frame}
\vspace{5pt}
\frametitle{Conceitos Elementares}      
\begin{columns}
\column{\dimexpr\paperwidth-15pt}

\begin{definition}
\begin{enumerate}
\item Se $p \in X$ e $f(p) = p$, então $p$ é um ponto fixo de $f$.

\item Se $p \in X$ e $f^n(p) = p$ para algum $n \geq 1$, então $p$ é um ponto periódico de $f$ de período $n$.

\item Se $p \in X$, $f^n(p) = p$ para algum $n \geq 1$ e $f^k(p) \neq p$ para todo $1 \leq k < n$, então $p$ é um ponto periódico $f$ de período primo $n$.
\end{enumerate}
\end{definition}
\pause
O conjunto dos pontos periódicos de $f$ será denotado por $\per(f)$.

O conjunto dos pontos periódicos de $f$ de período primo $n$ será denotado por $\per_n(f)$.

\end{columns}
\end{frame}

%--------------------

\begin{frame} 
\vspace{5pt}
\frametitle{Conceitos Elementares}
\begin{columns}
\column{\dimexpr\paperwidth-15pt}

\begin{definition}
\begin{enumerate}
\item Se $x \in X$, então $\orbit(x) = \lbrace f^k(x) : k \geq 0 \rbrace$ é a órbita de $x$.

\item Se $p \in \per_n(f)$, então $$\basin(p) = \lbrace x \in  X: \lim_{k \to \infty} f^{kn}(x) = p \rbrace$$ é o conjunto estável de $p$.
\end{enumerate}
\end{definition}

\pause

Além disso, dizemos que $$\basin(\infty) = \lbrace x \in  X: \lim_{k \to \infty} |f^k(x)| = \infty \rbrace$$ é o conjunto estável do infinito.

\end{columns}
\end{frame}

%--------------------

\begin{frame}
\vspace{5pt}
\frametitle{Conceitos Elementares}
\begin{columns}
\column{\dimexpr\paperwidth-15pt}

\begin{proposition}
Seja $f: [a, b] \to \RR$ uma função contínua. Se $f([a, b]) \subset [a, b]$ ou $f([a, b]) \supset [a, b]$, então $f$ possui ponto fixo.
\end{proposition}

\begin{theorem}
Sejam $f: \RR \to \RR$ uma função de classe $\class^1$ e $p \in \per_n(f)$.
\begin{enumerate}
\item Se $|D f^n(p)| < 1$, então existe uma vizinhança de $p$ contida em $\basin(p)$.
\item Se $|D f^n(p)| > 1$, então existe uma vizinhança $V$ de $p$ com a seguinte propriedade: se $x \in V \backslash \lbrace p \rbrace$, então  $f^{kn}(x) \notin V$ para algum $k \geq 1$. 
\end{enumerate}
\end{theorem}

\end{columns}
\end{frame}

%--------------------

\begin{frame}
\vspace{5pt}
\frametitle{Conceitos Elementares}
\begin{columns}
\column{\dimexpr\paperwidth-15pt}

\begin{definition}
Sejam $f: \RR \to \RR$ uma função de classe $\class^1$ e $p \in \per_n(f)$.
\begin{enumerate}
\item Se $|D f^n(p)| < 1$, então $p$ é um ponto atrator.
\item Se $|D f^n(p)| > 1$, então $p$ é um ponto repulsor.
\end{enumerate}
\end{definition}
\pause
A definição anterior pode ser estendida para órbitas de pontos periódicos. De fato, se um ponto é atrator, então todos os pontos de sua órbita também são atratores e, nesse caso, dizemos que a órbita é atratora.

\end{columns}
\end{frame}

%--------------------
