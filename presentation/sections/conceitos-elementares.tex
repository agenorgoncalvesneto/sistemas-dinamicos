\section{Conceitos Elementares}

%--------------------

\begin{frame}
\frametitle{Conceitos Elementares}
\begin{columns}
\column{\dimexpr\paperwidth-15pt}

\begin{definition}
Um sistema dinâmico é função $f: X \to X$, onde $X$ é um espaço métrico.
\end{definition}

Dado $x \in X$, queremos estudar as propriedades da sequência
$$f^0(x) = x, \,\,\, f^1(x) = f(x), \,\,\, f^2(x) = f(f(x)), \,\,\, f^3(x) = f(f(f(x))), \,\,\, \dots.$$

\end{columns}
\end{frame}

%--------------------

\begin{frame}
\vspace{5pt}
\frametitle{Conceitos Elementares}      
\begin{columns}
\column{\dimexpr\paperwidth-15pt}

\begin{definition}
Se $x \in X$, então $\lbrace f^k(x) : k \geq 0 \rbrace$ é a órbita de $x$.
\end{definition}

\begin{definition}
Seja $p \in X$.
\begin{enumerate}[i.]
\item Se $f(p) = p$, então $p$ é um ponto fixo de $f$.

\item Se $f^n(p) = p$ para algum $n \geq 1$, então $p$ é um ponto periódico de $f$ de período $n$.

\item Se $f^n(p) = p$ para algum $n \geq 1$ e $f^k(p) \neq p$ para todo $1 \leq k < n$, então $p$ é um ponto periódico $f$ de período primo $n$.
\end{enumerate}
\end{definition}

O conjunto dos pontos periódicos de $f$ será denotado por $\per(f)$ e o conjunto dos pontos periódicos de $f$ de período primo $n$ será denotado por $\per_n(f)$.

\end{columns}
\end{frame}

%--------------------

\begin{frame} 
\vspace{5pt}
\frametitle{Conceitos Elementares}
\begin{columns}
\column{\dimexpr\paperwidth-15pt}

\begin{proposition}
Seja $f: [a, b] \to \RR$ uma função contínua. Se $f([a, b]) \subset [a, b]$ ou $f([a, b]) \supset [a, b]$, então $f$ possui ponto fixo.
\end{proposition}

\vspace{10pt}

\begin{definition} 
Se $p \in \per_n(f)$, então
$$\basin(p) = \lbrace x \in  X: \lim_{k \to \infty} f^{kn}(x) = p \rbrace$$
é a bacia de atração de $p$. Além disso,
$$\basin(\infty) = \lbrace x \in  X: \lim_{k \to \infty} |f^k(x)| = \infty \rbrace$$
é a bacia de atração do infinito.
\end{definition}

\end{columns}
\end{frame}

%--------------------

\begin{frame}
\vspace{5pt}
\frametitle{Conceitos Elementares}
\begin{columns}
\column{\dimexpr\paperwidth-15pt}

\begin{definition}
Sejam $f: \RR \to \RR$ uma função de classe $\class^1$ e $p \in \per_n(f)$.
\begin{enumerate}[i.]
\item Se $|D f^n(p)| < 1$, então $p$ é um ponto atrator.
\item Se $|D f^n(p)| > 1$, então $p$ é um ponto repulsor.
\end{enumerate}
\end{definition}

\vspace{10pt}

\begin{theorem}
Sejam $f: \RR \to \RR$ uma função de classe $\class^1$ e $p \in \per_n(f)$.
\begin{enumerate}
\item Se $|D f^n(p)| < 1$, então existe uma vizinhança de $p$ contida na bacia de atração de $p$.
\item Se $|D f^n(p)| > 1$, então existe uma vizinhança de $p$ tal que as órbitas de seus pontos que são diferentes de $p$ não estão contidas nela própria.
\end{enumerate}
\end{theorem}

\end{columns}
\end{frame}

%--------------------
