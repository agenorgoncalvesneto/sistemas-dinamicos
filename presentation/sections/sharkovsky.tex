\section{Teorema de Sharkovsky}

\begin{frame}[t]
\vspace{5pt}
\frametitle{Teorema de Sharkovsky}
\begin{columns}
\column{\dimexpr\paperwidth-5pt}

\begin{definition}{Ordenação de Sharkovsky}
$3 \, \triangleright \, 5 
\, \triangleright \, \cdots \, \triangleright \,
2 \cdot 3 \, \triangleright \, 2 \cdot 5 
\, \triangleright \, \cdots \, \triangleright \,
2^2 \cdot 3 \, \triangleright \, 2^2 \cdot 5
\, \triangleright \, \cdots \, \triangleright \,
2^k \cdot 3 \, \triangleright \, 2^k \cdot 5
\, \triangleright \, \cdots \, \triangleright \,
2^2 \, \triangleright \, 2 \, \triangleright \, 1$
\end{definition}

\pause

\begin{block}{Teorema de Sharkovsky}
Se $\per_n(f) \neq \emptyset$, então $\per_m(f) \neq \emptyset$ para todo $n \, \triangleright \, m$.
\end{block}

\end{columns}
\end{frame}


\begin{frame}[t]
\vspace{5pt}
\frametitle{Teorema de Sharkovsky}
\begin{columns}
\column{\dimexpr\paperwidth-5pt}

\begin{theorem}
Se $n \geq 1$, então existe uma função $f$ com as seguintes propriedades:
\begin{enumerate}
\item $\per_n(f) \neq \emptyset$.
\item $\per_m(f) =  \emptyset$ para todo $m \, \triangleright \, n$.
\end{enumerate}
\end{theorem}

\end{columns}
\end{frame}

\begin{frame}[t]
\vspace{5pt}
\frametitle{asdf}
\begin{columns}
\column{\dimexpr\paperwidth-5pt}
\begin{theorem}
asdfasd
\end{theorem}
\end{columns}
\end{frame}

