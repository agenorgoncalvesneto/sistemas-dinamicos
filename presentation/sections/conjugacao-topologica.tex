\subsection{Conjugação Topológica}

%--------------------

\begin{frame}[fragile]
\vspace{5pt}
\frametitle{Família Quadrática: Conjugação Topológica}
\begin{columns}
\column{\dimexpr\paperwidth-15pt}

\begin{definition}
Sejam $f: X \to X$, $g: Y \to Y$ e $\tau: X \to Y$ funções. Dizemos que $f$ e $g$ são topologicamente conjugadas por $\tau$ se as seguintes condições são válidas:
\begin{enumerate}[i.]
\item $\tau$ é um homeomorfismo.
\item $\tau \circ f = g \circ \tau$.
\end{enumerate}
\end{definition}

\vspace{10pt}

\begin{adjustbox}{center}
\begin{tikzpicture}
  \matrix (m) [matrix of math nodes, row sep=3em, column sep=3em]
    { x \in X  & f(x) \in X \\
      \tau(x) \in Y & g(\tau(x)) = \tau(f(x)) \in Y \\ };
      
  { [start chain]
    \chainin (m-1-1);
    { [start branch=A] \chainin (m-2-1)
        [join={node[right,labeled] {\tau}}];}
        
    \chainin (m-1-2) [join={node[above,labeled] {f}}];
    { [start branch=B] \chainin (m-2-2)
        [join={node[right,labeled] {\tau}}];} }
    
  { [start chain] \chainin (m-2-1);
    \chainin (m-2-2) [join={node[above,labeled] {g}}]; }
\end{tikzpicture}
\end{adjustbox}

\end{columns}
\end{frame}

%--------------------

\begin{frame}
\vspace{5pt}
\frametitle{Família Quadrática: \subsecname}
\begin{columns}
\column{\dimexpr\paperwidth-15pt}

\begin{proposition}
Sejam $f: X \to X$, $g: Y \to Y$ e $\tau: X \to Y$ funções. Se $f$ e $g$ são topologicamente conjugadas por $\tau$, então
\begin{enumerate}
\item $\per(f)$ é denso em $X$ se, e somente se, $\per(g)$ é denso em $Y$.
\item $f$ é topologicamente transitiva se, e somente se, $g$ é topologicamente transitiva.
\end{enumerate}
\end{proposition}

\end{columns}
\end{frame}

%--------------------

\begin{frame}
\vspace{5pt}
\frametitle{Família Quadrática: \subsecname}
\begin{columns}
\column{\dimexpr\paperwidth-15pt}

\begin{lemma}
A função $T: [0,1] \to [0,1]$ dada por
\[ T(x) =
  \begin{cases}
    2x, & x \in \left[ 0, \frac{1}{2} \right] \\
    2 - 2x, & x \in \left[ \frac{1}{2}, 1 \right] \\
  \end{cases}
\]
é caótica.
\end{lemma}

\end{columns}
\end{frame}

%--------------------

\begin{frame}
\vspace{5pt}
\frametitle{Família Quadrática: \subsecname}
\begin{columns}
\column{\dimexpr\paperwidth-15pt}

\begin{theorem}
Se $\mu = 4$, então $h$ é caótica.
\end{theorem}

\begin{proof}
Basta observar que $\tau \circ T = h \circ \tau$, onde $\tau: [0, 1] \to [0, 1]$ é o homeomorfismo dado por $\tau(x) = \sen^2(\frac{\pi x}{2})$.
\end{proof}

\end{columns}
\end{frame}

%--------------------
