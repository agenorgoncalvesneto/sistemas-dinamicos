\subsection{Caos}

\begin{frame}
\vspace{5pt}
\frametitle{Família Quadrática: \subsecname}
\begin{columns}
\column{\dimexpr\paperwidth-15pt}

\begin{definition}
Seja $f: X \to X$ uma função. Dizemos que $f$ é topologicamente transitiva se dados $x, y \in X$ e $\varepsilon > 0$,  existem $z \in X$ e $k \geq 0$ tais que $|x - z| < \varepsilon$ e $|y - f^k(z)| < \varepsilon$.
\end{definition} 

\begin{proposition}
Se $\mu > 2 + \sqrt{5}$, então $h|_\Lambda$ é topologicamente transitiva.
\end{proposition}

\end{columns}
\end{frame}

%--------------------

\begin{frame}
\vspace{5pt}
\frametitle{Família Quadrática: \subsecname}
\begin{columns}
\column{\dimexpr\paperwidth-15pt}

\begin{definition}
Seja $f: X \to X$ uma função. Dizemos que $f$ depende sensivelmente das condições iniciais se existe $\delta > 0$ com a seguinte propriedade: dados $x \in X$ e $\varepsilon > 0$, existem $y \in X$ e $k \geq 0$ tais que $|x - y| < \varepsilon$ e $|f^k(x) - f^k(y)| > \delta$.
\end{definition}

\begin{proposition}
Se $\mu > 2 + \sqrt{5}$, então $h|_\Lambda$ depende sensivelmente das condições iniciais.
\end{proposition}

\end{columns}
\end{frame}

%--------------------

\begin{frame}
\vspace{5pt}
\frametitle{Família Quadrática: \subsecname}
\begin{columns}
\column{\dimexpr\paperwidth-15pt}

\begin{definition}
Seja $f: X \to X$ uma função. Dizemos que $f$ é caótica se as seguintes condições são válidas:
\begin{enumerate}[i.]
\item $\per(f)$ é denso em $X$.
\item $f$ é topologicamente transitiva.
\item $f$ depende sensivelmente das condições iniciais.
\end{enumerate}
\end{definition}

\end{columns}
\end{frame}

%--------------------

\begin{frame}
\vspace{5pt}
\frametitle{Família Quadrática: \subsecname}
\begin{columns}
\column{\dimexpr\paperwidth-15pt}

\begin{theorem}
Se $\mu > 2 + \sqrt{5}$, então $h|_\Lambda$ é caótica.
\end{theorem}

\begin{block}{Observação}
Esse teorema é válido para $4 < \mu < 2 + \sqrt{5}$, porém a demonstração é mais complicada.
\end{block}

\end{columns}
\end{frame}

%--------------------

\begin{frame}
\vspace{5pt}
\frametitle{Família Quadrática: \subsecname}
\begin{columns}
\column{\dimexpr\paperwidth-15pt}

\begin{theorem}
Seja $f: X \to X$ é uma função contínua, onde $X$ é um conjunto infinito. Se $\per(f)$ é denso em $X$ e $f$ é topologicamente transitiva, então $f$ é caótica.
\end{theorem}

\begin{proof}
Ver \cite{holmgren}.
\end{proof}

\end{columns}
\end{frame}
