\subsection{Bifurcação}

%--------------------

\begin{frame}
\vspace{5pt}
\frametitle{Família Quadrática: \subsecname}
\begin{columns}
\column{\dimexpr\paperwidth-15pt}

\begin{definition}
Seja $f_\lambda$ uma família parametrizada de funções no parâmetro $\lambda$. Dizemos que essa família sofre uma bifurcação em $\lambda_0$ se existe $\varepsilon > 0$ com a seguinte propriedade:\\se $\lambda_1 \in (\lambda_0 - \varepsilon, \lambda_0)$ e $\lambda_2 \in (\lambda_0, \lambda_0 + \varepsilon)$, então $f_{\lambda_1}$ e $f_{\lambda_2}$ não são topologicamente conjugadas.
\end{definition}

\end{columns}
\end{frame}

%--------------------

\begin{frame}
\vspace{5pt}
\frametitle{Família Quadrática: \subsecname}
\begin{columns}
\column{\dimexpr\paperwidth-15pt}

\begin{block}{Exemplo}
A família $E_\lambda$ de funções dadas por $E_\lambda(x) = e^{x + \lambda}$ sofre uma bifurcação em $\lambda_0 = -1$.

\vspace{10pt}

\begin{figure}[!htb]
\centering
\includegraphics[scale=0.4]{images/e_lambda.png}
\caption{Gráficos de $E_\lambda$ numa vizinhança de $1$ para $\lambda = -1.1$, $\lambda = -1$ e $\lambda = -0.9$.}
\end{figure}
Uma bifurcação com essas características é chamada de bifurcação tangente.
\end{block}

\end{columns}
\end{frame}

%--------------------

\begin{frame}
\vspace{5pt}
\frametitle{Família Quadrática: \subsecname}
\begin{columns}
\column{\dimexpr\paperwidth-15pt}

\begin{block}{Exemplo}
A família quadrática sofre uma bifurcação em $\mu_0 = 3$.

\vspace{10pt}

\begin{figure}[!htb]
\centering
\includegraphics[scale=0.4]{images/h_3^2.png}
\caption{Gráficos de $h^2$ numa vizinhança de $p_\mu$ para $\mu = 2.9$, $\mu = 3$ e $\mu = 3.1$.}
\end{figure}
Uma bifurcação com essas características é chamada de bifurcação com duplicação de período.
\end{block}

\end{columns}
\end{frame}
%
%%--------------------
%
%\begin{frame}
%\vspace{5pt}
%\frametitle{Família Quadrática: \subsecname}
%\begin{columns}
%\column{\dimexpr\paperwidth-15pt}
%
%Vimos que a dinâmica de $h$ é simples para $1 < \mu < 3$ e caótica para $\mu \geq 4$. O que acontece com a dinâmica quando $3 < \mu < 4$?
%
%Temos, na figura abaixo, os gráficos de $h^2$ para alguns valores de $\mu$.
%
%\begin{figure}[!htb]
%\centering
%\includegraphics[scale=0.375]{images/h^2withboxes.png}
%\caption{Gráficos de $h^2$ para $\mu = 2.75$, $\mu = 3$, $\mu = 3.25$, $\mu = 3.5$ e $\mu = 3.75$.}
%\label{h^2-and-boxes}
%\end{figure}
%
%\end{columns}
%\end{frame}
%
%%--------------------
%
%\begin{frame}
%\vspace{5pt}
%\frametitle{Família Quadrática: \subsecname}
%\begin{columns}
%\column{\dimexpr\paperwidth-15pt}
%
%Observando o gráfico de $h^2$ restrito ao quadrado, esperamos que ela sofra uma bifurcação com duplicação de período conforme o parâmetro cresce. Repetindo esse processo, vemos uma sucessão de bifurcações com duplicação de período na família quadrática. Com auxílio de um computador, podemos verificar esse fato.
%
%\begin{figure}[!htb]
%\centering
%\includegraphics[scale=0.4]{images/period-doubling-and-zoom1.png}
%\caption{Diagrama de órbita de $h$ para $2 < \mu < 4$.}
%\end{figure}
%
%\end{columns}
%\end{frame}
%
%%--------------------
%
%\begin{frame}
%\vspace{5pt}
%\frametitle{Família Quadrática: \subsecname}
%\begin{columns}
%\column{\dimexpr\paperwidth-15pt}
%
%\begin{figure}[H]
%\centering
%\includegraphics[scale=0.4]{images/period-doubling-and-zoom2.png}
%\caption{Ampliação das regiões marcadas na figura anterior.}
%\end{figure}
%
%\end{columns}
%\end{frame}
