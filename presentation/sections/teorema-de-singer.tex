\section{Teorema de Singer}

%--------------------

\begin{frame}
\vspace{5pt}
\frametitle{Teorema de Singer}
\begin{columns}
\column{\dimexpr\paperwidth-15pt}

\begin{definition}[Derivada de Schwarz]
Seja $f: \RR \to \RR$ uma função de classe $\class^3$. A derivada de Schwarz de $f$ é a função $\schwarz f: \RR \backslash C_f \to \RR$ dada por
$$\schwarz f(x) = \frac{D^3 f(x)}{D f(x)} - \frac{3}{2} \left( \frac{D^2 f(x)}{D f(x)} \right)^2,$$
onde $C_f = \lbrace x \in \RR: Df(x) = 0 \rbrace$.
\end{definition}

\vspace{10pt}

Estamos interessados em funções que possuem a derivada de Schwarz negativa.
\begin{block}{Exemplo}
Se $\mu > 1$, então $Sh(x) = -6(1 - 2x)^{-2} < 0$ para todo $x \neq \frac{1}{2}$.
\end{block}
\end{columns}
\end{frame}

%--------------------

\begin{frame}
\vspace{5pt}
\frametitle{Teorema de Singer}
\begin{columns}
\column{\dimexpr\paperwidth-15pt}


\begin{theorem}[Singer]
Se $\schwarz f < 0$ e $f$ possui $n$ pontos críticos, então $f$ possui no máximo $n+2$ órbitas periódicas atratoras.
\end{theorem}

\vspace{10pt}

\begin{corollary}
Se $\mu > 1$, então $h$ possui no máximo $1$ órbita periódica atratora.
\end{corollary}

\end{columns}
\end{frame}
