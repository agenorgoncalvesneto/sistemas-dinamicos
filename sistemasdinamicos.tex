\documentclass[a4paper, 12pt]{article}

\usepackage[utf8]{inputenc}
\usepackage[brazil]{babel}

\usepackage{enumitem}

%abnt
\usepackage[left=3cm, top=3cm, right=2cm, bottom=2cm]{geometry}
\usepackage[onehalfspacing]{setspace}
\usepackage{indentfirst}

%math
\usepackage{amsmath}
\usepackage{amsthm}
\usepackage{amsfonts}
\usepackage{amssymb}
\usepackage{dsfont}
\usepackage{mathtools}

\theoremstyle{definition}
\newtheorem{definition}{Definição}[section]
\theoremstyle{plain}
\newtheorem{proposition}[definition]{Proposição}
\theoremstyle{plain}
\newtheorem{lemma}[definition]{Lema}
\theoremstyle{plain}
\newtheorem{theorem}[definition]{Teorema}
\theoremstyle{plain}
\newtheorem{corollary}[definition]{Corolário}
\theoremstyle{definition}
\newtheorem{example}[definition]{Exemplo}
\theoremstyle{plain}
\newtheorem*{affirmation}{Afirmação}
\theoremstyle{remark}
\newtheorem*{remark}{Observação}

%shortcuts
\newcommand{\RR}{\mathbb{R}}
\newcommand{\NN}{\mathbb{N}}
\newcommand{\CC}{\mathcal{C}}
\newcommand{\OO}{\mathcal{O}}
\newcommand{\SSS}{\mathcal{S}}
\newcommand{\UU}{\mathcal{U}}
\newcommand{\limit}{\displaystyle\lim}
%\newcommand{\l(}{\left(}
%\newcommand{\r)}{\right)}
\DeclareMathOperator{\sen}{sen}
%\renewcommand{\qedsymbol}{blacksquare}

\usepackage{subfiles}

\begin{document}

%\subfile{sections/definicoeselementares}
%\newpage
%\subfile{sections/diferenciabilidade}
%\newpage
%\subfile{sections/estudoinicial}
%\newpage
%\subfile{sections/conjuntosdecantor}
%\newpage
%\subfile{sections/caos}
%\newpage
%\subfile{sections/conjugacaotopologica}
%\newpage
%\subfile{sections/dinamicasimbolica}
%\newpage
%\subfile{sections/sharkovsky}
%\newpage
%\subfile{sections/derivadadeschwarz}
%\newpage
%\subfile{sections/bifurcacao}
%\newpage
\subfile{sections/estabilidadeestrutural}

\end{document}

% --- substituir o intervalo I,J por A,B no primeiro parágrafo
% --- nome das definções em itálico
% --- remover \mu de F_\mu
% --- arrrumar a indução de k para k-1 da proposição ~4.2
% --- lema 4.3 provar que (F^n)'(c) > 1: mostrar que F^K(c) \in A_1 para todo k = 0, .., n-1 
% --- mostrar em algum lugar que \Lambda_n contém \Lambda_{n+1}
% --- provar em algum lugar que Per(F) é denso quando mu > 4
% --- trocar alpha e beta por x_1 e x_2
% --- tirar itálico da funçao seno (algum comentário na aula do lcmaquino)
% --- terminar a demonstração do lemma 4 1
% --- escolher entre 'transitiva topologicamente' e 'topologicamente transitiva'(essa é melhor)
% --- utilizar \textrm{} quando misturar ambiente matemático e texto
% --- utilizar \left( e \right) para deixar o parêntese do tamanho da expressão de dento
% --- substituir lim por \lim
% --- usar \to ao invés de \rightarrow
% --- mudar equação quadrática para função logística
% --- separar conjuntos de cantor e caos
% utilizar o terceiro item do lemma 4 1 na demonstração das outras proposições
% --- separar os capítulos em arquivos
% --- atualizar a numeração das referências no tipo "caos proposition 2"
% adicionar \orbit(x) com O em letra cursiva
% colocar \newpage dentro de cada seção
% trocar 'conjunto estável' por 'base de atração' e utilizar \mathcal{B} 