\documentclass[a4paper, 12pt]{article}

\usepackage[utf8]{inputenc}
\usepackage[brazil]{babel}

%abnt
\usepackage[left=3cm, top=3cm, right=2cm, bottom=2cm]{geometry}
\usepackage[onehalfspacing]{setspace}
\usepackage{indentfirst}

%math
\usepackage{amsmath}
\usepackage{amsthm}
\usepackage{amsfonts}
\usepackage{amssymb}
\usepackage{dsfont}
\usepackage{mathtools}

\theoremstyle{definition}
\newtheorem{definition}{Definição}[section]
\theoremstyle{plain}
\newtheorem{proposition}[definition]{Proposição}
\theoremstyle{plain}
\newtheorem{lemma}[definition]{Lema}
\theoremstyle{plain}
\newtheorem{theorem}[definition]{Teorema}
\theoremstyle{remark}
\newtheorem*{remark}{Observação}

%shortcuts
\newcommand{\RR}{\mathbb{R}}
\newcommand{\NN}{\mathbb{N}}

\begin{document}

\section{Definições elementares}

Ao longo do texto, $f$ representará uma função $f: I \rightarrow J$, onde $I$ e $J$ são subconjuntos de $\RR$. Além disso, a função $f^n$ representará a n-ésima iterada de $f$ e $f^{(n)}$ representará n-ésima derivada de $f$.
 
\begin{definition}
Sejam $f: I \rightarrow J$ uma função, $p \in I$ e $n \geq 1$.  Dizemos que $p$ é um ponto periódico de $f$ com período $n$ se $f^n(p) = x$. Se $f^k(p) \neq x$ para todo $1 \leq k < n$, então $n$ é chamado de período principal. Em particular, se $n=1$, dizemos que $p$ é um ponto fixo de $f$.
\end{definition}

\begin{definition}
Sejam $f: I \rightarrow J$ uma função, $p \in I$ e $n \geq 1$. Dizemos que $p$ é um ponto eventualmente periódico de $f$, com período $n$, se existe $m > 1$ tal que $f^k(p) = f^{k+n}(p)$ para todo $k \geq m$. Em particular, se $n = 1$, dizemos que p é um ponto eventualmente fixo de $f$.
\end{definition}

\begin{definition}
Sejam $f:I \rightarrow J$ uma função e $x \in I$. A órbita de $x$ é o conjunto $O(x) = \lbrace x, f(x), f^2(x), \cdots \rbrace$.
\end{definition}

\begin{definition}
Sejam $f: I \rightarrow J$ uma função, $p$ um ponto periódico de período $n$ e $x \in I$. Dizemos que $x$ tende assintoticamente para $p$ se $lim_{k \rightarrow \infty} f^{kn}(x) = p$. O conjunto dos pontos que tendem assintoticamente para $p$, denotado por $W^s(p)$, é chamado chamado de conjunto estável de $p$. Dizemos que $x$ tende assintoticamente para infinito se $lim_{k \rightarrow \infty} |f^{k}(x)| = \infty$. O conjunto dos pontos que tendem assintoticamente para infinito, denotado por $W^s(\infty)$, é chamado de conjunto estável do infinito.
\end{definition}

\begin{proposition}
Os conjuntos estáveis de dois pontos periódicos distintos possuem intersecção vazia.
\end{proposition}

\begin{proof}
Suponha que existam pontos periódicos distintos $p$ e $q$ de uma função $f$, de períodos $m$ e $n$ respectivamente, tais que $W^s(p) \cap W^s(q) \neq \emptyset$. Seja $x \in W^s(p) \cap W^s(q)$. Temos que $|f^{km}(x) - p| \rightarrow 0$ e $|f^{kn}(x) - q| \rightarrow 0$ quando $k \rightarrow \infty$.

Desse modo, dado $\varepsilon > 0$ existe $N \in \NN$ tal que $|f^{km}(x) - p| < \frac{\varepsilon}{2}$ e $|f^{kn}(x) - q| < \frac{\epsilon}{2}$ para todo $k > N$. Portanto, $|p - q| = |p - f^{kmn}(x) + f^{kmn}(x) - q| \leq |f^{(kn)m}(x) - p| + |f^{(km)n}(x) - q| < \frac{\varepsilon}{2} + \frac{\varepsilon}{2} = \varepsilon$. Temos então que $p = q$, pois $\varepsilon$ é arbitrário. Absurdo.
\end{proof}

O objetivo do estudo de Sistemas Dinâmicos é entender a natureza das órbitas, identificando pontos periódicos, eventualmente periódicos, que tendem assintoticamente, etc.

\section{Implicações da diferenciabilidade}

Ao longo desse seção, $I$ representará um intervalo fechado de $\RR$.  

\begin{proposition}
\label{prop ponto fixo}
Seja $f: I \rightarrow I$ uma função contínua. Então $f$ possui ponto fixo.
\end{proposition}

\begin{proof}
Seja $I = [a, b]$ e considere a função contínua $g(x) = f(x) - x$ definida em $I$. Como $f(a), f(b) \in I$, temos que $g(a) = f(a) - a \geq 0$ e $g(b) = f(b) - b \leq 0$. Pelo Teorema do Valor Intermediário, existe $p \in I$ tal que $g(p) = f(p) -p = 0$. Desse modo, $p$ é ponto fixo de $f$.
\end{proof}

\begin{theorem}
\label{teo ponto fixo unico}
Seja $f:I \rightarrow I$ uma função diferenciável. Suponha que $|f'(x)|<1$ para todo $x \in I$. Então $|f(x) - f(y)| < |x - y|$ para todo $x, y \in I$, $x \neq y$. Além disso,  $f$ admite um único ponto fixo.
\end{theorem}

\begin{proof}
Sejam $x, y \in I$, $x < y$. Pelo Teorema do Valor Médio, existe $c \in [x, y]$ tal que $f(x) - f(y) = f'(c)(x - y)$. Portanto, $|f(x) - f(y)| = |f'(c)||x - y| < |x - y|$.

Pela Proposição \ref{prop ponto fixo}, $f$ admite um ponto fixo $p$. Suponha que exista um ponto fixo $q$ diferente de $p$. Então, pela primeira parte da demonstração, $|p - q| = |f(p) - f(q)| < |p - q|$. Absurdo.
\end{proof}

\begin{definition}
Sejam $f: I \rightarrow J$ uma função diferenciável e $p$ um ponto periódico de $f$ com período principal $n$. Dizemos que $p$ é um ponto hiperbólico se $|(f^n)'(p)| \neq 1$. Se $|(f^n)'(p)| > 1$ dizemos que $p$ é um ponto hiperbólico atrator e se $|(f^n)'(p)| < 1$ dizemos que $p$ é um ponto hiperbólico repulsor. Dizemos que $p$ é um ponto não hiperbólico se $|(f^n)'(p)| = 1$.
\end{definition}

\begin{theorem}
Sejam $f: I \rightarrow I$ uma função $C^1$ e $p$ um ponto periódico de $f$ com período principal $n$. Se $p$ é um ponto hiperbólico atrator, existe uma vizinhança $U$ de $p$ tal que $lim_{k \rightarrow \infty} f^{kn}(x) = p$ para todo $x \in U$. Se $p$ é um ponto hiperbólico repulsor, existe uma vizinhança $V$ de $p$ tal que, se $x \in V$ e $x \neq p$, $f^{kn}(x) \notin V$ para algum $k \geq 1$. 
\end{theorem}

\begin{proof}
Suponha que $p$ é um ponto hiperbólico atrator. Como $f'$ é contínua, existe $\varepsilon > 0$ tal que $|(f^n)'(x)| \leq \lambda < 1$ para todo $x \in (p - \varepsilon, p + \varepsilon)$. Pelo Teorema do Valor Médio, se $x \in U$ então $|f^n(x) - p| = |f^n(x) - f^n(p)| \leq \lambda|x - p|$. Por indução, $|f^{kn}(x) - p| \leq \lambda^k|x - p|$. Desse modo, $f^{kn}(x) \rightarrow p$ quando $k \rightarrow \infty$.

Suponha que $p$ é ponto hiperbólico repulsor. De maneira análoga, existe $\varepsilon > 0$ tal que  $|f^{kn}(x) - p| \geq \lambda^k|x - p| > 1$ para todo $x \in (p - \varepsilon, p + \varepsilon)$, $x \neq p$. Como $\lambda^k|x - p| \rightarrow \infty$ quando $k \rightarrow \infty$, temos que  $f^{kn}(x) \notin V$ para algum $k \geq 1$. 
\end{proof}

\section{A família quadrática}

Nosso objetivo será estudar, durante essa seção e as próximas, a dinâmica da família de funções $F_{\mu}: \RR \rightarrow \RR$ dadas por $F_{\mu}(x) = \mu x(1-x)$, onde $\mu > 0$. Tal família é chamada de família quadrática.

Ao longo dessa e das próximas seções, a menos que dito explicitamente o contrário, $I$ denotará o intervalo $[0, 1]$ da reta real.

\begin{proposition}
\begin{enumerate}
\item $F_\mu (0) = 0$ e $F_\mu(p_\mu) = p_\mu$, onde $p_\mu = \frac{\mu - 1}{\mu}$.
\item $F_\mu (1) = 0$ e $F_\mu(\frac{1}{\mu}) = p_\mu$.
\item Se $\mu > 1$ então $0 < p_\mu < 1$.
\item O vértice da parábola é o ponto $(\frac{1}{2}, \frac{\mu}{4})$.
\end{enumerate}
\end{proposition}

\begin{proof}
Trivial.
\end{proof}

No resto dessa seção estudaremos o caso onde $1 < \mu < 3$.

\begin{proposition}
\label{conjunto estavel infinito}
Suponha que $\mu > 1$. Então $W^s(\infty) = (-\infty, 0) \cup (1, \infty)$.
\end{proposition}

\begin{proof}
Se $x < 0$, a sequência  $(F^n _\mu (x))_n$ é monótona decrescente pois  $F_\mu (x) < x$. Se $(F^n_{\mu}(x))_n \rightarrow x_0$, para algum $x_0 < 0$, a continuidade de $F_\mu$ implica que $(F^{n+1}_{\mu}(x))_n \rightarrow F_\mu (x_0) < x_0$. Absurdo. Portanto, $(F^n_{\mu}(x))_n \rightarrow -\infty$.

Como $F_\mu (x) < 0$ para todo $x > 1$, concluímos que $W^s(\infty) = (-\infty, 0) \cup (1, \infty)$.
\end{proof}

A proposição anterior nos diz que resta estudar a dinâmica de $F_\mu$ restrita ao intervalo $I = [0, 1]$.

\begin{proposition}
Suponha que $1 < \mu < 3$.
\begin{enumerate}
\item $0$ é um ponto repulsor e $p_\mu$ é um ponto atrator.
\item $lim_{n \rightarrow \infty} F^n_{\mu}(x) = p_\mu$ para todo $x \in (0, 1)$.
\end{enumerate}
\end{proposition}

\begin{proof}
Como $F_\mu '(x) = \mu - 2\mu x$ e $1 < \mu < 3$, a primeira parte é verdadeira pois $|F'_\mu(0)| = \mu > 1$ e $|F'_\mu (p_\mu)| = |2 - \mu| < 1$.

Falta provar o item 2.
%Para mostrar a segunda parte, dividiremos em casos. Suponha inicialmente que $1 < \mu < 2$. Como $F_\mu(x) \leq \frac{\mu}{4} < 1/2$ para todo $x \in (0, 1)$, basta analisar o limite quando $x \in (0, \frac{1}{2})$. Se $x \in (0, \frac{1}{2})$ então, analisando a função $F_\mu(x) - x$, temos que $|F_\mu(x) - p_\mu| < |x - p_\mu|$ e, pelo continuidade de $F_\mu$, $lim_n F^n_\mu(x) = p_\mu$.
\end{proof}

Portanto, conhecemos completamente a dinâmica de $F_\mu$ quando $1 < \mu < 3$. Temos $W^s(0) = \{0, 1\}$, $W^s(p_\mu) = (0, 1)$ e $W^s(\infty) = (-\infty, 0) \cup (1, \infty)$.

\section{Conjuntos de Cantor e Caos}

Analisaremos nessa seção a dinâmica de $F_\mu$ quando $\mu > 4$ e, para entendê-la, será preciso estudar o que é um conjunto de Cantor.

\begin{definition}[Conjunto de Cantor]
Um subconjunto $\Gamma$ de $\RR$, $\Gamma \neq \emptyset$, é um conjunto de Cantor se
\begin{enumerate}
\item $\Gamma$ é fechado e limitado.
\item $\Gamma$ não possui intervalos.
\item Todo ponto de $\Gamma$ é um ponto de acumulação de $\Gamma$.
\end{enumerate}
\end{definition}

Observe inicialmente que $F_\mu(\frac{1}{2})$ = $\frac{\mu}{4} > 1$ quando $\mu > 4$, ou seja, existem pontos em $I$ que não permanecem em $I$ após uma iteração de $F_\mu$. Em vista da Proposição \ref{conjunto estavel infinito}, a dinâmica de $F_\mu$ em tais pontos é determinada, pois pertencem ao conjunto $W^s(\infty)$. De modo mais geral, se um ponto de $I$ não permanece $I$ após um número finito de iterações, então ele pertence ao conjunto $W^s(\infty)$. 

Desse modo, considere o conjunto $\Lambda_n = \{x \in I : F^n_\mu (x) \in I\}$, isto é, o conjunto formado pelos pontos de $I$ que permanecem em $I$ após $n$ iterações de $F_\mu$ e considere o conjunto $\Lambda = \cap \Lambda_n$, isto é, o conjunto formado pelos pontos de $I$ que permanecem em $I$ por iteração de $F_\mu$. Portanto, pela observação anterior, resta entender a dinâmica de $F_\mu$ nos pontos de $\Lambda$ e, para isso, é necessário entender o que é o conjunto $\Lambda$.

\begin{proposition}
\label{proposition 4}
Suponha $\mu > 4$.
\begin{enumerate}
\item $\Lambda_1 = [0, \alpha] \cup [\beta, 1]$, onde $\alpha = \frac{1}{2} - \frac{\sqrt{\mu^2 - 4\mu}}{2\mu}$ e $\beta = \frac{1}{2} + \frac{\sqrt{\mu^2 - 4\mu}}{2\mu}$.
\item $\Lambda_n$ é a união de $2^n$ intervalos fechados disjuntos.
\item Se $J$ é um dos intervalos fechados que formam $\Lambda_n$, então $F^n_\mu: J \rightarrow I$ é um homeomorfismo.
\end{enumerate}
\end{proposition}

\begin{proof}
Analisando $F'_\mu$ observamos que $F_\mu$ é estritamente crescente no intervalo $[0, \frac{1}{2}]$ e estritamente decrescente no intervalo $[\frac{1}{2}, 1]$. Como $F_\mu (0) = F_\mu (1) = 0$ e $F_\mu (\frac{1}{2}) > 1$ o Teorema do Valor Intermediário implica que existem $\alpha$ e $\beta$ tais que $F_\mu (\alpha) = F_\mu (\beta) = 1$. Os valores de $\alpha$ e $\beta$ são encontrados resolvendo a equação de segundo grau $\mu x(1-x) = 1$. Logo, $F_\mu ([0, \alpha]) = F_\mu ([\beta, 1]) = [0, 1]$ e $F_\mu(x) > 1$ para todo $x \in (\alpha, \beta)$. Portanto, $\Lambda_1 = [0, \alpha] \cup [\beta, 1]$ e o item 1 está demonstrado.

A demonstração dos itens $2$ e $3$ será feita por indução. Pela primeira parte dessa demonstração, $\Lambda_1$ é a união de $2^1 = 2$ intervalos fechados disjuntos e $F_\mu$ é um homeomorfismo em cada um deles pois $F_\mu([0, \alpha]) > 0$ e $F_\mu([\beta, 1]) < 0$.

Suponha que $\Lambda_k$ é a união de $2^k$ intervalos fechados disjuntos de modo que $F^k_\mu: J \rightarrow I$ é um homeomorfismo para cada intervalo $J$ que forma $\Lambda_k$.

Sendo $F^k_\mu$ um homeomorfismo, então $F^k_\mu (x) > 0$ ou $F^k_\mu (x) < 0$ para todo $x \in J$. Suponha que $F^k_\mu (x) > 0$. O caso $F^k_\mu (x) < 0$ é tratado de maneira análoga.

Existem $a, x_1, x_2, b \in J$, $a < x_1 < x_2 < b$, tais que $F^k_\mu (a) = 0$, $F^k_\mu (x_1) = \alpha$, $F^k_\mu (x_2) =\beta$ e $F^k_\mu (b) = 1$. Considere os intervalos $J_1 = [a, x_1]$ e $J_2 = [x_2,b]$. Desse modo, $F^{k+1}_\mu (J_1) = F_\mu(F^k_\mu (J_1)) = F_\mu([0, \alpha]) = [0, 1]$ e, analogamente, $F^{k+1}_\mu (J_2) = [0, 1]$. Como $(F^{k+1}_\mu)' (J_1) = F'_\mu(F^k_\mu (J_1))(F^k_\mu)' (J_1) = F'_\mu([0, \alpha])(F^k_\mu)' (J_1) > 0$ e, analogamente, $(F^{k+1}_\mu)' (J_2) = F'_\mu([\beta, 1])(F^k_\mu)' (J_2) < 0$. Logo, $F^{k+1}_\mu$ é um homeomorfismo entre $J_1$ e $I$ e entre $J_2$ e $I$.

A partir de cada intervalo fechado de $\Lambda_k$ construímos dois novos intervalos fechados disjuntos tais que $F^{k+1}_\mu$ restrita em cada um desses intervalos é um homeomorfismo sobre $I$ e, portanto, esses intervalos fazem parte de $\Lambda_{k+1}$. Desse modo, se $\Lambda_k$ é formado por $2^k$ intervalos, então $\Lambda_{k+1}$ é formado por $2\cdot2^k = 2^{k+1}$ intervalos fechados disjuntos. Assim, o resultado está provado.
\end{proof}

Vamos mostrar agora que $\Lambda$ é um conjunto de Cantor quando $\mu > 2 + \sqrt{5}$ e, para isso, utilizaremos o resultado abaixo.

\begin{lemma}
Suponha $\mu > 2 + \sqrt{5}$. Existe $\lambda > 1$ tal que $|F'(x)| > \lambda$ para todo $x \in \Lambda_1$. Além disso, o tamanho de cada intervalo em $\Lambda_n$ é menor que $(\frac{1}{\lambda})^n$.
\end{lemma}

\begin{proof}
Para provar a primeira parte, observamos inicialmente que $\mu^2 - 4\mu > 1$ quando $\mu > 2 + \sqrt{5}$. Desse modo, $F'(x_1) = \sqrt{\mu^2 - 4\mu} > 1$ e $F'(x_2) = -\sqrt{\mu^2 - 4\mu} < -1$, onde $x_1 = \frac{1}{2} - \frac{\sqrt{\mu^2 - 4\mu}}{2\mu}$ e $x_2 = \frac{1}{2} + \frac{\sqrt{\mu^2 - 4\mu}}{2\mu}$.

Temos também que $F'$ é estritamente decrescente, pois $F''(x) = -2\mu < 0$.  Portanto, $F'(x) \geq F'(x_1) > 1$ para todo $x \in [0, x_1]$ e $F'(x) \leq F'(x_2) < -1$ para todo $x \in [x_2, 1]$. De acordo com a Proposição \ref{proposition 4}, $\Lambda_1 = [0, x_1] \cup [x_2, 1]$ e, desse modo, $|F'(x)| > 1$ para todo $x \in \Lambda_1$. (Sendo $F'$ contínua?)Portanto, existe $\lambda > 1$ tal que $|F'(x)| > 1$ para todo $x \in \Lambda_1$.

Ainda de acordo com a Proposição \ref{proposition 4}, $\Lambda_n$ é formado pela união de $2^n$ intervalos disjuntos. Seja $[a, b]$ um desses intervalos. Observe que $F^n(a), F^n(b) \in \{0, 1\}$ e $F^n(a) \neq F^n(b)$. Se $c \in [a, b]$, em particular $c \in \Lambda_1$ e, portanto, $(F^n)'(c) = F'(F^{n-1}(c)) \cdots F'(c) > \lambda^n$. Pelo Teorema do Valor Médio, existe $c \in [a, b]$ tal que $1 = |F^n(b) - F^n(a)| = |(F^n)'(c)|b - a| > \lambda^n|b - a|$. Desse modo, $|b - a| < \frac{1}{\lambda^n}$ e a segunda parte está provada.
\end{proof} 

\begin{theorem}
\label{Lambda is Cantor}
Suponha $\mu > 2 + \sqrt{5}$. Então $\Lambda$ é um conjunto de Cantor.
\end{theorem}

\begin{proof}
Como $F^n_\mu(0) = 0 \in I$ para todo $n$ temos $0 \in \Lambda$ e, portanto, $\Lambda$ é não vazio. Como $\Lambda_n \in I$ para todo $n$ temos que $\Lambda$ é limitado. Como $\Lambda$ é intersecção de conjuntos fechados  temos que $\Lambda$ é fechado.

Agora, suponha que $\Lambda$ contém algum intervalo. Então, existem $x, y \in I$, $x < y$, tais que $[x, y] \subset \Lambda$. Existe $N$ tal que $(\frac{1}{\lambda})^N < |x - y|$. De acordo com o Lema \ref{Lambda is Cantor}, isso implica que $[x, y] \notin \Lambda_N$ pois os intervalos de $\Lambda_N$ possuem tamanho menor que $(\frac{1}{\lambda})^N$. Absurdo e, portanto, $\Lambda$ não possui intervalos.

Por fim, observe que, se $x$ é um ponto extremo de algum intervalo de $\Lambda_n$, então $x \in \Lambda$ pois $F^{n+1}_\mu (x) = 0$. Sejam $x \in \Lambda$ e $\varepsilon > 0$.  Existe $N$ tal que $(\frac{1}{\lambda})^N < \varepsilon$. Como $x \in \Lambda$, temos que $x \in \Lambda_N$ e, portanto, $x$ é elemento de algum intervalo cujo tamanho é menor que $\varepsilon$. Portanto, existe $y$ ponto extremo do intervalo que contém $x$ tal que $|x - y| < \varepsilon$. Pela observação feita no início do parágrafo, $y \in \Lambda$. Como $\varepsilon$ é arbitrário, temos que $x$ é um ponto de acumulação de $\Lambda$.

Concluímos então que $\Lambda$ é um conjunto de Cantor.
\end{proof}

\begin{remark}
O Teorema \ref{Lambda is Cantor} é válido para $4 < \mu \leq 2 + \sqrt{5}$, porém a demonstração é mais complicada.
\end{remark}

Durante o restante da seção, vamos definir o conceito de caos e mostrar que $F$ é uma função caótica para $\mu > 4$.

\begin{definition}
Seja $f: D \rightarrow D$ uma função. Dizemos que $f$ é topologicamente transitiva se dados $x, y \in D$ e $\varepsilon > 0$,  existem $z \in D$ e $k \geq 1$ tais que $|z - x| < \varepsilon$ e $|f^k(z) - y| < \varepsilon$.
\end{definition} 

Intuitivamente, uma função $f$ topologicamente transitiva é uma função mistura quaisquer dois conjuntos abertos, isto é, existe um ponto de um dos conjuntos que é levado para o outro conjunto após um número finito de iterações da $f$.

\begin{proposition}
Se $\mu > 2 + \sqrt{5}$, então $F$ é topologicamente transitiva.
\end{proposition}

\begin{proof}

\end{proof}

\begin{definition}
Seja $f: D \rightarrow D$ uma função. Dizemos que $f$ depende sensivelmente das condições iniciais se para algum $\delta > 0$, dados $x \in D$ e $\varepsilon > 0$, existem $y \in D$ e $k \geq 1$ tais que $|x - y| < \varepsilon$ e $|f^k(x) - f^k(y)| > \delta$.
\end{definition}

\begin{proposition}
Se $\mu > 2 + \sqrt{5}$, então $F$ depende sensivelmente das condições iniciais.
\end{proposition}

\begin{proof}

\end{proof}

Podemos agora definir o que é um função caótica.

\begin{definition}
Seja $f: D \rightarrow D$ uma função. Dizemos que $f$ é caótica se
\begin{enumerate}
\item O conjunto de pontos periódicos de $f$ é denso.
\item $f$ é topologicamente transitiva.
\item $f$ depende sensivelmente das condições iniciais.
\end{enumerate}
\end{definition}

\begin{theorem}
Se $\mu > 2 + \sqrt{5}$, então $F$ é caótica.
\end{theorem}

\begin{proof}

\end{proof}

\end{document}

% substituir o intervalo I,J por A,B no primeiro parágrafo
% nome das definções em itálico
% remover \mu de F_\mu
% arrrumar a indução de k para k-1 da proposição ~4.2
% lema 4.3 provar que (F^n)'(c) > 1: mostrar que F^K(c) \in A_1 para todo k = 0, .., n-1 
% mostrar em algum lugar que \Lambda_n contém \Lambda_{n+1}
% provar em algum lugar que Per(F) é denso quando mu > 4