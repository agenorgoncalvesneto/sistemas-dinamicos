\documentclass[a4paper, 12pt]{article}

\usepackage[utf8]{inputenc}
\usepackage[brazil]{babel}

\usepackage{enumitem}

%abnt
\usepackage[left=3cm, top=3cm, right=2cm, bottom=2cm]{geometry}
\usepackage[onehalfspacing]{setspace}
\usepackage{indentfirst}

%math
\usepackage{amsmath}
\usepackage{amsthm}
\usepackage{amsfonts}
\usepackage{amssymb}
\usepackage{dsfont}
\usepackage{mathtools}

\theoremstyle{definition}
\newtheorem{definition}{Definição}[section]
\theoremstyle{plain}
\newtheorem{proposition}[definition]{Proposição}
\theoremstyle{plain}
\newtheorem{lemma}[definition]{Lema}
\theoremstyle{plain}
\newtheorem{theorem}[definition]{Teorema}
\theoremstyle{remark}
\newtheorem*{remark}{Observação}

%shortcuts
\newcommand{\RR}{\mathbb{R}}
\newcommand{\NN}{\mathbb{N}}

\begin{document}

\section{Definições Elementares}

%Ao longo do texto, $f$ representará uma função $f: I \rightarrow J$, onde $I$ e $J$ são subconjuntos de $\RR$. Além disso, a função $f^n$ representará a n-ésima iterada de $f$ e $f^{(n)}$ representará n-ésima derivada de $f$.
Começamos nosso estudo definindo o que são pontos periódicos e pontos eventualmente periódicos. Esses pontos desempenham um papel central no estudos de Sistemas Dinâmicos.
 
\begin{definition}
Sejam $f: X \rightarrow X$ uma função, $p \in X$ e $n \geq 1$.  Dizemos que $p$ é um \textit{ponto periódico com período $n$}, se $f^n(p) = p$. Se $f^k(p) \neq p$ para todo $1 \leq k < n$, então $n$ é chamado de \textit{período principal}. Em particular, se $n=1$, dizemos que $p$ é um \textit{ponto fixo}.%Denotamos o conjunto de todos os pontos periódicos de $f$ por $Per(f)$ e denotamos o conjunto de todos os pontos periódicos de $f$ com período $n$ por $Per_n(f)$. 
\end{definition}

\begin{definition}
Sejam $f: X \rightarrow X$ uma função, $p \in X$ e $n \geq 1$. Dizemos que $p$ é um \textit{ponto eventualmente periódico com período $n$}, se existe $m > 1$ tal que $f^k(p) = f^{k+n}(p)$ para todo $k \geq m$. Em particular, se $n = 1$, dizemos que p é um \textit{ponto eventualmente fixo}.
\end{definition}

\begin{definition}
Sejam $f:X \rightarrow X$ uma função e $x \in X$. A \textit{órbita de $x$} é o conjunto $O(x) = \lbrace x, f(x), f^2(x), \dots \rbrace$.
\end{definition}

\begin{definition}
Sejam $f: X \rightarrow X$ uma função, $p$ um ponto periódico período $n$ e $x \in X$. Dizemos que \textit{$x$ tende assintoticamente para $p$} se $lim_{k \rightarrow \infty} f^{kn}(x) = p$. O conjunto dos pontos que tendem assintoticamente para $p$, denotado por $W^s(p)$, é chamado chamado de \textit{conjunto estável de $p$}. Dizemos que \textit{$x$ tende assintoticamente para infinito} se $lim_{k \rightarrow \infty} |f^{k}(x)| = \infty$. O conjunto dos pontos que tendem assintoticamente para infinito, denotado por $W^s(\infty)$, é chamado de \textit{conjunto estável do infinito}.
\end{definition}

A Proposição abaixo nos mostra que os conjuntos estáveis de dois pontos periódicos distintos possuem intersecção vazia
\begin{proposition}
Sejam $f: X \rightarrow X$ uma função e $p_1$, $p_2$ pontos periódicos distintos. Então $W^s(p_1) \cap W^s(p_2) = \emptyset$.
\end{proposition}

\begin{proof}
Sejam $n_1$, $n_2$ os períodos de $p_1$, $p_2$, respectivamente. Suponha que exista $x \in W^s(p_1) \cap W^s(p_2)$. Sabemos que $|f^{kn_1}(x) - p_1| \longrightarrow 0$ e $|f^{kn_2}(x) - p_2| \longrightarrow 0$, quando $k \longrightarrow \infty$. Desse modo, dado $\varepsilon > 0$ existe $N \geq 1$ tal que $|f^{kn_1}(x) - p_1| < \frac{\varepsilon}{2}$ e $|f^{kn_2}(x) - p_2| < \frac{\epsilon}{2}$ para todo $k > N$. Portanto, $|p_1 - p_2| = |p_1 - f^{kn_1n_2}(x) + f^{kn_1n_2}(x) - p_2| \leq |f^{kn_2n_1}(x) - p_1| + |f^{kn_1n_2}(x) - p_2| < \varepsilon$. Temos então que $p = q$, pois $\varepsilon$ é arbitrário. Absurdo.
\end{proof}

O objetivo do estudo de Sistemas Dinâmicos é entender a natureza das órbitas, identificando pontos periódicos, eventualmente periódicos, que tendem assintoticamente, etc.

\section{Implicações da Diferenciabilidade}

Nessa seção, estudaremos as implicações da diferenciabilidade na dinâmica de uma função real. Caso não seja dito o contrário, $I$ representará um intervalo fechado de $\RR$.  

\begin{proposition}
\label{proposition 2 1}
Seja $f: I \rightarrow \RR$ uma função contínua. Se $f(I) \subset I$ ou $f(I) \supset I$, então $f$ possui ponto fixo.
\end{proposition}

\begin{proof}
Seja $I = [a, b]$. Suponha que $f(I) \subset I$. Considere a função contínua $g(x) = f(x) - x$ definida em $I$. Como $f(a), f(b) \in I$, temos que $g(a) = f(a) - a \geq 0$ e $g(b) = f(b) - b \leq 0$. Pelo Teorema do Valor Intermediário, existe $p \in I$ tal que $g(p) = f(p) -p = 0$. Desse modo, $p$ é ponto fixo de $f$.

Suponha que $f(I) \supset I$. Por definição, existem $c, d \in I$ tais que $f(c) = a$ e$f(d) = b$. Considere a função contínua $g(x) = f(x) - x$ definida em $I$. Temos que $g(c) = a - c \leq 0$ e $g(d) = b - d \geq 0$. Pelo Teorema do Valor Intermediário, existe $p \in I$ tal que $g(p) = f(p) - p = 0$. Desse modo, $p$ é ponto fixo de $f$.
\end{proof}

\begin{theorem}
Seja $f:I \rightarrow I$ uma função diferenciável. Se $|f'(x)|<1$ para todo $x \in I$, então $f$ admite um único ponto fixo e $|f(x) - f(y)| < |x - y|$ para todo $x, y \in I$ distintos.
\end{theorem}

\begin{proof}
Sejam $x, y \in I$, $x < y$. Pelo Teorema do Valor Médio, existe $c \in [x, y]$ tal que $f(x) - f(y) = f'(c)(x - y)$. Portanto, $|f(x) - f(y)| = |f'(c)||x - y| < |x - y|$.

Pela Proposição \ref{proposition 2 1}, $f$ admite um ponto fixo $p$. Suponha que exista um ponto fixo $q$ diferente de $p$. Então, pela primeira parte da demonstração, $|p - q| = |f(p) - f(q)| < |p - q|$. Absurdo.
\end{proof}

Introduziremos agora a noção de ponto hiperbólico para uma função diferenciável e logo após provaremos um resultado que ajuda a compreender a dinâmica da função em uma vizinhança desses pontos.

\begin{definition}
Sejam $f: I \rightarrow I$ uma função diferenciável e $p$ um ponto periódico com período principal $n$. Dizemos que $p$ é um \textit{ponto hiperbólico} se $|(f^n)'(p)| \neq 1$. Se $|(f^n)'(p)| > 1$, dizemos que $p$ é um \textit{ponto atrator} e se $|(f^n)'(p)| < 1$, dizemos que $p$ é um \textit{ponto  repulsor}. Dizemos que $p$ é um \textit{ponto não hiperbólico} se $|(f^n)'(p)| = 1$.
\end{definition}

O Teorema abaixo ajuda a compreender o porquê dos nomes \textit{atrator} e \textit{repulsor} para um ponto hiperbólico.

\begin{theorem}
Sejam $f: I \rightarrow I$ uma função $C^1$ e $p$ um ponto periódico com período principal $n$. Se $p$ é um ponto hiperbólico atrator, existe uma vizinhança de $p$ contida em $W^s(p)$. Se $p$ é um ponto hiperbólico repulsor, existe uma vizinhança $U$ de $p$ tal que, se $x \in U$ e $x \neq p$, $f^{kn}(x) \notin U$ para algum $k \geq 1$. 
\end{theorem}

\begin{proof}
Suponha que $p$ é um ponto hiperbólico atrator. Como $f'$ é contínua, existe $\varepsilon > 0$ tal que $|(f^n)'(x)| \leq \lambda < 1$ para todo $x \in (p - \varepsilon, p + \varepsilon)$. Pelo Teorema do Valor Médio, se $x \in U$ então $|f^n(x) - p| = |f^n(x) - f^n(p)| \leq \lambda|x - p|$. Por indução, $|f^{kn}(x) - p| \leq \lambda^k|x - p|$. Desse modo, $f^{kn}(x) \longrightarrow p$ quando $k \longrightarrow \infty$.

Suponha que $p$ é ponto hiperbólico repulsor. De maneira análoga, existe $\varepsilon > 0$ tal que $|(f^n)'(x)| \geq \lambda > 1$ para todo $x \in (p- \varepsilon, p + \varepsilon)$. Fixado $x \in (p - \varepsilon, p + \varepsilon)$, $x \neq p$, suponha que $f^{kn}(x) \in (p - \varepsilon, p + \varepsilon)$ para todo $k \geq 1$. Pelo Teorema do Valor Médio, $|f^{kn}(x) - p| \geq \lambda^k|x - p|$ para todo $k \geq 1$. Absurdo, pois $\lambda^k|x - p| \longrightarrow \infty$ quando $k \longrightarrow \infty$.
\end{proof}

\begin{remark}
A segunda parte do teorema afirma que existe uma vizinhança de $p$ tal que todo ponto diferente de $p$ nessa vizinhança é movida para fora dela após um número de iterações da $f$. Observe o ponto pode voltar para vizinhança após mais iterações da $f$, justamente por sabermos que o valor da derivada é maior que um apenas nessa vizinhança.
\end{remark}

\section{Família Quadrática I: Estudo inicial}

Nosso objetivo será estudar, durante essa seção e as próximas, é estudar a dinâmica da família de funções $F_{\mu}: \RR \rightarrow \RR$ dadas por $F_{\mu}(x) = \mu x(1-x)$, onde $\mu > 0$. Tal família é chamada de família quadrática. Quando não houver ambiguidade, escreveremos $F$ ao invés de $F_\mu$. Nessa seção estudaremos a dinâmica de $F$ quando $1 < \mu < 3$.

\begin{proposition}
\begin{enumerate}
\item $F (1) = F(0) = 0$ e $F(\frac{1}{\mu}) = F(p_\mu) = p_\mu$, onde $p_\mu = \frac{\mu - 1}{\mu}$.
\item Se $\mu > 1$ então $0 < p_\mu < 1$.
\item O vértice da parábola é o ponto $(\frac{1}{2}, \frac{\mu}{4})$.
\end{enumerate}
\end{proposition}

\begin{proof}
Apenas aplicação direta das definições.
\end{proof}

\begin{proposition}
\label{proposition 3 1}
Se $\mu > 1$, então $(-\infty, 0) \cup (1, \infty) \subset W^s(\infty)$.
\end{proposition}

\begin{proof}
Se $x < 0$, a sequência  $x, F(x), F^2(x), \dots$ é estritamente decrescente pois  $F(x) < x$. Se $(F^n(x))_n \longrightarrow x_0$ quando $n \longrightarrow \infty$, a continuidade de $F$ implica que $(F^{n+1}(x))_n \longrightarrow F (x_0) < x_0$. Absurdo. Portanto, $(F^n(x))_n \longrightarrow -\infty$ quando $n \longrightarrow \infty$. Como $F(x) < 0$ para todo $x > 1$, concluímos que $(-\infty, 0) \cup (1, \infty) \subset W^s(\infty)$.
\end{proof}

Pelo Proposição anterior, conhecemos a dinâmica da $F$, quando $\mu > 1$, nos pontos menores que zero e maiores que um. Portanto, nos resta estudar a dinâmica de $F$ restrita ao intervalo $[0, 1]$.

\begin{proposition}
Se $1 < \mu < 3$, então
\begin{enumerate}
\item $0$ é um ponto repulsor e $p_\mu$ é um ponto atrator.
\item $lim_{n \rightarrow \infty} F^n(x) = p_\mu$ para todo $x \in (0, 1)$.
\end{enumerate}
\end{proposition}

\begin{proof}
A primeira parte é verdadeira pois $|F'(0)| = \mu > 1$ e $|F'(p_\mu)| = |2 - \mu| < 1$, quando $1 < \mu < 3$.

Falta provar o item 2.
%Para mostrar a segunda parte, dividiremos em casos. Suponha inicialmente que $1 < \mu < 2$. Como $F_\mu(x) \leq \frac{\mu}{4} < 1/2$ para todo $x \in (0, 1)$, basta analisar o limite quando $x \in (0, \frac{1}{2})$. Se $x \in (0, \frac{1}{2})$ então, analisando a função $F_\mu(x) - x$, temos que $|F_\mu(x) - p_\mu| < |x - p_\mu|$ e, pelo continuidade de $F_\mu$, $lim_n F^n_\mu(x) = p_\mu$.
\end{proof}

Desse modo, pelas Proposições anteriores, conhecemos completamente a dinâmica de $F$ quando $1 < \mu < 3$: $W^s(0) = \{0, 1\}$, $W^s(p_\mu) = (0, 1)$ e $W^s(\infty) = (-\infty, 0) \cup (1, \infty)$.

\section{Família Quadrática II: Conjuntos de Cantor e Caos}

Analisaremos nessa seção a dinâmica de $F$ quando $\mu > 4$. Para entendê-la, estudaremos o que são conjuntos de Cantor e o conceito de caos.

Observamos inicialmente que $F(\frac{1}{2})$ = $\frac{\mu}{4} > 1$ quando $\mu > 4$, ou seja, existem pontos em $[0, 1]$ que não permanecem em $[0, 1]$ após uma iteração de $F$. Em vista da Proposição \ref{proposition 3 1}, a dinâmica de $F$ em tais pontos é determinada, pois pertencem ao conjunto $W^s(\infty)$. De modo mais geral, se um ponto de $[0, 1]$ não permanece $[0, 1]$ após um número finito de iterações, então ele pertence ao conjunto $W^s(\infty)$. 

Desse modo, considere o conjunto $\Lambda_n = \{x \in [0, 1] : F^n(x) \in [0, 1]\}$, que é formado pelos pontos que permanecem em $[0, 1]$ após $n$ iterações de $F$, e considere o conjunto $\Lambda = \cap \Lambda_n = \{x \in [0, 1] : F^n(x) \in [0, 1]$ para todo $n \geq 1\}$, que é formado pelos pontos que permanecem para sempre em $[0, 1]$ por iterações de $F$. Observe que, por definição, $\Lambda_n \supset \Lambda_{n+1}$ para todo $n \geq 1$.

Nos resta, portanto, estudar a dinâmica de $F$ restrita ao conjunto $\Lambda$. A Proposição a seguir nos ajuda a começar compreender a natureza de $\Lambda$.

\begin{proposition}
\label{proposition 4 1}
Se $\mu > 4$, então
\begin{enumerate}
\item $\Lambda_1 = [0, x_1] \cup [x_2, 1]$, onde $x_1 = \frac{1}{2} - \frac{\sqrt{\mu^2 - 4\mu}}{2\mu}$ e $x_2 = \frac{1}{2} + \frac{\sqrt{\mu^2 - 4\mu}}{2\mu}$.
\item $\Lambda_n$ é a união de $2^n$ intervalos fechados disjuntos.
\item Se $I$ é um dos $2^n$ intervalos fechados disjuntos que formam $\Lambda_n$, então $F^n: I \rightarrow [0, 1]$ é bijetora.
\end{enumerate}
\end{proposition}

\begin{proof}
Analisando $F'$ observamos que $F$ é estritamente crescente no intervalo $[0, \frac{1}{2}]$ e estritamente decrescente no intervalo $[\frac{1}{2}, 1]$. Como $F(0) = F(1) = 0$ e $F(\frac{1}{2}) > 1$, o Teorema do Valor Intermediário garante que existem $x_1$ e $x_2$ tais que $F(x_1) = F(x_2) = 1$. Os valores de $x_1$ e $x_2$ são encontrados resolvendo a equação de segundo grau $\mu x(1-x) = 1$. Logo, $F([0, x_1]) = F([x_2, 1]) = [0, 1]$ e $F(x) > 1$ para todo $x \in (x_1, x_2)$. Portanto, $\Lambda_1 = [0, x_1] \cup [x_2, 1]$ e o item 1 está demonstrado.

A demonstração dos itens $2$ e $3$ será feita por indução. Pela primeira parte dessa demonstração, $\Lambda_1$ é a união de $2^1 = 2$ intervalos fechados disjuntos e $F$ restrita é cada um desses intervalos é uma bijeção com o intervalo $[0,1]$.

Suponha que $\Lambda_{k-1}$ é a união de $2^{k-1}$ intervalos fechados disjuntos de modo que $F^{k-1}: [a, b] \rightarrow [0, 1]$ é bijetora para todo intervalo $[a, b]$ que forma $\Lambda_{k-1}$. Sendo $F^{k-1}$ bijetora, $(F^{k-1})'(x) > 0$ ou $(F^{k-1})'(x) < 0$ para todo $x \in [a, b]$. Como as demonstrações para os dois casos são análogas, supomos que $(F^{k-1})'(x) > 0$.

Como $F^{k-1}$ é estritamente crescente, o Teorema do Valor Intermediário afirma que existem únicos $\overline{x_1}, \overline{x_2} \in [a, b]$ tais que
\begin{enumerate}[label=(\alph*)]
\item$a < \overline{x_1} < \overline{x_2} < b$,
\item$F^{k-1}([a, \overline{x_1}]) = [0, x_1]$,
\item$F^{k-1}((\overline{x_1}, \overline{x_2})) = (x_1, x_2)$ e
\item$F^{k-1}([\overline{x_2}, 1]) = [x_2, 1]$.
\end{enumerate}

Desse modo, $F^k([a, \overline{x_1}]) = F([0, x_1]) = [0, 1]$ e, analogamente, $F^k([\overline{x_2}, 1]) = [0, 1]$. Além disso, $(F^k)'([a, \overline{x_1}]) = F'(F^{k-1}([a, \overline{x_1}]))(F^{k-1})'([a, \overline{x_1}]) = F'([0, x_1])(F^{k-1})'([a, \overline{x_1}]) > 0$ e, analogamente, $(F^k)'([\overline{x_2}, 1]) = F'([x_2, 1])(F^{k-1})'([\overline{x_2}, 1]) < 0$. Logo, $F^k$ é uma bijeção entre $[a, \overline{x_1}]$ e $[0, 1]$ e entre $[\overline{x_2}, 1]$ e $[0, 1]$.

Portanto, a partir de cada intervalo fechado de $\Lambda_{k-1}$, construímos dois novos intervalos fechados disjuntos tais que $F^k$ restrita em cada um desses intervalos é um bijeção com $[0, 1]$ e, dessa maneira, esses intervalos estão contidos em $\Lambda_k$. Desse modo, se $\Lambda_{k-1}$ é formado por $2^{k-1}$ intervalos, então $\Lambda_k$ é formado por $2 \cdot 2^{k-1} = 2^k$ intervalos fechados disjuntos. Assim, o resultado está provado.
\end{proof}

Vamos agora definir o que é um conjunto de Cantor para prosseguir entendendo a natureza de $\Lambda$.

\begin{definition}[Conjunto de Cantor]
Um conjunto $\Gamma \subset \RR$ não vazio é um \textit{conjunto de Cantor} se
\begin{enumerate}
\item $\Gamma$ é fechado e limitado.
\item $\Gamma$ não possui intervalos.
\item Todo ponto de $\Gamma$ é um ponto de acumulação de $\Gamma$.
\end{enumerate}
\end{definition}

No restante dessa seção, restringiremos nossa atenção para o caso $\mu > 2 + \sqrt{5}$, para facilitar as demonstrações.

\begin{lemma}
\label{lemma 4 1}
Se $\mu > 2 + \sqrt{5}$, então existe $\lambda > 1$ tal que $|F'(x)| > \lambda$ para todo $x \in \Lambda_1$. Além disso, o tamanho de cada intervalo fechado em $\Lambda_n$ é menor que $\frac{1}{\lambda^n}$.
\end{lemma}

\begin{proof}
Para provar a primeira parte, observamos inicialmente que $\mu^2 - 4\mu > 1$ quando $\mu > 2 + \sqrt{5}$. Desse modo, $F'(x_1) = \sqrt{\mu^2 - 4\mu} > 1$ e $F'(x_2) = -\sqrt{\mu^2 - 4\mu} < -1$, onde $x_1 = \frac{1}{2} - \frac{\sqrt{\mu^2 - 4\mu}}{2\mu}$ e $x_2 = \frac{1}{2} + \frac{\sqrt{\mu^2 - 4\mu}}{2\mu}$.

Observamos também que $F'$ é estritamente decrescente, pois $F''(x) = -2\mu < 0$.  Portanto, $F'(x) \geq F'(x_1) > 1$ para todo $x \in [0, x_1]$ e $F'(x) \leq F'(x_2) < -1$ para todo $x \in [x_2, 1]$. De acordo com a Proposição \ref{proposition 4 1}, $\Lambda_1 = [0, x_1] \cup [x_2, 1]$ e, desse modo, $|F'(x)| > 1$ para todo $x \in \Lambda_1$. Sendo $F'$ contínua e $\Lambda_1$ compacto, existe $\lambda > 1$ tal que $|F'(x)| > \lambda$ para todo $x \in \Lambda_1$.

Ainda de acordo com a Proposição \ref{proposition 4 1}, $\Lambda_n$ é formado pela união de $2^n$ intervalos disjuntos. Seja $[a, b]$ um desses intervalos. Se $c \in [a, b]$, em particular $c \in \Lambda_1$ e, portanto, $(F^n)'(c) = F'(F^{n-1}(c)) F'(F^{n-2}(c)) \dots F'(c) > \lambda^n$. Pelo Teorema do Valor Médio, existe $c \in [a, b]$ tal que $|F^n(b) - F^n(a)| = |(F^n)'(c)||b - a| > \lambda^n|b - a|$. Como $F^n: [a, b] \rightarrow [0 ,1]$ é um bijeção, temos que $|F^n(b) - F^n(a)| = 1$. Desse modo, $|b - a| < \frac{1}{\lambda^n}$ e a segunda parte está provada.
\end{proof} 

\begin{theorem}
\label{theorem 4 1}
Se $\mu > 2 + \sqrt{5}$, então $\Lambda$ é um conjunto de Cantor.
\end{theorem}

\begin{proof}
$\Lambda$ é não vazio pois $0 \in \Lambda$, é limitado pois $\Lambda_1 \in [0, 1]$ e é fechado pois é intersecção de conjuntos fechados.

Agora, suponha que $\Lambda$ contém algum intervalo. Então, existem $x, y \in I$, $x < y$, tais que $[x, y] \subset \Lambda$. Seja $k$ tal que $\frac{1}{\lambda^k} < |x - y|$. Em particular, $[x, y] \subset \Lambda_k$. Mas, de acordo com o Lema \ref{lemma 4 1}, os intervalos de $\Lambda_k$ possuem tamanho menor que $\frac{1}{\lambda^k}$. Absurdo e, portanto, $\Lambda$ não possui intervalos.

Por fim, observe que, se $x$ é um ponto extremo de algum intervalo de $\Lambda_n$, então $x \in \Lambda$ pois $F^{n+1}(x) = 0$. Sejam $x \in \Lambda$, $\varepsilon > 0$ e $k \geq 1$ tal que $\frac{1}{\lambda^k} < \varepsilon$. Em particular, $x \in \Lambda_k$ e, portanto, $x$ é elemento de algum intervalo cujo tamanho é menor que $\varepsilon$, de acordo com o Lema \ref{lemma 4 1}. Portanto, existe $y \in \Lambda$ ponto extremo do intervalo que contém $x$ tal que $|x - y| < \varepsilon$. Como $\varepsilon$ é arbitrário, concluímos que $x$ é um ponto de acumulação de $\Lambda$.
\end{proof}

Com o Lema \ref{lemma 4 1}, também podemos mostrar que o conjunto de pontos periódicos de $F$ é denso quando $\mu > 2 + \sqrt{5}$.

\begin{proposition}
\label{proposition 4 2}
Se $\mu > 2 + \sqrt{5}$, então o conjunto de pontos periódicos de $F: \Lambda \rightarrow \Lambda$ é denso em $\Lambda$.
\end{proposition}

\begin{proof}
Sejam $x \in \Lambda$, $\varepsilon > 0$ e $k \geq 1$ tal que $\frac{1}{\lambda^k} < \varepsilon$. De acordo com o Lema \ref{lemma 4 1}, o intervalo fechado $I \subset \Lambda_k$ que contém $x$ possui tamanho menor que $\varepsilon$. Pela Proposição \ref{proposition 4 1}, $F^k: I \rightarrow [0, 1]$ é bijetora. Como $F^k(I) \supset I$, a Proposição \ref{proposition 2 1} afirma que existe $y \in I$ tal que $F^k(y) = y$. Observando que $y \in \Lambda$ e $|x - y| < \varepsilon$, o resultado está provado.
\end{proof}

No restante da seção, definiremos o que é uma função topologicamente transitiva, o que é uma função que depende sensivelmente das condições iniciais e, por fim, o que é uma função caótica. Vamos também mostrar $F$ possui cada uma dessas propriedades quando $\mu > 2 + \sqrt{5}$.

\begin{definition}
Seja $f: D \rightarrow D$ uma função. Dizemos que $f$ é \textit{topologicamente transitiva} se dados $x, y \in D$ e $\varepsilon > 0$,  existem $z \in D$ e $k \geq 1$ tais que $|z - x| < \varepsilon$ e $|f^k(z) - y| < \varepsilon$.
\end{definition} 

Intuitivamente, $f$ é uma função topologicamente transitiva se para todo par de conjuntos abertos existe um ponto de um dos conjuntos que é levado para o outro conjunto após um número finito de iterações da $f$.

\begin{proposition}
\label{proposition 4 3}
Se $\mu > 2 + \sqrt{5}$, então $F: \Lambda \rightarrow \Lambda$ é topologicamente transitiva.
\end{proposition}

\begin{proof}
Sejam $x, y \in \Lambda$ e $\varepsilon > 0$. Existe $k \geq 1$ tal que $\frac{1}{\lambda^k} < \varepsilon$. De acordo com o Lema \ref{lemma 4 1}, o tamanho de cada intervalo fechado em $\Lambda_k$ é menor que $\frac{1}{\lambda^k}$ e, portanto, menor que $\varepsilon$. Como $x \in \Lambda_ k$, existe um intervalo $[a, b] \subset \Lambda_ k$ que contém $x$. Pela Proposição \ref{proposition 4 1}, $F^k: [a, b] \rightarrow [0, 1]$ é bijetora e, pelo Teorema do Valor Intermediário, existe $z \in [a, b]$ tal que $F^k(z) = y$. Observando que $z \in \Lambda$, concluímos que $F$ é topologicamente transitiva.
\end{proof}

\begin{definition}
Seja $f: D \rightarrow D$ uma função. Dizemos que $f$ depende sensivelmente das condições iniciais se para algum $\delta > 0$, dados $x \in D$ e $\varepsilon > 0$, existem $y \in D$ e $k \geq 1$ tais que $|x - y| < \varepsilon$ e $|f^k(x) - f^k(y)| > \delta$.
\end{definition}

\begin{proposition}
\label{proposition 4 4}
Se $\mu > 2 + \sqrt{5}$, então $F: \Lambda \rightarrow \Lambda$ depende sensivelmente das condições iniciais.
\end{proposition}

\begin{proof}
Sejam $x \in \Lambda$ e $\varepsilon > 0$. Existe $k \geq 1$ tal que $\frac{1}{\lambda^k}$. Como na demonstração da Proposição anterior, seja $I$ o intervalo fechado contido em $\Lambda_k$ que contém $x$ e cujo tamanho é menor que $\varepsilon$. Como $F^k: I \rightarrow [0, 1]$ é um bijeção, então $F^k(a) = 0$ e  $F^k(b) = 1$, onde $a$ e $b$ são pontos extremos de $I$. Como $F(\frac{1}{2}) > 1$ e $x \in \Lambda$, segue que $F^{k}(x) \in [0, \frac{1}{2}) \cup (\frac{1}{2}, 1]$. Se $F^{k}(x) \in [0, \frac{1}{2})$, então $|F^k(x) - F^k(b)| = |F^k(x) - 1| > \frac{1}{2}$ e se $F^{k}(x) \in (\frac{1}{2}, 1]$, então $|F^k(x) - F^k(a)| = |F^k(x)| > \frac{1}{2}$. Observando que $|x - a| < \varepsilon$ e  $|x - b| < \varepsilon$, temos o resultado para $\delta = \frac{1}{2}$. 
\end{proof}

\begin{definition}
Seja $f: D \rightarrow D$ uma função. Dizemos que $f$ é caótica se
\begin{enumerate}
\item O conjunto de pontos periódicos de $f$ é denso.
\item $f$ é topologicamente transitiva.
\item $f$ depende sensivelmente das condições iniciais.
\end{enumerate}
\end{definition}

\begin{theorem}
\label{theorem 4 2}
Se $\mu > 2 + \sqrt{5}$, então $F: \Lambda \rightarrow \Lambda$ é caótica.
\end{theorem}

\begin{proof}
O resultado segue das Proposições \ref{proposition 4 2}, \ref{proposition 4 3} e \ref{proposition 4 4}.
\end{proof}

\begin{remark}
Os Teoremas \ref{theorem 4 1} e \ref{theorem 4 2} são válidos para $4 < \mu \leq 2 + \sqrt{5}$, porém a demonstração é mais complicada.
\end{remark}

\end{document}

% substituir o intervalo I,J por A,B no primeiro parágrafo
% nome das definções em itálico
% remover \mu de F_\mu
% arrrumar a indução de k para k-1 da proposição ~4.2
% lema 4.3 provar que (F^n)'(c) > 1: mostrar que F^K(c) \in A_1 para todo k = 0, .., n-1 
% mostrar em algum lugar que \Lambda_n contém \Lambda_{n+1}
% provar em algum lugar que Per(F) é denso quando mu > 4
% trocar alpha e beta por x_1 e x_2