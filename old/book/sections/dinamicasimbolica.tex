\section{Função Logística V: Dinâmica Simbólica}

\begin{definition}
$\Sigma_2 = \{s = (s_0s_1s_2\dots) : s_k = 0 \textrm{ ou } s_k = 1 \textrm{ para todo } k\geq 0\}$ é o \textit{espaço das sequências de $0$ e $1$}.
\end{definition}

\begin{proposition}
A função $d: \Sigma_2 \times \Sigma_2 \to \RR$, dada por
$$d(s, t) = \sum_{k=0}^{\infty} \frac{|s_k - t_k|}{2^k}$$
é uma distância em $\Sigma_2$.
\end{proposition}

\begin{proof}
Inicialmente, observamos que a função $d$ é bem definida pois
$$d(s, t) = \sum_{k=0}^{\infty} \frac{|s_k - t_k|}{2^k} \leq \sum_{k=0}^{\infty} \frac{1}{2^k} = 2$$
Além disso, é fácil verificar que as propriedades de uma distância são válidas em $d$, isto é,
\begin{enumerate}[label=(\alph*)]
\item $d(s, t) \geq 0$
\item $d(s, t) = 0$ se e somente se $s = t$
\item  $d(s, t) = d(t, s)$
\item $d(s, t) \leq d(s, r) + d(r, t)$
\end{enumerate}
para todo $r, s, t \in \Sigma_2$. Portando, $d$ é uma distância e a afirmação está provada.
\end{proof}

\begin{proposition}
\label{proposicao dinamicasimbolica 1}
Sejam $s = (s_0s_1s_2\dots) , t = (t_0t_1t_2\dots) \in \Sigma_2$. Se $s_k = t_ k$ para todo $0 \leq k \leq n$, então $d(s, t) \leq \frac{1}{2^n}$. Por outro lado, se $d(s, t) < \frac{1}{2^n}$, então $s_k = t_k$ para todo $0 \leq k \leq n$.
\end{proposition}

\begin{proof}
Suponha que $s_k = t_ k$ para todo $0 \leq k \leq n$. Desse modo, $$d(s, t) = \sum_{k=0}^{\infty} \frac{|s_k - t_k|}{2^k} = \sum_{k=n+1}^{\infty} \frac{|s_k - t_k|}{2^k} \leq \sum_{k=n+1}^{\infty} \frac{1}{2^k} = \frac{1}{2^n}$$
Por outro lado, se $s_i \neq t_i$ para algum $0 \leq i \leq n$, então $$d(s, t) = \sum_{k=0}^{\infty} \frac{|s_k - t_k|}{2^k} \geq \frac{1}{2^i} \geq \frac{1}{2^n}$$ Portanto, se $s_k = t_k$ para todo $0 \leq k \leq n$, concluímos que $d(s, t) < \frac{1}{2^n}$.
\end{proof}

\begin{definition}
A função $\sigma: \Sigma_2 \to \Sigma_2$, dada por $\sigma(s_0s_1s_2\dots) = (s_1s_2s_3\dots)$, é chamada de \textit{função shift}.
\end{definition}

\begin{proposition}
$\sigma$ é contínua.
\end{proposition}

\begin{proof}
Sejam $s = (s_0s_1s_2\dots) \in \Sigma_2$, $\varepsilon > 0$ e $n \geq 1$ tal que $\frac{1}{2^n} < \varepsilon$. Se $t = (t_0t_1t_2\dots) \in \Sigma_ 2$ e $d(s, t) < \frac{1}{2^{n+1}}$, então $s_k = t_k$ para todo $0 \leq k \leq n+1$, de acordo com a Proposição  \ref{proposicao dinamicasimbolica 1}. Como $\sigma(s) = (s_1s_2s_3\dots)$ e $\sigma(t) = (t_1t_2t_3\dots)$, temos que as primeiras $n+1$ entradas de $\sigma(s)$ e $\sigma(t)$ são iguais. Novamente, utilizando a Proposição \ref{proposicao dinamicasimbolica 1}, temos que $d(\sigma(s), \sigma(t)) \leq \frac{1}{2^n} < \varepsilon$. Como $s$ é um ponto arbitrário em $\Sigma_2$, concluímos que $\sigma$ é contínua.
\end{proof}

\begin{proposition}
Se $\sigma$ é a função shift, então
\begin{enumerate}
\item existem $2^n$ pontos periódicos com período $n$.
\item existe um ponto cuja órbita é densa.
\item o conjunto dos pontos periódicos é denso.
\item o conjunto dos pontos não periódicos que são eventualmente periódicos é denso.
\item o conjunto dos pontos que não são periódicos e nem eventualmente periódicos é denso.
\end{enumerate}
\end{proposition}

\begin{proof}
\begin{enumerate}
\item Se $s = (s_0s_1s_2\dots)$ é um ponto periódico com período $n$, então
$$\sigma^k(\sigma^n(s)) = (s_{n+k}s_{n+k+1}s_{n+k+2}\dots) = (s_ks_{k+1}s_{k+2}\dots) = \sigma^k(s)$$
para todo $k \geq 0$. Desse modo, $s$ é formado pela repetição das entradas $s_0 s_1 \dots s_{n-1}$. Pelo Princípio Fundamental da Contagem, existem $2^n$ sequências distintas para  $s_0 s_1 \dots s_{n-1}$ e, portanto, a afirmação está provada.

\item Considere o ponto $s^* = (0 \, 1 \, 0 0 \, 0 1 \, 1 0 \, 1 1 \, 0 0 0 \, 0 0 1 \dots)$ formado por todos os blocos de tamanho $1$, depois por todos os blocos de tamanho $2$, e assim sucessivamente. 

Sejam $s \in \Sigma_2$, $\varepsilon > 0$ e $n \geq 1$ tal que $\frac{1}{2^n} < \varepsilon$. É fácil ver que existe $k \geq 0$ de modo que $\sigma^k(s^*)$ e $s$ são iguais nas primeiras $n+1$ entradas. De acordo com a Proposição \ref{proposicao dinamicasimbolica 1}, $d(s, \sigma^k(s^*)) \leq \frac{1}{2^n} < \varepsilon$ e, portanto, a afirmação está provada.

\item Sejam $s = (s_0 s_1 s_2 \dots) \in \Sigma_2$, $\varepsilon > 0$ e $n \geq 1$ tal que $\frac{1}{2^n} < \varepsilon$. Considere o ponto periódico $t$ com período $n+1$ formado pela repetição da sequência $s_0 s_1 s_2 \dots s_n$. De acordo com a Proposição \ref{proposicao dinamicasimbolica 1}, $d(s, t) \leq \frac{1}{2^n} < \varepsilon$ e, portanto, a afirmação está provada.
\end{enumerate}
\end{proof}