\section{Função Logística III: Caos}

%Nessa seção estudaremos o conceito de caos. Seguindo a sequência da seção anterior, estudaremos o caso em que $\mu > 2 + \sqrt{5}$. Inicialmente, provaremos que o conjunto de pontos periódicos de $F$ é denso em $\Lambda$.

\begin{proposition}
\label{proposicao caos 1}
Se $\mu > 2 + \sqrt{5}$, então o conjunto de pontos periódicos de $F: \Lambda \to \Lambda$ é denso em $\Lambda$.
\end{proposition}

\begin{proof}
Sejam $x \in \Lambda$, $\varepsilon > 0$ e $k \geq 1$ tal que $\frac{1}{\lambda^k} < \varepsilon$. De acordo com o Lema \ref{lema conjuntosdecantor 1}, o intervalo fechado $I \subset \Lambda_k$ que contém $x$ possui tamanho menor que $\varepsilon$. Pela Proposição \ref{proposicao conjuntosdecantor 1}, $F^k: I \to [0, 1]$ é bijetora. Como $F^k(I) \supset I$, a Proposição \ref{proposicao diferenciabilidade 1} afirma que existe $y \in I$ tal que $F^k(y) = y$. Observando que $y \in \Lambda$ e $|x - y| < \varepsilon$, o resultado está provado.
\end{proof}

%No restante da seção, definiremos o que é uma função topologicamente transitiva, o que é uma função que depende sensivelmente das condições iniciais e, por fim, o que é uma função caótica. Vamos também mostrar $F$ possui cada uma dessas propriedades quando $\mu > 2 + \sqrt{5}$.

\begin{definition}
Seja $f: D \to D$ uma função. Dizemos que $f$ é \textit{topologicamente transitiva} se dados $x, y \in D$ e $\varepsilon > 0$,  existem $z \in D$ e $k \geq 1$ tais que $|z - x| < \varepsilon$ e $|f^k(z) - y| < \varepsilon$.
\end{definition} 

%Intuitivamente, $f$ é uma função topologicamente transitiva se para todo par de conjuntos abertos existe um ponto de um dos conjuntos que é levado para o outro conjunto após um número finito de iterações da $f$.

\begin{proposition}
\label{proposicao caos 2}
Se $\mu > 2 + \sqrt{5}$, então $F: \Lambda \to \Lambda$ é topologicamente transitiva.
\end{proposition}

\begin{proof}
Sejam $x, y \in \Lambda$ e $\varepsilon > 0$. Existe $k \geq 1$ tal que $\frac{1}{\lambda^k} < \varepsilon$. De acordo com o Lema \ref{lema conjuntosdecantor 1}, o tamanho de cada intervalo fechado em $\Lambda_k$ é menor que $\frac{1}{\lambda^k}$ e, portanto, menor que $\varepsilon$. Como $x \in \Lambda_ k$, existe um intervalo $[a, b] \subset \Lambda_ k$ que contém $x$. Pela Proposição \ref{proposicao conjuntosdecantor 1}, $F^k: [a, b] \to [0, 1]$ é bijetora e, pelo Teorema do Valor Intermediário, existe $z \in [a, b]$ tal que $F^k(z) = y$. Observando que $z \in \Lambda$, concluímos que $F$ é topologicamente transitiva.
\end{proof}

\begin{definition}
Seja $f: D \to D$ uma função. Dizemos que \textit{$f$ depende sensivelmente das condições iniciais} se para algum $\delta > 0$, dados $x \in D$ e $\varepsilon > 0$, existem $y \in D$ e $k \geq 1$ tais que $|x - y| < \varepsilon$ e $|f^k(x) - f^k(y)| > \delta$.
\end{definition}

\begin{proposition}
\label{proposicao caos 3}
Se $\mu > 2 + \sqrt{5}$, então $F: \Lambda \to \Lambda$ depende sensivelmente das condições iniciais.
\end{proposition}

\begin{proof}
Sejam $x \in \Lambda$, $\varepsilon > 0$ e $k \geq 1$ tal que $\frac{1}{\lambda^k} < \varepsilon$. Como na demonstração da Proposição anterior, seja $I$ o intervalo fechado contido em $\Lambda_k$ que contém $x$ e cujo tamanho é menor que $\varepsilon$. Como $F^k: I \to [0, 1]$ é um bijeção, então $F^k(a) = 0$ e  $F^k(b) = 1$, onde $a$ e $b$ são pontos extremos de $I$. Como $F(\frac{1}{2}) > 1$ e $x \in \Lambda$, segue que $F^{k}(x) \in \left[0, \frac{1}{2}\right) \cup \left(\frac{1}{2}, 1\right]$. Se $F^{k}(x) \in \left[0, \frac{1}{2}\right)$, então $|F^k(x) - F^k(b)| = |F^k(x) - 1| > \frac{1}{2}$ e se $F^{k}(x) \in \left(\frac{1}{2}, 1\right]$, então $|F^k(x) - F^k(a)| = |F^k(x)| > \frac{1}{2}$. Observando que $|x - a| < \varepsilon$ e  $|x - b| < \varepsilon$, temos o resultado para $\delta = \frac{1}{2}$. 
\end{proof}

\begin{definition}
Seja $f: D \to D$ uma função. Dizemos que \textit{$f$ é caótica} se
\begin{enumerate}
\item O conjunto de pontos periódicos de $f$ é denso em $D$.
\item $f$ é topologicamente transitiva.
\item $f$ depende sensivelmente das condições iniciais.
\end{enumerate}
\end{definition}

\begin{theorem}
\label{teorema caos 1}
Se $\mu > 2 + \sqrt{5}$, então $F: \Lambda \to \Lambda$ é caótica.
\end{theorem}

\begin{proof}
O resultado segue das Proposições \ref{proposicao caos 1}, \ref{proposicao caos 2} e \ref{proposicao caos 3}.
\end{proof}

\begin{remark}
O Teorema \ref{teorema caos 1} é válido para $4 < \mu \leq 2 + \sqrt{5}$, porém a demonstração é mais difícil.
\end{remark}

\begin{theorem}
\label{teorema caos 2}
Se $D$ é um subconjunto infinito de $\RR$ e $f: D \to D$ é uma função topologicamente transitiva cujo conjunto de pontos periódicos é denso, então $f$ é caótica.
\end{theorem}

\begin{proof}
Por demonstrar.
\end{proof}
