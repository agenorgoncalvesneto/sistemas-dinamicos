\section{Função Logística I: Estudo Inicial}

%Nosso objetivo será estudar, durante essa seção e as próximas, a dinâmica da família das funções logísticas, que são as funções $F_{\mu}: \RR \to \RR$ dadas por $F_{\mu}(x) = \mu x(1-x)$, onde $\mu > 0$. Quando não houver ambiguidade, escreveremos $F$ ao invés de $F_\mu$. Nessa seção estudaremos a dinâmica de $F$ quando $1 < \mu < 3$.

Durante essa seção e as próximas, estudaremos a dinâmica da função logística, que é dada por $F(x) = \mu x(1-x)$ para $\mu > 0$.

\begin{proposition}
Se $\mu > 1$, então
\begin{enumerate}
\item $F (1) = F(0) = 0$ e $F(\frac{1}{\mu}) = F(p_\mu) = p_\mu$, onde $p_\mu = \frac{\mu - 1}{\mu}$.
\item $0 < p_\mu < 1$.
\item o vértice da parábola de $F$ é o ponto $(\frac{1}{2}, \frac{\mu}{4})$.
\end{enumerate}
\end{proposition}

\begin{proof}
Aplicação direta das definições.
\end{proof}

\begin{proposition}
\label{proposicao estudoinicial 1}
Se $\mu > 1$, então $(-\infty, 0) \cup (1, \infty) \subset W^s(\infty)$.
\end{proposition}

\begin{proof}
Se $x < 0$, a sequência  $(x, F(x), F^2(x), \dots)$ é estritamente decrescente pois  $F(x) < x$. Se $(F^n(x))_n \to x_0$ quando $n \to \infty$, a continuidade de $F$ implica que $(F^{n+1}(x))_n \to F (x_0) < x_0$. Absurdo. Portanto, $(F^n(x))_n \to -\infty$ quando $n \to \infty$. Como $F(x) < 0$ para todo $x > 1$, concluímos que $(-\infty, 0) \cup (1, \infty) \subset W^s(\infty)$.
\end{proof}

%Pelo Proposição anterior, conhecemos a dinâmica da $F$, quando $\mu > 1$, nos pontos menores que zero e maiores que um. Portanto, nos resta estudar a dinâmica de $F$ restrita ao intervalo $[0, 1]$.

\begin{proposition}
Se $1 < \mu < 3$, então
\begin{enumerate}
\item $0$ é um ponto repulsor e $p_\mu$ é um ponto atrator.
\item $\limit_{n \to \infty} F^n(x) = p_\mu$ para todo $x \in (0, 1)$.
\end{enumerate}
\end{proposition}

\begin{proof}
A primeira parte é verdadeira pois $|F'(0)| = \mu > 1$ e $|F'(p_\mu)| = |2 - \mu| < 1$, quando $1 < \mu < 3$.

Falta provar o item 2.
%Para mostrar a segunda parte, dividiremos em casos. Suponha inicialmente que $1 < \mu < 2$. Como $F_\mu(x) \leq \frac{\mu}{4} < 1/2$ para todo $x \in (0, 1)$, basta analisar o limite quando $x \in (0, \frac{1}{2})$. Se $x \in (0, \frac{1}{2})$ então, analisando a função $F_\mu(x) - x$, temos que $|F_\mu(x) - p_\mu| < |x - p_\mu|$ e, pelo continuidade de $F_\mu$, $\lim_n F^n_\mu(x) = p_\mu$.
\end{proof}

Desse modo, conhecemos completamente a dinâmica de $F$ quando $1 < \mu < 3$: $$W^s(0) = \{0, 1\}\textrm{, } W^s(p_\mu) = (0, 1)\textrm{ e }W^s(\infty) = (-\infty, 0) \cup (1, \infty).$$