% !TeX root = ../sistemasdinamicos.tex
\section{Bifurcação}

Ao longo de seção, $f_\lambda$ representará uma família parametrizada de funções no parâmetro $\lambda$ de modo que a função
$$G(x, \lambda) = f_\lambda(x)$$
definida num aberto de $\RR^2$ seja de classe $\CC^\infty$ nas variáveis $x$ e $\lambda$.

\begin{theorem}[Função Implícita]
Sejam $U \subset \RR^2$ um aberto e $F: U \to \RR$ uma função de classe $\CC^k$, $1 \leq k \leq \infty$. Suponha que
\begin{enumerate}
\item $F(x_0, y_0) = c$
\item $\frac{\partial F}{\partial y} (x_0, y_0) \neq 0$
\end{enumerate} 
Então existem uma vizinhança $I$ de $x_0$ e uma função $f: I \to \RR$ de classe $\CC^k$ tais que
\begin{enumerate}
\item $f(x_0) = y_0$
\item $F(x, f(x)) = c$ para todo $x \in I$
\end{enumerate}
\end{theorem}

\begin{theorem}
\label{theorem1}
Seja $f_\lambda$ uma família parametrizada de funções. Suponha que \begin{enumerate}
\item $f_{\lambda_0}(x_0) = x_0$
\item $f'_{\lambda_0}(x_0) \neq 1$ 
\end{enumerate}
Então existem vizinhanças $I$ e $J$ de $\lambda_0$ e $x_0$, respectivamente, e uma função $p: I \to J$ de classe $\CC^\infty$ tais que
\begin{enumerate}
\item $p(\lambda_0) = x_0$ 
\item $f_\lambda(p(\lambda)) = p(\lambda)$ para todo $\lambda \in I$
\end{enumerate}
Além disso, $f_\lambda$ não possui outros pontos fixos em $J$.
\end{theorem}

\begin{proof}
Seja $G(x, \lambda) = f_\lambda(x) - x$. Observe que $x$ é ponto fixo de $f_\lambda$ se e somente se $G(x, \lambda) = 0$.

Pelo Teorema da Função Implícita, como $G(x_0, \lambda_0) = 0$ e
$$\frac{\partial G}{\partial x} (x_0, \lambda_0) = f'_{\lambda_0}(x_0) - 1 \neq 0,$$
existem vizinhanças $I$ e $J$ de $\lambda_0$ e $x_0$, respectivamente, e uma função $p: I \to J$ de classe $\CC^\infty$ tal que $p(\lambda_0) = x_0$ e $G(p(\lambda), \lambda) = 0$ para todo $\lambda \in I$.

Além disso, para cada $\lambda \in I$ está associado um único $x \in J$ e, portanto, $x \in J$ e $G(x, \lambda) = 0$ se e somente se $x = p(\lambda)$.
\end{proof}

De acordo com o Teorema anterior, se $x_0$ é um ponto fixo hiperbólico de $f_{\lambda_0}$, então $f_\lambda$ possui um único ponto fixo numa vizinhança de $x_0$ para cada $\lambda$ numa vizinhança de $\lambda_0$.
 
Utilizando a notação do Teorema anterior, considere a função $g_\lambda(x) = f_\lambda(x + p(\lambda)) - p(\lambda)$. Observe que $g_\lambda(0) = f(p(\lambda)) - p(\lambda) = 0$ para todo $\lambda \in I$, ou seja, $0$ é ponto fixo de $g_\lambda$ para todo $\lambda \in I$. Além disso, $f_\lambda$ e $g_\lambda$ são topologicamente conjugadas por $h_\lambda(x) = x - p(\lambda)$.

\begin{theorem}[Bifurcação Tangente]
Suponha que
\begin{enumerate}
\item $f_{\lambda_0}(0) = 0$
\item $f'_{\lambda_0}(0) = 1$
\item $f''_{\lambda_0}(0) \neq 0$
\item $\frac{\partial f_\lambda}{\partial \lambda} |_{\lambda = \lambda_0}(0) \neq 0$
\end{enumerate}
Então existem uma vizinhança $I$ de $0$ e uma função $p: I \to \RR$ de classe $\CC^\infty$ tais que
\begin{enumerate}
\item $p(0) = \lambda_0$
\item $f_{p(x)}(x) = x$
\end{enumerate}
Além disso, $p'(0) = 0$ e $p''(0) \neq 0$.
\end{theorem}

\begin{proof}
Considere a função $G(x, \lambda) = f_\lambda(x) - x$. Observe que $x$ é um ponto fixo de $f_\lambda$ se e somente se $G(x, \lambda) = 0$.

Pelo Teorema da Função Implícita, como $G(0, \lambda_0) = 0$ e 
$$\frac{\partial G}{\partial \lambda}(0, \lambda_0) = \frac{\partial f_\lambda}{\partial \lambda}|_{\lambda = \lambda_0}(0) \neq 0,$$
existem uma vizinhança $I$ de $0$ e uma função $p: I \to \RR$ tais que $p(0) = \lambda_0$ e $G(x, p(x)) = 0$ para todo $x \in I$.

Além disso, pela Regra da Cadeia, é válido que
$$p'(0) = - \frac{\frac{\partial G}{\partial x}(0, \lambda_0)}
{\frac{\partial G}{\partial \lambda}(0, \lambda_0)} = - \frac{f'_{\lambda_0}(0) - 1}{\frac{\partial f_\lambda}{\partial \lambda}|_{\lambda = \lambda_0}(0)} = 0$$
e
$$ p''(0) = - \frac{\frac{\partial^2 G}{\partial x^2}(0, \lambda_0)\frac{\partial G}{\partial \lambda}(0, \lambda_0) - \frac{\partial G}{\partial x}(0, \lambda_0) \frac{\partial^2 G}{ \partial x \partial \lambda}(0, \lambda_0)}
{\left( \frac{\partial G}{\partial \lambda}(0, \lambda_0) \right)^2}  = - \frac{\frac{\partial^2 G}{\partial x^2}(x, \lambda_0)}{ \frac{\partial f_\lambda}{\partial \lambda}|_{\lambda = \lambda_0}(0)} \neq 0 $$
\end{proof}

\begin{theorem}[Bifurcação com Duplicação de Período]
Suponha que
\begin{enumerate}
\item $f_{\lambda_0}(0) = 0$ para todo $\lambda$ numa vizinhança de $\lambda_0$
\item $f'_{\lambda_0}(0) = -1$
\item $\frac{\partial (f^2_\lambda)'}{\partial \lambda} |_{\lambda = \lambda_0}(0) \neq 0$
\item $S_{f_{\lambda_0}}(0) \neq 0$
\end{enumerate}
Então existem uma vizinhança $I$ de $0$ e uma função $p: I \to \RR$ de classe $\CC^\infty$ tais que 
\begin{enumerate}
\item $p(0) = \lambda_0$
\item $f_{p(x)}(x) \neq x$ para todo $x \in I$
\item $f^2_{p(x)}(x) = x$ para todo $x \in I$
\end{enumerate}
Além disso, $p'(0) = 0$ e $p''(0) \neq 0$.
\end{theorem}

\begin{proof}
Seja $G(x, \lambda) = f^2_\lambda (x) - x$. Sendo $G(0, \lambda) = 0$ para todo $\lambda$ numa vizinhança de $\lambda_0$, temos que
$$\frac{\partial G}{\partial \lambda}(0, \lambda_0) = 0$$
e, portanto, não podemos utilizar o Teorema da Função Implícita diretamente.
Seja
\[ H(x, \lambda) =
    \begin{cases} 
      \dfrac{G(x, \lambda)}{x} & \textrm{ se } x \neq 0 \\
      \\
      \dfrac{\partial G}{\partial x}(0, \lambda) & \textrm{ se } x = 0
   \end{cases}
\]
Desse modo, $H$ é de classe $\CC^\infty$ e são válidas as igualdades
\begin{enumerate}
\item[(I)]
$H(0, \lambda_0) = \frac{\partial G}{\partial x}(0, \lambda_0) = (f^2_{\lambda_0})'(0) - 1 = f'_{\lambda_0}(f_{\lambda_0}(0)) f'_{\lambda_0}(0) - 1 = 0 $

\item[(II)]
$\frac{\partial H}{\partial \lambda}(0, \lambda_0) = \frac{\partial}{\partial \lambda} \left( \frac{\partial G}{\partial x}(0, \lambda) \right) |_{\lambda = \lambda_0} = \frac{\partial}{\partial \lambda}((f^2_\lambda)'(0) - 1)|_{\lambda = \lambda_0} = \frac{\partial (f^2_\lambda)'}{\partial \lambda}|_{\lambda = \lambda_0}(0) \neq 0$
 
\item[(III)]
$ \frac{\partial H}{\partial x}(0, \lambda_0) = \frac{1}{2}\frac{\partial^2 G}{\partial x^2}(0, \lambda_0) $
\item[(IV)]
$\frac{\partial^2 H}{\partial x^2}(0, \lambda_0) = \frac{1}{3} \frac{\partial^3 
G}{\partial x^3}(0, \lambda_0)$
\end{enumerate}

Para provar as igualdades $(III)$ e $(IV)$, observe que podemos escrever

$$G(x,  \lambda_0) = G(0, \lambda_0) + x \frac{\partial G}{\partial x}(0, \lambda_0) + \frac{x^2}{2} \frac{\partial ^2 G}{\partial x^2} (0, \lambda_0) + \frac{x^3}{6} \frac{\partial ^3 G}{\partial x^3} (0, \lambda_0) + \cdots$$
para todo $x$ numa vizinhança de $0$. Desse modo, para $x \neq 0$ nessa vizinhança, podemos escrever
\begin{align*}
H(x, \lambda_0) & = \frac{\partial G}{\partial x}(0, \lambda_0) + \frac{x}{2} \frac{\partial ^2 G}{\partial x^2} (0, \lambda_0) + \frac{x^2}{6} \frac{\partial ^3 G}{\partial x^3} (0, \lambda_0) + \cdots \\
& = H(0, \lambda_0) + x \frac{\partial H}{\partial x}(0, \lambda_0) + \frac{x^2}{2} \frac{\partial ^2 H}{\partial x^2} (0, \lambda_0) + \cdots
\end{align*}
e, portanto, igualando as séries termo a termo concluímos que $\frac{\partial H}{\partial x}(0, \lambda_0) = \frac{1}{2} \frac{\partial ^2 G}{\partial x^2} (0, \lambda_0)$ e $\frac{1}{2} \frac{\partial ^2 H}{\partial x^2} (0, \lambda_0) = \frac{1}{6} \frac{\partial ^3 G}{\partial x^3} (0, \lambda_0)$

Pelas igualdades (I) e (II), e pelo Teorema da Função Implícita, existem uma vizinhança $I$ de $0$ e uma função $p: I \to \RR$ de classe $\CC^\infty$ tais que $p(0) = \lambda_0$ e $H(x, p(x)) = 0$ para todo $x \in I$. Em particular, se $x \neq 0$,
$$0 = \frac{G(x, p(x))}{x} = \frac{f^2_{p(x)}(x) - x}{x}$$
ou seja, $f^2_{p(x)}(x) = x$ para todo $x \in I$. Além disso, pelo Teorema \ref{theorem1}, $f_\lambda$ possui um único ponto fixo numa vizinhança de $0$ e, portanto, podemos considerar que $f_{p(x)}(x) \neq x$ para todo $x \in I$, $x \neq 0$.

Como
\begin{align*}
\frac{\partial^2 G}{\partial x^2}(0, \lambda_0) & = (f_{\lambda_0})''(x)|_{x = 0} \\
& = [f_{\lambda_0}'(f_{\lambda_0}(x)) f'_{\lambda_0}(x)]'|_{x = 0} \\
& = [f_{\lambda_0}''(f_{\lambda_0}(x))(f_{\lambda_0}'(x))^2 + f_{\lambda_0}'(f_{\lambda_0}(x))f_{\lambda_0}''(x)]|_{x = 0} \\
& = f_{\lambda_0}''(f_{\lambda_0}(0)) - f_{\lambda_0}''(0) = 0
\end{align*} temos que
$$p'(0) = -\frac{\frac{\partial H}{\partial x}(0, \lambda_0) }{\frac{\partial H }{\partial \lambda}(0, \lambda_0)} = -\frac{1}{2} \frac{\frac{\partial^2 G}{\partial x^2}(0, \lambda_0) }{\frac{\partial H }{\partial \lambda}(0, \lambda_0)} = 0$$

Por fim,
\begin{align*}
\frac{\partial^3 G}{\partial x^3}(0, \lambda_0) & = [f_{\lambda_0}''(f_{\lambda_0}(x))(f_{\lambda_0}'(x))^2
+ f_{\lambda_0}'(f_{\lambda_0}(x))f_{\lambda_0}''(x)]'|_{x = 0} \\
& = [f_{\lambda_0}'''(f_{\lambda_0}(x))(f_{\lambda_0}'(x))^3
+ 2f_{\lambda_0}''(f_{\lambda_0}(x))f_{\lambda_0}''(x)f_{\lambda_0}'(x)
+ f_{\lambda_0}''(f_{\lambda_0}(x)) f_{\lambda_0}'(x)f_{\lambda_0}''(x) \\
& \ \ \ \ + f_{\lambda_0}'(f_{\lambda_0}(x))f_{\lambda_0}'''(x)]|_{x = 0} \\
& = f_{\lambda_0}'''(0)(f_{\lambda_0}'(0))^3  + 2(f_{\lambda_0}''(0))^2f_{\lambda_0}'(0) + (f_{\lambda_0}''(0))^2f_{\lambda_0}'(0) + f_{\lambda_0}'(0)f_{\lambda_0}'''(0) \\
& = -2 f_{\lambda_0}'''(0) - 3(f_{\lambda_0}''(0))^2 \\
& = 2\frac{f_{\lambda_0}'''(0)}{f_{\lambda_0}'(0)} - 3\left( \frac{f_{\lambda_0}''(0)}{f_{\lambda_0}'(0)} \right)^2 = 2 S_{f_{\lambda_0}}(0)
\end{align*}
e, portanto,
$$ p''(0) = - \frac{\frac{\partial^2 H}{\partial x^2}(0, \lambda_0)\frac{\partial H}{\partial \lambda}(0, \lambda_0)}{ \left( \frac{ \partial H}{\partial \lambda}(0, \lambda_0) \right)^2} \\
= -\frac{1}{3} \frac{\frac{\partial^3 G}{\partial x^3}(0, \lambda_0)}{\frac{\partial H}{\partial \lambda}(0, \lambda_0)} = -\frac{2}{3} \frac{S_{f_{\lambda_0}}(0)}{\frac{\partial H}{\partial \lambda}(0, \lambda_0)} \neq 0.$$
\end{proof}