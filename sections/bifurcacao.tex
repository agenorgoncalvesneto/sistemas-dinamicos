% !TeX root = ../sistemasdinamicos.tex
\section{Bifurcação}

\begin{theorem}[Função Implícita]
Seja $F: \RR^2 \to \RR$ uma função de classe $\CC^\infty$. Suponha que
\begin{enumerate}
\item[i.] $F(x_0, y_0) = c$
\item[ii.] $\partial_y F(x_0, y_0) \neq 0$
\end{enumerate} 
Então existem uma vizinhança $I$ de $x_0$ e uma função $f: I \to \RR$ de classe $\CC^\infty$ tais que
\begin{enumerate}
\item $f(x_0) = y_0$
\item $F(x, f(x)) = c$ para todo $x \in I$
\item $f'(x) = -\frac{\partial_x F(x)}{\partial_y F(x)}$ para todo $x \in I$
\end{enumerate}
\end{theorem}

\begin{theorem}
\label{theorem1}
Seja $f_\lambda$ uma família parametrizada de funções. Suponha que $f_{\lambda_0}(x_0) = x_0$ e $f'_{\lambda_0}(x_0) \neq 1$. Então existem vizinhanças $I$ e $J$ de $\lambda_0$ e $x_0$, respectivamente, e uma função $p: I \to J$ de classe $\CC^\infty$ tal que $p(\lambda_0) = x_0$ e $f_\lambda(p(\lambda)) = p(\lambda)$ para todo $\lambda \in I$. Além disso, $f_\lambda$ não possui outros pontos fixos em $J$.
\end{theorem}

\begin{proof}
Seja $G(x, \lambda) = f_\lambda(x) - x$. Temos que $G(x_0, \lambda_0) = 0$ e $\partial_x G(x_0, \lambda_0) = f'_{\lambda_0}(x_0) - 1 \neq 0$. Utilizando o Teorema da Função Implícita, existem vizinhanças $I$ e $J$ de $\lambda_0$ e $x_0$, respectivamente, e uma função $p: I \to J$ de classe $\CC^\infty$ tal que $p(\lambda_0) = x_0$ e $G(p(\lambda), \lambda) = f_\lambda(p(\lambda)) - p(\lambda) = 0$ para todo $\lambda \in I$. Além disso, para $\lambda \in I$ está associado um único $x \in J$ e, portanto, $x \in J$ e $G(x, \lambda) = 0$ se e somente se $x = p(\lambda)$.
\end{proof}

De acordo com o Teorema anterior, se $x_0$ é um ponto fixo não hiperbólico de $f_{\lambda_0}$, então $f_\lambda$ possui um único ponto fixo próximo de $x_0$ para cada $\lambda$ numa vizinhança de $\lambda_0$.
 
Com a notação do Teorema anterior, considere a função $g_\lambda(x) = f_\lambda(x + p(\lambda)) - p(\lambda)$. Observe que $g_\lambda(0) = f(p(\lambda)) - p(\lambda) = 0$ para todo $\lambda \in I$, ou seja, $0$ é ponto fixo de $g_\lambda$ para todo $\lambda \in I$. Se $h_\lambda(x) = x - p(\lambda)$, então $g_\lambda \circ h_\lambda(x) = f_\lambda(x) - p(\lambda) = h_\lambda \circ f_\lambda(x)$, ou seja, $f_\lambda$ e $g_\lambda$ são topologicamente conjugadas.

\begin{theorem}
Suponha que
\begin{enumerate}
\item $f_{\lambda_0}(0) = 0$
\item $f'_{\lambda_0}(0) = 1$
\item $f''_{\lambda_0}(0) \neq 0$
\item $\partial_\lambda f_\lambda |_{\lambda = \lambda_0}(0) \neq 0$
\end{enumerate}
Então existe uma vizinhança $I$ de $0$ e uma função $p: I \to \RR$ de classe $\CC^\infty$ tal que $p(0) = \lambda_0$ e $f_{p(x)}(x) = x$. Além disso, $p'(0) = \lambda_0$ e $p''(0) \neq 0$.
\end{theorem}

\begin{proof}
Considere a função $G(x, \lambda) = f_\lambda(x) - x$. Observe que $x$ é um ponto fixo de $f_\lambda$ se $G(x, \lambda) = 0$.

Temos que $G(0, \lambda_0) = 0$ e 
$$\frac{\partial G}{\partial \lambda}(0, \lambda_0) = \frac{\partial f_\lambda}{\partial \lambda}(0)|_{\lambda = \lambda_0} \neq 0$$
Pelo Teorema da Função Implícita, existe uma vizinhança $I$ de $0$ e uma função $p: I \to \RR$ tal que $p(0) = \lambda_0$, $G(x, p(x)) = 0$ para todo $x \in I$ e
$$p'(x) = - \frac{\frac{\partial G}{\partial x}(x, p(x))}
{\frac{\partial G}{\partial \lambda}(x, p(x))}$$
e, portanto,
$$p'(0) = - \frac{f'_{\lambda_0}(0) - 1}{\frac{\partial f_\lambda}{\partial \lambda}(0)|_{\lambda = \lambda_0}} = 0$$
Além disso, utilizando a Regra da Cadeia,
$$p''(x) = - \frac{\frac{\partial^2 G}{\partial x^2}(x, p(x))\frac{\partial G}{\partial \lambda}(x, p(x)) - \frac{\partial G}{\partial x}(x, p(x)) \frac{\partial^2 G}{\partial \lambda \partial x}(x, p(x))}
{\left( \frac{\partial G}{\partial \lambda}(x, p(x)) \right)^2}$$
e, portanto,
$$p''(0) = - \frac{\frac{\partial^2 G}{\partial x^2}(x, \lambda_0) \frac{\partial G}{\partial \lambda}(0, \lambda_0)}{\left( \frac{\partial G}{\partial \lambda}(0, \lambda_0) \right)^2} = - \frac{\frac{\partial^2 G}{\partial x^2}(x, \lambda_0)}{ \frac{\partial f_\lambda}{\partial \lambda}(0)|_{\lambda = \lambda_0}} \neq 0$$
\end{proof}

\begin{theorem}
Suponha que
\begin{enumerate}
\item $f_{\lambda_0}(0) = 0$ para todo $\lambda$ numa vizinhança de $\lambda_0$
\item $f'_{\lambda_0}(0) = -1$
\item $\partial_\lambda (f^2_\lambda)' |_{\lambda = \lambda_0}(0) \neq 0$
\end{enumerate}
Então existe uma vizinhança $I$ de $0$ e uma função $p: I \to \RR$ de classe $\CC^\infty$ tal que $p(0) = \lambda_0$, $p'(0) = 0$,  $f_{p(x)}(x) \neq x$ e  $f^2_{p(x)}(x) = x$ para todo $x \in I$. Além disso, $p''(0) \neq 0$ se $S_{f_{\lambda_0}}(0) \neq 0$.
\end{theorem}

\begin{proof}
Seja $G(x, \lambda) = f^2_\lambda (x) - x$. Sendo $G(0, \lambda) = 0$ para todo $\lambda$ numa vizinhança de $\lambda_0$, temos que
$$\frac{\partial G}{\partial \lambda}(0, \lambda_0) = 0$$
e, portanto, não podemos utilizar o Teorema da Função Implícita diretamente.
Seja
\[ H(x, \lambda) =
    \begin{cases} 
      \dfrac{G(x, \lambda)}{x} & \textrm{ se } x \neq 0 \\
      \\
      \dfrac{\partial G}{\partial x}(0, \lambda) & \textrm{ se } x = 0
   \end{cases}
\]
Desse modo, $H$ é de classe $\CC^\infty$ e são válidas as igualdades
\begin{enumerate}
\item[(I)]
$H(0, \lambda_0) = \frac{\partial G}{\partial x}(0, \lambda_0) = (f^2_{\lambda_0})'(0) - 1 = f'_{\lambda_0}(f_{\lambda_0}0)) f'_{\lambda_0}(0) - 1 = 0 $

\item[(II)]
$\frac{\partial H}{\partial \lambda}(0, \lambda_0) =\frac{\partial}{\partial \lambda}((f^2_\lambda)'(0) - 1)|_{\lambda = \lambda_0} = \frac{\partial (f^2_\lambda)'(0)}{\partial \lambda}|_{\lambda = \lambda_0}$
 
\item[(III)]
$ \frac{\partial H}{\partial x}(0, \lambda_0) = \lim_{x \to 0} \frac{H(x, \lambda_0) - H(0, \lambda_0)}{x}
= \lim_{x \to 0} \frac{G(x, \lambda_0)}{x^2}
= \frac{1}{2} \lim_{x \to 0} \frac{\frac{\partial G}{\partial x}(x, \lambda_0)}{x} = \frac{1}{2}\frac{\partial^2 G}{\partial x^2}(0, \lambda_0) $
\item[(IV)]
$\frac{\partial^2 H}{\partial x^2}(0, \lambda_0) = \frac{1}{3} \frac{\partial^3 G}{\partial x^3}(0, \lambda_0)$
\end{enumerate}

Pelas igualdades (I) e (II), e pelo Teorema da Função Implícita, existe uma vizinhança $I$ de $0$ e uma função $p: I \to \RR$ de classe $\CC^\infty$ tal que $p(0) = \lambda_0$ e $H(x, p(x)) = 0$ para todo $x \in I$. Em particular, se $x \neq 0$,
$$0 = \frac{G(x, p(x))}{x} = \frac{f^2_{p(x)}(x) - x}{x}$$
ou seja, $f^2_{p(x)}(x) = x$ para todo $x \in I$. Além disso, pelo Teorema \ref{theorem1}, $f_\lambda$ possui apenas $1$ ponto fixo numa vizinhança de $0$ e, portanto, podemos considerar que $f_{p(x)}(x) \neq x$ para todo $x \in I$, $x \neq 0$.

Como
\begin{align*}
\frac{\partial^2 G}{\partial x^2}(0, \lambda_0) & = (f_{\lambda_0})''(x)|_{x = 0} \\
& = [f_{\lambda_0}'(f_{\lambda_0}(x)) f'_{\lambda_0}(x)]'|_{x = 0} \\
& = [f_{\lambda_0}''(f_{\lambda_0}(x))(f_{\lambda_0}'(x))^2 + f_{\lambda_0}'(f_{\lambda_0}(x))f_{\lambda_0}''(x)]_{x = 0} \\
& = f_{\lambda_0}''(0) - f_{\lambda_0}''(0) = 0
\end{align*} temos que
$$p'(0) = -\frac{\frac{\partial H}{\partial x}(0, \lambda_0) }{\frac{\partial H }{\partial \lambda}(0, \lambda_0)} = -\frac{\frac{1}{2} \frac{\partial^2 G}{\partial x^2}(0, \lambda_0) }{\frac{\partial H }{\partial \lambda}(0, \lambda_0)} = 0$$

Por fim,
\begin{align*}
\frac{\partial^3 G}{\partial x^3}(0, \lambda_0) & = \frac{\partial}{\partial x} \left(\frac{\partial^2 G}{\partial x^2}\right)(0, \lambda_0) \\
& = [f_{\lambda_0}''(f_{\lambda_0}(x))(f_{\lambda_0}'(x))^2
+ f_{\lambda_0}'(f_{\lambda_0}(x))f_{\lambda_0}''(x)]'_{x = 0} \\
& = [f_{\lambda_0}'''(f_{\lambda_0}(x))(f_{\lambda_0}'(x))^3
+ 2f_{\lambda_0}''(f_{\lambda_0}(x))f_{\lambda_0}''(x)f_{\lambda_0}'(x)
+ f_{\lambda_0}''(f_{\lambda_0}(x)) f_{\lambda_0}'(x)f_{\lambda_0}''(x) \\
+ f_{\lambda_0}'(f_{\lambda_0}(x))f_{\lambda_0}'''(x)]_{x = 0} \\
& = f_{\lambda_0}'''(0)(f_{\lambda_0}'(0))^3  + 2(f_{\lambda_0}''(0))^2f_{\lambda_0}'(0) + (f_{\lambda_0}''(0))^2f_{\lambda_0}'(0) + f_{\lambda_0}'(0)f_{\lambda_0}'''(0) \\
& = - 2f_{\lambda_0}'''(0) + 3(f_{\lambda_0}''(0))^2 \\
& = - 2\frac{f_{\lambda_0}'''(0)}{f_{\lambda_0}'(0)} +3\left( \frac{f_{\lambda_0}''(0)}{f_{\lambda_0}'(0)} \right)^2 = -2 S_{f_{\lambda_0}}(0)
\end{align*}
e, portanto,
$$ p''(0) = - \frac{\frac{\partial^2 H}{\partial x^2}(0, \lambda_0)\frac{\partial H}{\partial \lambda}(0, \lambda_0)}{ \left( \frac{ \partial H}{\partial \lambda}(0, \lambda_0) \right)^2} \\
= -\frac{1}{3} \frac{\frac{\partial^3 G}{\partial x^3}(0, \lambda_0)}{\frac{\partial H}{\partial \lambda}(0, \lambda_0)} = \frac{2}{3} \frac{S_{f_{\lambda_0}}(0)}{\frac{\partial H}{\partial \lambda}(0, \lambda_0)} \neq 0$$
quando $S_{f_{\lambda_0}}(0) \neq 0$.
\end{proof}