\section{Bifurcação}

\begin{theorem}[Função Implícita]
Seja $F: \RR^2 \to \RR$ uma função de classe $\CC^\infty$ e suponha que
\begin{enumerate}
\item $F(x_0, y_0) = c$
\item $\partial_y F(x_0, y_0) \neq 0$
\end{enumerate} 
Então existem uma vizinhança $I$ de $x_0$ e uma função $f: I \to \RR$ de classe $\CC^\infty$ tal que
\begin{enumerate}
\item $f(x_0) = y_0$
\item $F(x, f(x)) = c$ para todo $x \in I$
\item $f'(x) = -\frac{\partial_x F(x)}{\partial_y F(x)}$ para todo $x \in I$
\end{enumerate}
\end{theorem}

\begin{theorem}
Seja $f_\lambda$ uma família parametrizada de funções. Suponha que $f_{\lambda_0}(x_0) = x_0$ e $f'_{\lambda_0}(x_0) \neq 1$. Então existem vizinhanças $I$ e $J$ de $\lambda_0$ e $x_0$, respectivamente, e uma função $p: I \to J$ de classe $\CC^\infty$ tal que $p(\lambda_0) = x_0$ e $f_\lambda(p(\lambda)) = p(\lambda)$ para todo $\lambda \in I$. Além disso, $f_\lambda$ não possui outros pontos fixos em $J$.
\end{theorem}

\begin{proof}
Seja $G(x, \lambda) = f_\lambda(x) - x$. Temos que $G(x_0, \lambda_0) = 0$ e $\partial_x G(x_0, \lambda_0) = f'_{\lambda_0}(x_0) - 1 \neq 0$. Utilizando o Teorema da Função Implícita, existem vizinhanças $I$ e $J$ de $\lambda_0$ e $x_0$, respectivamente, e uma função $p: I \to J$ de classe $\CC^\infty$ tal que $p(\lambda_0) = x_0$ e $G(p(\lambda), \lambda) = f_\lambda(p(\lambda)) - p(\lambda) = 0$ para todo $\lambda \in I$. Além disso, $G(x, \lambda) \neq 0$ se $x \neq p(\lambda)$.
\end{proof}

Com a notação do Teorema Anterior, considere a função $g_\lambda(x) = f_\lambda(x + p(\lambda)) - p(\lambda)$. Observe que $g_\lambda(0) = f(p(\lambda)) - p(\lambda) = 0$ para todo $\lambda \in I$, ou seja, $0$ é ponto fixo de $g_\lambda$ para todo $\lambda \in I$. Se $h_\lambda(x) = x - p(\lambda)$, então $g_\lambda \circ h_\lambda(x) = f_\lambda(x) - p(\lambda) = h_\lambda \circ f_\lambda(x)$, ou seja, $f_\lambda$ e $g_\lambda$ são topologicamente conjugadas.

\begin{theorem}
Suponha que
\begin{enumerate}
\item $f_{\lambda_0}(0) = 0$
\item $f'_{\lambda_0}(0) = 1$
\item $f''_{\lambda_0}(0) \neq 0$
\item $\partial_\lambda f_\lambda |_{\lambda = \lambda_0}(0) \neq 0$
\end{enumerate}
Então existe uma vizinhança $I$ de $0$ e uma função $p: I \to \RR$ de classe $\CC^\infty$ tal que $p(0) = \lambda_0$ e $f_{p(x)}(x) = x$. Além disso, $p'(0) = \lambda_0$ e $p''(0) \neq 0$.
\end{theorem}

\begin{proof}

\end{proof}

\begin{theorem}
Suponha que
\begin{enumerate}
\item $f_{\lambda_0}(0) = 0$ para todo $\lambda$ numa vizinhança de $\lambda_0$
\item $f'_{\lambda_0}(0) = -1$
\item $\partial_\lambda (f^2_\lambda)' |_{\lambda = \lambda_0}(0) \neq 0$
\end{enumerate}
Então existe uma vizinhança $I$ de $0$ e uma função $p: I \to \RR$ de classe $\CC^\infty$ tal que $f_{p(x)}(x) \neq x$ e  $f^2_{p(x)}(x) = x$.
\end{theorem}

\begin{proof}

\end{proof}