\section{Subshift e Matriz de Transição}

Se $N \geq 1$, definimos o conjunto $\Sigma_N$ formado sequências de números naturais limitados entre $1$ e $N$. Precisamente,

$$\Sigma_N = \{ (x_n)_n \in \NN^{\NN} : 1 \leq x_n \leq N \}.$$

Definimos também a função $d_N : \Sigma_N \times \Sigma_N \to \RR$ dada por
$$d_N(x, y) = \sum_{i=0}^\infty \frac{|x_i - y_i|}{N^i},$$
onde $x = (x_n)_n$ e $y = (y_n)_n$. Observe que 


\begin{proposition}
$(\Sigma_N, d_N)$ é um espaço métrico.
\end{proposition}


\begin{proof}

\end{proof}


\begin{proposition}
Sejam $x = (x_n)_n, y = (y_n)_n \in \Sigma_N$.
\begin{enumerate}
\item Se $x_i = y_i$ para todo $0 \leq i \leq k$, então $d_N(x, y) \leq \frac{1}{N^k}$.
\item Se $d_N(x, y) < \frac{1}{N^k}$, então $x_i = y_i$ para todo $0 \leq i \leq k$.
\end{enumerate}
\end{proposition}


\begin{proof}

\end{proof}


\begin{definition}
Seja $A = (a_{ij})_{1 \leq i,j \leq N}$ uma matriz quadrada de ordem $N$. Dizemos que $A$ é uma matriz de transição de ordem $N$ se $a_{ij} \in \{ 0, 1 \}$ para todo $1 \leq i,j \leq N$.
\end{definition}

Seja $A$ uma matriz de transição de ordem $N$. Definimos o conjunto $\Sigma_A$ como
$$\Sigma_A = \{ (x_n)_n \in \Sigma_N : a_{x_i x_{i+1}} = 1 \textrm{ para todo } i \geq 0 \}.$$

\begin{proposition}
$\Sigma_A$ é fechado em $(\Sigma_N, d_N)$.
\end{proposition}


\begin{proof}

\end{proof}


\begin{proposition}
Se $x \notin \{ 0, a_1, a_2, a_3 \}$ é um ponto periódico de $F$, então $x \in I_1 \cup I_2$.
\end{proposition}


\begin{proof}

\end{proof}


\begin{lemma}
$\Lambda$ é um conjunto hiperbólico.
\end{lemma}


\begin{proof}

\end{proof}


\begin{theorem}
$F|_\Lambda$ e $\sigma_A$ são topologicamente conjugadas.
\end{theorem}


\begin{proof}

\end{proof}


\begin{proposition}
Seja $A$ uma matriz de transição de ordem $N$. Então $\sigma_A$ possui $Tr(A^k)$ pontos periódicos de período $k$.
\end{proposition}


\begin{proof}

\end{proof}











