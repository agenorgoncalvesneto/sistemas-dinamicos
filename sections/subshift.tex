\section{Subshift e Matriz de Transição}

\begin{definition}
$\Sigma_N = \{ x = (x_n)_{n \in \NN}: 1 \leq x_n \leq N \}$
\end{definition}

\begin{proposition}
$(\Sigma_N, d_N)$ é um espaço métrico.
\end{proposition}

\begin{proposition}
Se $x_i = y_i$ para todo $0 \leq i \leq k$, então $d_N(x, y) \leq \frac{1}{N^k}$. Se $d_N(x, y) < \frac{1}{N^k}$, então $x_i = y_i$ para todo $0 \leq i \leq k$.
\end{proposition}

\begin{definition}
Seja $A = (a_{ij})_{1 \leq i,j \leq N}$ uma matriz quadrada de ordem $N$ tal que $a_{ij} \in \{ 0, 1 \}$ para todo $1 \leq i,j \leq N$. Então
$$\Sigma_A = \{ x = (x_n) \in \Sigma_N : a_{x_i x_{i+1}} = 1\}$$
\end{definition}

\begin{proposition}
$\Sigma_A$ é fechado em $(\Sigma_N, d_N)$.
\end{proposition}

\begin{proposition}
Se $x$ é um ponto periódico de $F$ diferente de $0, a_1, a_2, a_3$, então $x \in I_1 \cup I_2$.
\end{proposition}

\begin{lemma}
$\Lambda$ é um conjunto hiperbólico.
\end{lemma}

\begin{theorem}
$F_\Lambda$ e $\sigma_A$ são topologicamente conjugadas.
\end{theorem}

\begin{proposition}
Seja $A$ uma matriz de transição de ordem $N$. Então $\sigma_A$ possui $Tr(A^k)$ pontos periódicos de período $k$.
\end{proposition}














