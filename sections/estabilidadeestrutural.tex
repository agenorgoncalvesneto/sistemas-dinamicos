\section{Estabilidade Estrutural}

\begin{definition}
Sejam $f, g: D \to \RR$ funções de classe $\CC^k$. A $k-$distância entre $f$ e $g$ é definida por
$$d_k(f, g) = \sup_{x \in D} \left \{ |f(x) - g(x)|, |f^{(2)}(x), g^{(2)}(x)|, \dots, |f^{(k)}(x) - g^{(k)}(x)| \right \}$$
\end{definition}

\begin{definition}
Seja $f: D \to \RR$. Dizemos que $f$ é $k-$estável se existe $\varepsilon > 0$ tal que $d_k(f, g) < \varepsilon$ implica que $f$ e $g$ são topologicamente conjugadas.
\end{definition}

\begin{example}
Seja $L: \RR \to \RR$, dada por $L(x) = \frac{x}{2}$. Vamos provar que $d_1(L, g) < \frac{1}{2}$ implica que $L$ e $g$ são topologicamente conjugadas.

Inicialmente, $g$ possui pelo menos 1 ponto fixo. Como $\left| \frac{x}{2} - g(x) \right| < \frac{1}{2}$ para todo $x \in \RR$, temos que $-\frac{1}{2} < \frac{x}{2} - g(x) < \frac{1}{2}$ e, portanto, $-\frac{1}{2} - \frac{x}{2} < g(x) - x < \frac{1}{2} - \frac{x}{2}$. Definindo $h(x) = g(x) - x$, temos que $0 < h(-1) < 1$ e $-1 < h(1) < 0$. Pelo Teorema do Valor Intermediário, existe $x_0 \in (-1, 1)$ tal que $h(x_0) = 0$ e, portanto, $g(x_0) = x_0$.

Além disso, $g$ possui no máximo 1 ponto fixo. Como $\left| \frac{1}{2} - g'(x) \right| < \frac{1}{2}$ para todo $x \in \RR$, temos que $0 < g'(x) < 1$. De acordo com o Teorema do Valor Médio, se $g$ possui $2$ pontos fixos, então existe $x_0$ tal que $g'(x_0) = 1$, o que é um absurdo.

Seja $J = [-10, -5) \cup (5, 10]$. Observe que se $x \in \RR - \{0\}$, então existe um único $n_x \in \ZZ$ tal que $L^{n_x}(x) \in J$. Analogamente, se $x \in \RR$ e $x$ não é ponto fixo de $g$, então existe um único $n_x$ tal que $g^{n_x}(x) \in [-10, g(-10)) \cup (g(10), 10]$.

Seja $h$ uma função tal que $h|_{[-10, -5]}$ é um homeomorfismo crescente entre $[-10, -5]$ e $[-10, g(-10)]$ e $h|_{[5, 10]}$ é um homeomorfismo crescente entre $[5, 10]$ e $[g(10), 10]$.

Seja $x \in \RR - \{0\}$. Como $L^{n_x}(x) \in J$, temos que $h \circ L^{n_x}(x)$ está bem definido. Sendo $g$ um homeomorfismo, $g^{-n_x} \circ h \circ L^{n_x}(x)$ também está bem definido. Defina $h(x) = g^{-n_x} \circ h \circ L^{n_x}(x)$ para todo $x \in \RR - \{0\}$. Observe que se $x \in J$, então $n_x = 0$ , portanto, está bem definida em $J$. Por fim, defina $h(0)$ como sendo o ponto fixo de $g$. Resta mostrar que $h \circ L(x) = g \circ h(x)$ para todo $x \in \RR$.

Se $x \neq 0$, então $h(x) = g^{-n_x} \circ h \circ L^{n_x}(x)$. Se $y = L(x)$, então $y \neq 0$ e $L^{n_x - 1}(y) = L^{n_x - 1}(L(x)) = L^{n_x}(x) \in J$, ou seja, $n_y = n_x - 1$. Desse modo,
$$h \circ L(x) = h(y) = g^{-n_y} \circ h \circ L^{n_y}(y) = g \circ g^{-n_x} \circ h \circ L^{n_x}(x) = g \circ h(x)$$
e $g(h(0)) = h(0) = h(L(0))$.

Assim, $h \circ L = g \circ h$ e $h$ é um homeomorfismo pois é composição de homeomorfismos. Desse modo, $L$ e $g$ são topologicamente conjugadas e, portanto, $L$ é $\CC^1-$estável.
\end{example}

\begin{theorem}
Se $\mu > 2 + \sqrt{5}$, então $F_\mu$ é $\CC^2-$estável.
\end{theorem}

\begin{proof}
Vamos mostrar que existe $\varepsilon > 0$ tal que $d_2(F_\mu, g) < \varepsilon$ implica que $F_\mu$ e $g$ são topologicamente conjugadas.

Seja $\varepsilon_1 > 0$ tal que $d_2(F_\mu, g) < \varepsilon_1$ implica que $g'' < 0$ e, portanto, a concavidade de $g$ é para baixo. Existe $\varepsilon_1$ com essa propriedade pois $F''_\mu = -2\mu$. 

Seja $0 < \varepsilon_2 < \varepsilon_1$ tal que $d_2(F_\mu, g) < \varepsilon_2$ implica que $g$ possui dois pontos fixos $\alpha < \beta$ com $g'(\alpha) > 1$ e $g'(\beta) < -1$. Existe $\varepsilon_2$ com essa propriedade pois $F_\mu$ possui os pontos fixos $0, 1$ com $F'_\mu(0) > 1$ e $F'_\mu(1) < -1$.

Pelo Teorema do Valor Médio, temos que $g$ possui um ponto crítico $c \in (\alpha, \beta)$. Sendo $g'' < 0$, o ponto crítico de $g$ é único. Além disso, $g$ é estritamente crescente em $(-\infty, c)$ e estritamente decrescente em $(c, \infty)$. Desse modo, existem $\alpha'$ e $\beta'$ tais que $g(\alpha') = \alpha$ e $g(\beta') = \beta$.

Por fim, seja $0 < \varepsilon < \varepsilon_2$ tal que $d_2(F_\mu, g) < \varepsilon$ implica que $g^{-1}(\alpha')$ possui dois elementos $a_0$ e $a_1$ e que $|g'(x)| > 1$ para todo $x \in [\alpha, a_0] \cup [a_1, \beta]$.

Desse modo, se $d_2(F_\mu, g) < \varepsilon$, então $g$ possui as mesmas propriedades de $F_\mu$. Em particular, utilizando as imagens inversas de $(a_0, a_1)$ por $g$, é possível provar que $g$ e $\sigma$ são topologicamente conjugadas. Portanto, $F_\mu$ e $g$ são topologicamente conjugadas por transitividade.
\end{proof}
















