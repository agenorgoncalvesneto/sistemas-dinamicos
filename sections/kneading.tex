\section{Teoria Kneading}

\begin{definition}
A função $f: I \to I$ é unimodal se
\begin{enumerate}
\item $f(0) = f(1) = 0$.
\item $f$ possui um único ponto crítico em $(0, 1)$.
\end{enumerate}
\end{definition}

\begin{definition}
O itinerário de $x \in I$ é  a sequência $S(x) = (s_0, s_1, s_2, \dots)$, onde

\[ s_k = 
\begin{cases} 
  0 & \textrm{ se } f^k(x) < c \\
  1 & \textrm{ se } f^k(x) > c \\
  C & \textrm{ se } f^k(x) = c \\
\end{cases}
\]
\end{definition}

\begin{definition}
A sequência kneading de $f$ é a sequência $K(f) = S(f(c))$, ou seja, é o itinerário de $f(c)$.
\end{definition}

Sejam $s = (s_0, s_1, s_2, \dots)$ e $t = (t_0, t_1, t_2, \dots)$ duas sequências. Dizemos que $s$ e $t$ são $n$-discrepantes se $s_i = t_i$ para todo $0 \leq i < n$ e $s_n \neq t_n$. Definimos $\tau_n(s)$ como a quantidade de elementos em $\{ s_0, \dots, s_{n-1} \}$ que são iguais à $1$.

\begin{definition}
Suponha que $s$ e $t$ são $n$-discrepantes. Dizemos que $s \prec t$ se
\begin{enumerate}
\item $n = 0$ e $s_0 < t_0$.
\item $\tau_{n-1}(s)$ é par e $s_n < t_n$.
\item $\tau_{n-1}(s)$ é ímpar e $s_n > t_n$.
\end{enumerate}
\end{definition}

\begin{theorem}
Sejam $x, y \in I$.
\begin{enumerate}
\item Se $S(x) \prec S(y)$, então $x < y$.
\item Se $x < y$, então $S(x) \preceq S(y)$.
\end{enumerate}
\end{theorem}

\begin{proof}
O primeiro item será provado por indução em $n$, onde $S(x)$ e $S(y)$ são $n$-discrepantes e, por contra-positiva, o segundo item segue imediatamente do primeiro. Seja $S(x) = (s_0, s_1, s_2, \dots)$ e $S(y) = (t_0, t_1, t_2, \dots)$.

Se $S(x)$ e $S(y)$ são $0$\,-\,discrepantes e $S(x) \prec S(y)$, então $x < y$. Suponha que essa propriedade é válida quando as sequências são $(n-1)$\,-\,discrepantes.

Se $S(x)$ e $S(y)$ são $n$\,-\,discrepantes, $n \geq 1$, então as sequências
$$S(f(x)) = (s_1, s_2, s_3, \dots) \textrm{ e } S(f(y)) = (t_1, t_2, t_3, \dots)$$
são $(n-1)$\,-\,discrepantes. Se $s_0 = t_0 = 0$, então $S(f(x)) \prec S(f(y))$ pois a quantidade de elementos iguais à $1$ antes da discrepância permanece inalterada e portanto, $f(x) < f(y)$ por hipótese de indução. Sendo $f$ estritamente crescente em $[0, c)$, concluímos que $x < y$. Analogamente, se $s_0 = t_0 = 1$, então $S(f(x)) \succ S(f(y))$ pois a quantidade de elementos iguais à $1$ antes da discrepância é diminuída em uma unidade e, portanto, $f(x) > f(y)$ por hipótese de indução. Sendo $f$ estritamente decrescente em $(c, 1]$, concluímos que $x > y$. Por fim, se $s_0 = t_0 = C$, então $x = y = c$.
\end{proof}

\begin{lemma}
Seja $s = (s_0, s_1, s_2, \dots)$ e suponha que $s_i \neq C$ para todo $0 \leq i \leq n$. Então, o conjunto
$$\{ x \in I : S(x)_i = s_i \textrm{ para todo } 0 \leq i \leq n \}$$
é um aberto em $I$.
\end{lemma}

\begin{proof}
Seja $x \in I$ tal que $S(x)_i = s_i$ para todo $0 \leq i \leq n$ e, portanto, $f^i(x) \in [0,c) \cup (c, 1]$ para todo $0 \leq i \leq n$. Para cada $0 \leq i \leq n$, seja
\[ V_i = 
\begin{cases} 
  [0, c) & \textrm{ se } f^i(x) < c \\
  (c, 1] & \textrm{ se } f^i(x) > c \\
\end{cases}
\]
Observe que cada $V_i$ é aberto em $I$. Pela continuidade de $f^i$, cada $f^{-i}(V_i)$ é aberto em $I$. Definindo $V = \cap_{i=0}^n f^{-i}(V_i)$, temos que $V$ é aberto e se $y \in V$, então $S(y)_i = s_i$ para todo $0 \leq i \leq n$.  
\end{proof}

\begin{theorem}
Suponha que $f: I \to I$ é unimodal e $c$ não é periódico. Se $t$ é uma sequência tal que $\sigma^n(t) \prec K(f)$ para todo $n \geq 1$, então $t$ é $f$-admissível.
\end{theorem}