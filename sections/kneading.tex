\section{Teoria Kneading}

\begin{definition}
Seja $f: [0, 1] \to [0, 1]$ uma função de classe $\CC^1([0, 1])$. $f$ é unimodal se
\begin{enumerate}
\item $f(0) = f(1) = 0$.
\item $f$ possui um único ponto crítico em $(0, 1)$.
\end{enumerate}
\end{definition}

No restante dessa seção, estudaremos uma função $f$ unimodal cujo ponto crítico em $(0, 1)$ será denotado por $c$. Além disso, fixaremos um símbolo $C$ de modo que o conjunto $\{0, C, 1\}$ seja ordenado pelas relações $0 < C$, $C < 1$ e $0 < 1$.
%Assim, $f$ é estritamente crescente em $[0, c)$ e estritamente decrescente em $(c, 1]$.

\begin{definition}
Seja $x \in [0, 1]$. O itinerário de $x$ por $f$ é a sequência infinita $S(x) = (s_0 \, s_1 \, s_2 \, \dots)$, onde

\[ s_j = 
\begin{cases} 
  0 & \text{ se } f^j(x) < c \\
  C & \text{ se } f^j(x) = c \\
  1 & \text{ se } f^j(x) > c. \\
\end{cases}
\]
\end{definition}

\begin{definition}
A sequência kneading de $f$ é a sequência $K(f) = S(f(c))$, ou seja, é o itinerário de $f(c)$.
\end{definition}

Seja $$\Sigma_C = \{ (s_0\, s_1\, s_2\, \dots) : s_n \in \{0, C, 1\} \text{ para todo } n \geq 0\}.$$ Se $s = (s_0\, s_1\, s_2\, \dots)$ e $t = (t_0\, t_1\, t_2\, \dots)$ são elementos de $\Sigma_C$, dizemos que $s$ e $t$ possuem discrepância $n$ quando $s_i = t_i$ para todo $0 \leq i < n$ e $s_n \neq t_n$. Além disso, definimos $\tau_n(s)$ como a cardinalidade do conjunto $\{ s_j : 0 \leq j < n \text{ e } s_j = 1 \}$. Com isso, podemos definir a seguinte ordem em $\Sigma_C$.

\begin{definition}
Sejam $s = (s_0\, s_1\, s_2\, \dots)$ e $t = (t_0\, t_1\, t_2\, \dots)$ elementos de $\Sigma_C$. Suponha que $s$ e $t$ possuem discrepância $n$. Dizemos que $s \prec t$ se alguma das condições abaixo é válida.
\begin{enumerate}
\item $\tau_{n-1}(s)$ é par e $s_n < t_n$.
\item $\tau_{n-1}(s)$ é ímpar e $s_n > t_n$.
\end{enumerate}
\end{definition}

O seguinte teorema nos mostra as semelhanças da ordem $\prec$ em $\Sigma_C$ com a ordem usual na reta real.

\begin{theorem}
\label{teo1}
Sejam $x, y \in [0, 1]$.
\begin{enumerate}
\item Se $S(x) \prec S(y)$, então $x < y$.
\item Se $x < y$, então $S(x) \preceq S(y)$.
\end{enumerate}
\end{theorem}

\begin{proof}
O primeiro item será provado por indução em $n$, onde $S(x)$ e $S(y)$ são possuem discrepância $n$ e, por contrapositiva, o segundo item segue imediatamente do primeiro.

Sejam $S(x) = (s_0\, s_1\, s_2\, \dots)$ e $S(y) = (t_0\, t_1\, t_2\, \dots)$ tais que $S(x) \prec S(y)$. Se $S(x)$ e $S(y)$ possuem discrepância $0$, então $s_0 < t_0$ e, portanto, $x < y$. Suponha que essa propriedade é válida quando as sequências possuem discrepância $n-1$.

Se $S(x)$ e $S(y)$ possuem discrepância $n \geq 1$, então $s_0 = t_0$ e as sequências 
$$S(f(x)) = (s_1\, s_2\, s_3\, \dots) \textrm{\, e \,} S(f(y)) = (t_1\, t_2\, t_3\, \dots)$$
possuem discrepância $n-1$. Se $s_0 = 0$, então $S(f(x)) \prec S(f(y))$ pois a quantidade de elementos iguais à $1$ antes da discrepância permanece inalterada e, portanto, $f(x) < f(y)$ por hipótese de indução. Sendo $f$ estritamente crescente em $[0, c)$, concluímos que $x < y$. Se $s_0 = 1$, então $S(f(y)) \prec S(f(x))$ pois a quantidade de elementos iguais à $1$ antes da discrepância é diminuída em uma unidade e, portanto, $f(y) < f(x)$ por hipótese de indução. Sendo $f$ estritamente decrescente em $(c, 1]$, concluímos que $x < y$. Por fim, se $s_0 = C$, então $x = y = c$.
\end{proof}

Utilizaremos, nos resultados a seguir, o conceito de abertos em $[0, 1]$. Encarando $[0, 1]$ como um subespaço topológico de $\RR$, os conjuntos da forma $A \cap [0, 1]$ são abertos em $[0, 1]$ quando $A$ é aberto em $\RR$. Desse modo, $[0, c)$ e $(c, 1]$ são abertos em $[0, 1]$ por exemplo.

\begin{lemma}
Seja $s = (s_0\, s_1\, s_2\, \dots)$ um elemento de $\Sigma_C$. Suponha que $s_i \neq C$ para todo $0 \leq i \leq n$. Então, o conjunto
$$\{ x \in [0, 1] : s_i = t_i \textrm{ para todo } 0 \leq i \leq n, \text{ onde } S(x) = (t_0\, t_1\, t_2\, \dots) \}$$
é aberto em $[0, 1]$.
\end{lemma}

\begin{proof}
Seja $x \in [0, 1]$ tal que $s_i = t_i$ para todo $0 \leq i \leq n$, onde $S(x) = (t_0\, t_1\, t_2\, \dots)$. Assim, $f^i(x) \neq c$ para todo $0 \leq i \leq n$ e, portanto, podemos definir
\[ V_i = 
\begin{cases} 
  [0, c) & \textrm{ se } f^i(x) < c \\
  (c, 1] & \textrm{ se } f^i(x) > c. \\
\end{cases}
\]
Sendo cada $V_i$ é aberto em $[0, 1]$, a continuidade de $f^i$ implica que $(f^i)^{-1}(V_i)$ é aberto em $[0, 1]$. Definindo $V = \cap_{i=0}^n (f^i)^{-1}(V_i)$, temos que $V$ é aberto em $[0, 1]$ que contém $x$ e se $y \in V$, então $s_i = t_i$ para todo $0 \leq i \leq n$, onde $S(y) = (t_0\, t_1\, t_2\, \dots)$.  
\end{proof}

Seja $x \in I$ tal que $S(x) = (s_0\, s_1\, s_2\, \dots)$. Se $\sigma$ é a função shift em $\Sigma_C$, então $S(f^k(x)) = (s_k\, s_{k+1}\, s_{k+2}\, \dots) = \sigma^k(S(x))$ para todo $k \geq 0$.

Para prosseguir os estudos, é interessante restringir a atenção nos elementos de $\Sigma_C$ que são itinerários de algum $x \in [0, 1]$. Desse modo, definimos
$$\Sigma_{C, f} = \{ s \in \Sigma_C : S(x) = s \text{ para algum } x \in [0, 1] \}$$
e dizemos que os elementos de $\Sigma_{C, f}$ são admissíveis por $f$.

Suponha que $s = (s_0\, s_1\, s_2\, \dots)$ é admissível por $f$. Então, $S(x) = s$ para algum $x \in [0, 1]$. Sendo $f$ estritamente crescente em $[0, c)$ e estritamente decrescente em $(c, 1]$, temos que $f^n(x) \leq f(c)$ para todo $n \geq 1$. Pelo Teorema \ref{teo1}, temos que $\sigma^n(s) = S(f^n(x)) \preceq S(f(c)) = K(f)$ para todo $n \geq 1$. Desse modo, temos uma condição necessária para que uma sequência seja admissível por $f$.

\begin{theorem}
Seja $f: [0, 1] \to [0, 1]$ uma função unimodal. Suponha que $c$ não é periódico. Se $t$ é um elemento de $\Sigma_C$ tal que $\sigma^n(t) \prec K(f)$ para todo $n \geq 1$, então $t$ é admissível por $f$.
\end{theorem}

\begin{proof}
Se $t = (0\, 0\, 0\, \dots)$ ou $t = (1\, 0\, 0\, \dots)$, então $t$ é admissível por $f$, pois nesse caso $S(0) = t$ ou $S(1) = t$. Para mostrar os outros casos, denote $t = (t_0\, t_1\, t_2\, \dots)$ e considere os conjuntos 
$$L = \{ x \in [0, 1] : S(x) \prec t \} \text{\, e \,} R = \{ x \in [0, 1] : S(x) \succ t \}.$$
Vamos mostrar que $L$ é aberto em $[0, 1]$. Uma prova análoga pode ser feita para mostrar que $R$ é aberto em $[0, 1]$.

Seja $z \in L$ e denote $S(z) = s = (s_0 \, s_1 \, s_2 \, \dots)$. Como $s \prec t$, temos que $s \neq t$ e, portanto, $s$ e $t$ possuem discrepância $n$ para algum $n \geq 0$. Sendo $t_n \neq C$ para todo $n \geq 0$, temos que $t_n = 0$ ou $t_n = 1$. Vamos supor que $t_n = 1$. Uma prova análoga segue quando $t_n = 0$. Como $s_n \neq t_n$, temos que $s_n = 0$ ou $s_n = C$. 

Se $s_n = 0$, então $s_k \neq C$ para todo $0 \leq k \leq n$ e, pelo Lema anterior, existe uma vizinhança aberta de $z$ em $L$. Se $s_n = C$, então $K(f) = (s_{n+1}\, s_{n+2}\, s_{n+3}\, \dots)$. Observe que existe $\alpha > 0$ tal que $s_{n+\alpha} \neq t_{n+\alpha}$ pois, caso contrário, $\sigma^{n+1}(t) = (t_{n+1}\, t_{n+2}\, t_{n+3}\, \dots) = (s_{n+1}\, s_{n+2}\, s_{n+3}\, \dots) = K(f)$. Além disso, $s_{n+i} \neq C$ para todo $i > 0$ pois $c$ não é periódico. Seja $W$ a vizinhança de $z$ tal que, se $x \in W$, então
$$S(x) = (s_0\, \dots\, s_{n-1}\, *\, s_{n+1}\, \dots\, s_{n+\alpha}\, \dots),$$
onde $*$ é $0$, $1$ ou $C$. Então, $W$ é uma vizinhança aberta de $z$ em $L$. Além disso,
$$(s_0\, \dots\, s_{n-1}\, *\, \dots) \prec (s_0\, \dots\, s_{n-1}\, t_n, \dots) = (t_0\, \dots\, t_{n-1}\, t_n, \dots)$$
pois $s \prec t$ e, portanto, $S(x) \prec t$ para todo $x \in W$.

Assim, $L$ e $R$ são abertos em $[0, 1]$. Lembrando que $t \neq (0\, 0\, 0\, \dots)$ e $t \neq (1\, 0\, 0\, \dots)$, temos também que $L$ e $R$ não são vazios. Por fim, sendo $[0, 1]$ conexo, concluímos que existe um fechado não vazio em $[0, 1]$ cujos elementos possuem itinerário exatamente igual à $t$.
\end{proof}












