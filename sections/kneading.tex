\section{Teoria Kneading}

\begin{definition}
Seja $f: [0, 1] \to [0, 1]$ uma função de classe $\CC^1([0, 1])$. $f$ é unimodal se
\begin{enumerate}
\item $f(0) = f(1) = 0$.
\item $f$ possui um único ponto crítico em $(0, 1)$.
\end{enumerate}
\end{definition}

No restante dessa seção, estudaremos uma função $f$ unimodal cujo ponto crítico em $(0, 1)$ será denotado por $c$. Além disso, fixaremos um símbolo $C$ de modo que o conjunto $\{0, C, 1\}$ seja ordenado pelas relações $0 < C$, $C < 1$ e $0 < 1$.
%Assim, $f$ é estritamente crescente em $[0, c)$ e estritamente decrescente em $(c, 1]$.

\begin{definition}
Seja $x \in [0, 1]$. O itinerário de $x$ por $f$ é a sequência infinita $S(x) = (s_0 \, s_1 \, s_2 \, \dots)$, onde

\[ s_j = 
\begin{cases} 
  0 & \text{ se } f^j(x) < c \\
  C & \text{ se } f^j(x) = c \\
  1 & \text{ se } f^j(x) > c. \\
\end{cases}
\]
\end{definition}

\begin{definition}
A sequência kneading de $f$ é a sequência $K(f) = S(f(c))$, ou seja, é o itinerário de $f(c)$.
\end{definition}

Seja $$\Sigma_C = \{ (s_0\, s_1\, s_2\, \dots) : s_n \in \{0, C, 1\} \text{ para todo } n \geq 0\}.$$ Se $s = (s_0\, s_1\, s_2\, \dots)$ e $t = (t_0\, t_1\, t_2\, \dots)$ são elementos de $\Sigma_C$, dizemos que $s$ e $t$ possuem discrepância $n$ quando $s_i = t_i$ para todo $0 \leq i < n$ e $s_n \neq t_n$. Além disso, definimos $\tau_n(s)$ como a cardinalidade do conjunto $\{ s_j : 0 \leq j < n \text{ e } s_j = 1 \}$. Com isso, podemos definir a seguinte ordem em $\Sigma_C$.

\begin{definition}
Sejam $s = (s_0\, s_1\, s_2\, \dots)$ e $t = (t_0\, t_1\, t_2\, \dots)$ elementos de $\Sigma_C$. Suponha que $s$ e $t$ possuem discrepância $n$. Dizemos que $s \prec t$ se alguma das condições abaixo é válida.
\begin{enumerate}
\item $\tau_{n-1}(s)$ é par e $s_n < t_n$.
\item $\tau_{n-1}(s)$ é ímpar e $s_n > t_n$.
\end{enumerate}
\end{definition}

O seguinte teorema nos mostra as semelhanças da ordem $\prec$ em $\Sigma_C$ com a ordem usual na reta real.

\begin{theorem}
Sejam $x, y \in [0, 1]$.
\begin{enumerate}
\item Se $S(x) \prec S(y)$, então $x < y$.
\item Se $x < y$, então $S(x) \preceq S(y)$.
\end{enumerate}
\end{theorem}

\begin{proof}
O primeiro item será provado por indução em $n$, onde $S(x)$ e $S(y)$ são possuem discrepância $n$ e, por contrapositiva, o segundo item segue imediatamente do primeiro.

Sejam $S(x) = (s_0\, s_1\, s_2\, \dots)$ e $S(y) = (t_0\, t_1\, t_2\, \dots)$ tais que $S(x) \prec S(y)$. Se $S(x)$ e $S(y)$ possuem discrepância $0$, então $s_0 < t_0$ e, portanto, $x < y$. Suponha que essa propriedade é válida quando as sequências possuem discrepância $n-1$.

Se $S(x)$ e $S(y)$ possuem discrepância $n \geq 1$, então $s_0 = t_0$ e as sequências 
$$S(f(x)) = (s_1, s_2, s_3, \dots) \textrm{\, e \,} S(f(y)) = (t_1, t_2, t_3, \dots)$$
possuem discrepância $n-1$. Vamos analisar os possíveis Se $s_0 = t_0 = 0$, então $S(f(x)) \prec S(f(y))$ pois a quantidade de elementos iguais à $1$ antes da discrepância permanece inalterada e portanto, $f(x) < f(y)$ por hipótese de indução. Sendo $f$ estritamente crescente em $[0, c)$, concluímos que $x < y$. Analogamente, se $s_0 = t_0 = 1$, então $S(f(x)) \succ S(f(y))$ pois a quantidade de elementos iguais à $1$ antes da discrepância é diminuída em uma unidade e, portanto, $f(x) > f(y)$ por hipótese de indução. Sendo $f$ estritamente decrescente em $(c, 1]$, concluímos que $x > y$. Por fim, se $s_0 = t_0 = C$, então $x = y = c$.
\end{proof}

\begin{lemma}
Seja $s = (s_0, s_1, s_2, \dots)$ e suponha que $s_i \neq C$ para todo $0 \leq i \leq n$. Então, o conjunto
$$\{ x \in I : S(x)_i = s_i \textrm{ para todo } 0 \leq i \leq n \}$$
é um aberto em $I$.
\end{lemma}

\begin{proof}
Seja $x \in I$ tal que $S(x)_i = s_i$ para todo $0 \leq i \leq n$ e, portanto, $f^i(x) \in [0,c) \cup (c, 1]$ para todo $0 \leq i \leq n$. Para cada $0 \leq i \leq n$, seja
\[ V_i = 
\begin{cases} 
  [0, c) & \textrm{ se } f^i(x) < c \\
  (c, 1] & \textrm{ se } f^i(x) > c \\
\end{cases}
\]
Observe que cada $V_i$ é aberto em $I$. Pela continuidade de $f^i$, cada $f^{-i}(V_i)$ é aberto em $I$. Definindo $V = \cap_{i=0}^n f^{-i}(V_i)$, temos que $V$ é aberto e se $y \in V$, então $S(y)_i = s_i$ para todo $0 \leq i \leq n$.  
\end{proof}

\begin{theorem}
Suponha que $f: I \to I$ é unimodal e $c$ não é periódico. Se $t = (t_0\, t_1\, t_2\, \dots)$ é uma sequência tal que $\sigma^n(t) \prec K(f)$ para todo $n \geq 1$, então $t$ é $f$-admissível.
\end{theorem}

\begin{proof}
Se $t = (0\, 0\, 0\, \dots)$ ou $t = (1\, 0\, 0\, \dots)$, então $t$ é $f$-admissível já que, nesse caso, $S(0) = t$ ou $S(1) = t$. Para mostrar os outros casos, considere os conjuntos 
$$L = \{ x \in I : S(x) \prec t \} \text{\, e \,} R = \{ x \in I : S(x) \succ t \}.$$
Vamos mostrar que $L$ é aberto em $I$. Uma prova análoga pode ser feita para mostrar que $R$ é aberto em $I$.

Seja $z \in L$ e denote $S(z) = s = (s_0 \, s_1 \, s_2 \, \dots)$. Como $s \prec t$, temos que $s \neq t$ e, portanto, $s$ e $t$ são $n$-discrepantes para algum $n \geq 0$, ou seja, $s_n \neq t_n$. Sendo $t_n \neq C$ para todo $n \geq 0$, temos que $t_n = 0$ ou $t_n = 1$. Suponha que $t_n = 1$. Uma prova análoga segue quando $t_n = 0$. Os valores possíveis para $s_n$ são 
\end{proof}












